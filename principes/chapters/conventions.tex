\chapter{Conventions}
\label{app:conventions}


\section{Présentation}


\noindent
Nous avons distingués deux catégories de pratique :
\begin{itemize}
	\item les exercices : ils ont pour but de faire travailler un point précis et ils n'ont pas nécessairement d'application martiale directe ;
	\item les techniques : elles sont souvent décrites dans une source et ont un objectif martial bien précis.
\end{itemize}
Autant que possible une description de l'objectif accompagne chaque exercice.
La marge de droite contient des labels permettant de mieux cerner le type d'exercice/technique :
\begin{itemize}
	\item symétrique : il est possible de réaliser l'exercice en inversant toutes les directions ;
	\item mains nues : l'exercice est réalisé sans armes ;
	\item solitaire : l'exercice peut être réalisé seul ;
	\item drill ;
	\item dangereux : l'exercice comporte des risques importants et il doit être réalisé avec prudence ;
	\item échauffement : l'exercice est utile comme échauffement.
\end{itemize}
% précision, structure, fluidité
Un label entre parenthèses indique une variation de l'exercice (typiquement si un exercice se fait sur une cible immobile on peut choisir de le faire sur un partenaire ou non).

Le vocabulaire important a été classé en trois catégories :
\begin{itemize}
	\item coup : toute action offensive ;
	\item garde : toute position défensive (que ce soit une action dynamique ou une position d'attente) ;
	\item définition : tout autre vocabulaire important qui n'est ni un coup ni une garde.
\end{itemize}


\section{Descriptions}


Un défi majeur qui se pose lors de la rédaction d'un manuel d'escrime est de décrire clairement les techniques de sorte qu'elle puisse être reconstituée (presque) uniquement – et ce d'autant plus lorsqu'il n'y a pas d'images pour appuyer le texte.
Pour cette raison il est important d'établir un certain nombre de conventions et de s'y tenir autant que possible.

Dans les techniques et les exercices chaque point correspond à un temps.
Ainsi même si aucune conjonction n'est précisée entre deux phrases, il faut comprendre que les deux actions doivent avoir lieu en même temps.
Exemple : « Faire un pas à droite, porter un estoc. » signifie que l'estoc a lieu en même temps que le pas à droite.
Dans certains cas les temps sont extrêmement décomposés, mais une exécution fluide de la technique conduira à "fusionner" certains temps.

Afin d'éviter toute ambiguïté, la personne qui effectue l'action sera systématiquement désignée par une lettre : \A pour l'attaquant et \D pour le défenseur.
Lorsqu'une technique implique plusieurs attaquants et/ou défenseurs un numéro est ajouter à la lettre: \An{1}, \An{2}, etc.
L'attaquant sera généralement l'agresseur, c'est-à-dire celui qui prend l'initiative de la première attaque.
Toutefois la distinction entre \A et \D est parfois arbitraire et l'assignation des lettres n'a pas de sens profond dans ce cas (par exemple dans le cas d'un exercice symétrique) ; il faut le voir comme un label.

Pour gagner en précision nous éviterons généralement d'indiquer les alternatives dans les descriptions : nous préférerons écrire une deuxième description, quitte à ce que la majeure partie de la description se répète (toutefois certains arbres de décision).
Toutefois nous utiliserons parfois des crochets afin d'indiquer une alternative à un temps donné, en particulier pour arrêter l'exercice à un temps donné.

Généralement les termes techniques spécifiques à une discipline ne seront pas traduits : cela permet d'être plus précis dans la description, d'autant plus lorsque le terme n'a pas de traduction immédiate en française (par exemple « winden »).
Cette convention est largement répandue et permet donc de faciliter les échanges avec d'autres groupes (éventuellement étrangers), en particulier lors des stages.

Nous utiliserons parfois un vocabulaire tiré de styles proches si le terme nous paraît le plus approprié.
Finalement quand nous nous efforcerons d'utiliser les termes dans la langue originale (éventuellement dans une version modernisée) quand un tel terme existe.
Nous emploierons aussi certains termes d'escrime moderne quand ceux-ci sont adaptés.

La position de départ est donnée au début de chaque technique/exercice.
De plus la position à chaque temps (avant ou après l'action) sera indiquée si celle-ci correspond à une position bien définie.
De plus si le côté d'une action n'est pas spécifié alors il s'agit du côté de la personne qui effectue l'action.
Par exemple « \A effectue un oberhau à gauche » signifie que l'oberhau est effectué du côté gauche de \A (et donc du côté droit de \D).
Parfois nous indiquerons la cible visée, auquel cas le côté sera celui de la personne qui subit l'action.
Le contexte devrait indiquer clairement de quel cas il s'agit, et par défaut le premier cas a préséance.

% Les rotations sont précisées avec les termes direct/antihoraire et indirect/horaire.
% 
% On utilisera le terme de croix pour une épée afin de désigner la croix formée par la poignée et la garde.
% Celle-ci servira de cible pour certains exercices (l'épée en nylon étant tenue par la lame).
