\chapter{Sécurité et législation}


\section{Sécurité}


Comme tout sport de combat les arts martiaux historiques sont dangereux et il est primordial de respecter un certains nombres de règles de sécurité.
En effet la majorité du temps nous utilisons des simulateurs très proches des armes historiques dont le but étaient de tuer (ou d'invalider sérieusement) l'adversaire.
Il convient donc de s'entraîner avec un certain sérieux et de la prudence.
Voici une base pour des règles à appliquer lors d'un entraînement :
\begin{itemize}
	\item Saluer la personne en face avant de commencer un exercice, et attendre qu'elle fasse de même.
	Cela permet d'identifier son partenaire et de s'assurer qu'il est prêt.
	De même saluer quand l'exercice est fini pour signifier à l'autre personne que vous arrêtez.
	
	\item Être toujours attentif à ce qui se passe autour de soi, que ce soit en se déplaçant dans la salle, en pratiquant un exercice ou en écoutant les explications des instructeurs (d'autres peuvent commencer à essayer le mouvement sans faire attention).
	Il ne faut jamais partir du principe que tout le monde fait attention en permanence à son arme.
	
	\item Pour se déplacer, ne pas courir et faire attention aux autres personnes qui s'exercent ; ne pas passer entre deux combattants.
	
	\item Garder la pointe de l'arme toujours dans le champ de vision en dehors des exercices (et idéalement vers le bas) -- surtout avec les armes longues (bâton, lance…).
	
	\item Ne pas jouer avec les armes en dehors des exercices, en particulier quand l'autre n'est pas protégé et qu'il y a du monde autour.
	A fortiori ne pas placer sa pointe à hauteur du visage ou porter un coup dans le dos de quelqu'un.
	
	\item Ne pas chercher à aller vite ou à frapper fort sur un exercice que l'on ne maîtrise pas.
	Plus l'on devient à l'aise et plus l'on peut accélérer, mais il faut adopter le rythme du plus lent des partenaires.
	
	\item En dehors des combats et pour les exercices sans protection, ne jamais toucher la cible (sauf indication contraire explicite).
	
	\item Vérifier son matériel avant tout exercice.
	
	\item Toujours garder son calme.
	
	\item Il est possible (et même souhaitable) de continuer à s'entraîner si l'on commence à être fatigué, mais il faut s'arrêter (jusqu'à la fin de la séance) à la moindre douleur musculaire (car un vrai problème est très long à se réparer).
\end{itemize}


\section{Législation}


Du point de vue de la loi française~\footnotemark{} les simulateurs utilisés dans les arts martiaux historiques sont des armes de catégorie D (armes blanches, armes historiques, armes à poudre noire et éventuellement le matériel d'archerie)~\cite[p.~249]{Morel:2008:ProtegorGuidePratique}.
\footnotetext{Décret N° 2013-700 du 30 juillet portant application de la loi N° 2012-304 du mars 2012.\\
	\url{https://www.legifrance.gouv.fr/affichTexte.do?cidTexte=JORFTEXT000027792819}}
L'acquisition et la détention de ces armes sont libres, le port (sur soi) est interdit tandis que le transport est réglementé.
En pratique cela signifie qu'il est possible d'acheter des armes sans problème et que l'on peut les stocker chez soi.
Le transport doit être fait de telle sorte que les armes ne soient pas facilement accessibles~\footnotemark{} et en ayant un « motif légitime » (se rendre à son cours, suite un achat, participation à une reconstitution historique).
\footnotetext{Par exemple les armes peuvent être rangées dans une housse avec fermeture, éventuellement fermée par un cadenas.
	On peut aussi enrouler la ceinture du fourreau autour de la garde de l'épée.}
Celui-ci est entièrement laissé à l'appréciation des forces de l'ordre.
Dans tous les cas il est important d'avoir avec soi sa licence.
Finalement il faut noter que le port d'une arme est valable lors des « reconstitutions historiques ».

Le transport ou le port non légitime d'une arme de catégorie D peut conduire à la confiscation de l'arme, une amende voire à une peine d'1emprisonnement.

% TODO: peut-on transporter des armes même un jour en dehors des cours ?


\section{Défense personnelle}


Il peut être tentant de vouloir se défendre lorsque l'on se fait agresser dans la rue, a fortiori si l'on a ses armes sur soi.
En réalité il s'agit d'une fausse bonne idée pour plusieurs raisons :
\begin{itemize}
	\item La pratique des arts martiaux (historiques ou non) ne préparent généralement pas à une confrontation en situation réelle (contrairement à l'auto-défense).
	Ainsi on risque de mal réagir et la personne en face peut être plus expérimentée.
	
	\item La confrontation peut mal tourner et l'introduction d'une arme eut conduire à une escalade de la violence : l'autre peut s'emparer de l'arme et la retourner contre nous (au lieu de simplement donner des coups de poing) ou encore cela peut l'inciter à utiliser une arme plus dangereuse (couteau, pistolet…).
	
	\item Juridiquement cela peut nous être reprochés car le motif de légitime défense indique que la réponse doit être proportionnée à l'agression.
\end{itemize}
Il vaut donc mieux privilégier la fuite ou obtempérer si celle-ci n'est pas possible (si le motif de l'agression est un vol).
Pour des informations supplémentaires sur la défense personnelle on pourra se reporter à~\cite{Morel:2008:ProtegorGuidePratique}.
