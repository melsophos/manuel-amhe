\chapter{Introduction à l'escrime}


\section{Construire les bases}


Dans cette partie introductive liée à la structure nous allons étudier les concepts qui se trouvent à la base de l'escrime~\footnotemark{}.
Certains sont plus compliqués que d'autres : dans ce cas le lecteur ne doit pas hésiter à continuer sa lecture et à pratiquer les exercices simples pour revenir plus tard sur les points difficiles.
\footnotetext{Mon approche générale de l'escrime a été fortement influencée par Jean-Paul Blond, Agnès Daipra, Thomas Mainguy, Romain Wenz.
	D'autres sources d'inspiration ont été Martin Enzi~\cite{enzi:dijon:messer_inner:2015} et Ingulf Kohlweiss~\cite{Kohlweiss:2014:Dijon:RingenSchwert}.}

Les idées de cette partie ne représentent pas la vérité absolue de l'escrime : pour chaque affirmation il est possible de trouver une situation qui la contredit.
L'objectif est plutôt de construire un socle commun à toutes les armes en donnant des principes qui fonctionnent globalement et qui permettent de limiter les erreurs tout en permettant d'acquérir certains automatismes.
Il s'agit d'une sorte de corde de sûreté qui permet de se retenir à des concepts qui ont fait leurs preuves.
Par la suite nous rencontrerons des situations où un principe ne sera pas respecté : cela ne signifie pas qu'il soit faux, mais plutôt que nous avons besoin d'agir autrement afin d'obtenir un effet donné.
Ainsi un escrimeur peut se permettre de ne pas respecter un principe quand il en comprend la raison.

On pourrait faire l'analogie avec l'apprentissage de la musique : on commence par répéter les mêmes gammes un grand nombre de fois, on travaille au métronome et on apprend les rythmes classiques.
Puis avec l'expertise on s'aperçoit que l'on peut se passer de tout cela car les principes de la musique ont été intériorisés, et l'on peut même prendre des libertés avec les règles afin d'obtenir certains effets.


\section{À la recherche des principes}


Les notions qui sont développées dans cette partie forment le cœur de l'escrime : une fois que ces principes généraux ont été intégrés -- au niveau corporel et non intellectuel -- il devient possible de s'adapter rapidement à n'importe quelle arme.
En effet l'utilisation d'une arme est globalement déterminée par l'interaction entre ses caractéristiques et le corps humain selon ces principes généraux : il faut apprendre à sentir l'arme et comment elle et le corps s'influencent mutuellement.
Les techniques sont un raffinement : elles aident tout d'abord à construire cette compréhension avant d'ajouter une finesse supplémentaire à l'exécution.
Les notions décrites dans cette partie peuvent se retrouver modifiées en fonction de l'arme étudiée (par exemple la même garde ne sera pas utilisée avec une rapière ou une lance) -- et cela sera expliqué en temps voulu --, mais le principe reste le même.
% ne pas se concentrer sur la forme

Il existe un débat pour savoir s'il vaut mieux se concentrer ses efforts sur l'apprentissage de quelques armes (deux ou trois par exemple) ou s'il est souhaitable d'apprendre un grand nombre d'armes.
Les partisans de la première option considèrent qu'il est très difficile de devenir très bon dans plusieurs armes car la pratique et l'étude des sources prennent beaucoup de temps, et il s'agit un argument tout à fait valable.
D'un autre côté nous pensons qu'il est important de s'initier à autant d'armes que possible et à pousser l'étude pour un sous-ensemble important afin de pouvoir pointer du doigt les principes généraux que nous mentionnons précédemment : en effet, plus l'on sera habitué à utiliser des armes variées plus l'on sera forcé d'extraire les idées générales de l'escrime.
On remarquera finalement que de nombreux systèmes d'escrime (Liechtenauer, Fiore, Katori Shinto Ryu) proposent l'étude d'un ensemble varié d'armes.

% “…learn art which decorates you [and] in combat exalts with honor. Wrestle, good grappler; lance, spear, sword and Messer valiantly wield and make useless in other’s hands.”
% Johannes Liechtenauer, quoted by Sigmund Ringeck, MS Dresden C487. C.1504-1519. Translated by Christian Trosclair. Folios 11r-11v.

% TODO: développer ; déplacer ailleurs ?
À terme l'intérêt est de développer une compréhension des principes généraux qui vont fonctionner avec n'importe quelle arme, juste en adaptant la distance.
Parmi ces principes nous pouvons indiquer~\cite{enzi:dijon:messer_inner:2015} :
% cf Katori
\begin{enumerate}
	% explosivité
	\item tension/détente : la réaction doit être rapide et explosive, puis on est relâché jusqu'à la prochaine action ;
	\item mouvements circulaires : la direction est donnée en partant d'un angle de \ang{90} par rapport à la trajectoire de l'autre ;
	\item équilibre/déséquilibre : après un mouvement le corps peut être déséquilibré et il veut alors retourner dans une meilleur position ; en profiter pour le laisser faire efficacement (voir aussi~\cite{guidoux:dijon:thibault:2015}).
\end{enumerate}
L'objectif n'est pas d'avoir une technique parfaite, car en combat réel on aura rarement l'angle idéal, etc.


\section{Philosophie}


L'objectif principal de l'escrime est de survivre au combat : le fait de vaincre son adversaire est totalement secondaire.
Il s'agit d'une différence importante avec l'escrime moderne qui « cherche la touche », avec l'idée que le premier qui a touché a gagné (pour simplifier).
L'esprit de l'escrime ancienne est qu'il ne sert à rien d'avoir tué l'autre si l'on est aussi mort : il faut d'abord se préserver, et seulement ensuite voir ce que l'on peut faire pour mettre l'autre hors d'état de nuire.
En particulier cela implique qu'il n'est pas nécessaire de tuer pour atteindre ce but : une blessure au bras peut tout autant empêcher le maniement d'une arme, de même que le désarmement ou le fait de conduire à la reddition.
Toute manière de mettre fin au combat est correcte~\footnotemark{}.
\footnotetext{Certains styles, associés à un contexte particulier, privilégient certaines possibilités comme étant plus nobles : par exemple dans un duel à la rapière entre deux personnes du même groupe social le fait de désarmer l'adversaire sera considéré comme plus juste.}


\section{Pistes d'apprentissage}


% sparring, techniques, exercices, katas, lecture des sources

Une manière de progresser est de se fixer des objectifs concrets, en particulier lors des exercices ouverts ou du combat libre.
Il ne faut pas non plus être découragé si l'on ne parvient pas à retenir toutes les techniques étudiées : le but -- et surtout au début -- n'est pas tant connaître par cœur toutes les combinaisons possibles mais plutôt d'habituer le corps à manier une arme et de construire certains réflexes.

Il est fréquent de rencontrer des étudiants qui vont dire que telle ou telle technique ne marche pas car il existe un contre efficace, éventuellement en accélérant beaucoup le geste.
Ceci n'est pas un bon état esprit pour aborder l'escrime car les exercices et les techniques proposés permettent de travailler un élément précis : ils sont créés de telles sortes à ce que l'accent soit mis sur un ou plusieurs points, avec pour objectif d'améliorer la compréhension de l'arme étudiée.
Il existe certainement des contres plus judicieux (surtout si l'on sait ce que l'autre est censé faire en avance ou que l'on va délibérément plus vite que lui) ou d'autres manières de procéder, mais alors on perd l'intérêt de faire cet exercice et le partenaire ne peut plus travailler dans de bonnes conditions.
Cela ne veut pas dire qu'il ne faut pas résister – au contraire il est intéressant de varier le degré de résistance pour progresser –, mais il ne faut pas changer totalement les lignes de la technique (sauf avec accord du partenaire quand l'on souhaite explorer d'autres possibilités).


\section{Vocabulaire}


Le vocabulaire employé dans les sections qui suivent, et en particulier pour les déplacements et les positions de garde, est en grande partie inspiré de l'escrime moderne (pour un glossaire de l'escrime moderne voir~\cite{FIE:2014:BrefsGlossairesLescrime}).
Cela tient du fait que celle-ci a développé un langage précis -- hérité de l'escrime historique -- qui peut s'avérer utile pour décrire certains mouvements.
