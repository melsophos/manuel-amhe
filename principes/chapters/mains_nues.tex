\chapter{Combat à mains nues}


Le combat à mains nues (ou lutte) décrit toute forme de combat rapproché où au moins un des deux adversaires ne possède aucune arme.
Cela peut donc inclure des situations symétriques, où les deux adversaires sont désarmés, et des situations asymétriques, où l'un des deux opposants possède une arme.
Il peut sembler étrange de s'entraîner à la lutte lorsque l'on pratique l'escrime, mais cela provient de l'hypothèse erronée que l'on aura toujours une arme à disposition.
En réalité il existe deux situations courantes où l'on pouvait être amené à se défendre à mains nues : tout d'abord si l'on perd notre arme pendant un combat (désarmement, arme brisée…), et dans un deuxième temps si l'on se retrouve dans une situation où l'on est amené à se défendre sans avoir une arme à portée (rixe en taverne, agression dans la rue, etc.).
% Dans la pratique des \textsc{Amhe} nous serons plus particulièrement intéressés par le premier scénario.
Une autre raison peut conduire à préférer le combat à mains nues : durant le combat il peut arriver que l'on soit trop près de l'adversaire pour pouvoir utiliser correctement notre arme (revoir les distances page~\pageref{conc:def:distances}).
Et même si l'on peut encore utiliser notre arme donner un coup de poing ou un coup de pied peut être adapté, que ce soit pour surprendre l'adversaire ou pour le blesser.
En effet quelle que soit la situation un coup de poing sur le nez ou un coup de pied dans le genou n'est jamais agréable pour celui qui le subit.

Une seconde raison plus pragmatique pour pratiquer la lutte lorsque l'on débute l'escrime est qu'il s'agit d'un excellent moyen d'améliorer sa structure.
En effet beaucoup de défauts qui passent inaperçus avec une arme deviennent évident aux corps-à-corps car il est beaucoup plus difficile de compenser une mauvaise mécanique.

Dans ce chapitre nous explorons les configurations symétriques : le combat à mains nues contre une arme sera étudié au chapitre suivant.

Avant d'aborder les techniques, un point de sécurité : les torsions rencontrés lors du combat au corps-à-corps peuvent être particulièrement et dangereuses.
Ainsi dès que la personne qui reçoit la cible demande de s'arrêter il faut le faire immédiatement (éventuellement en accompagnant la détente, pour pas que la personne ne se cogne).


%%%%%%%%%%%%%%%%%%%%%
\section{Généralités}
%%%%%%%%%%%%%%%%%%%%%


Pour commencer, un conseil général est de ne jamais finir le bras tendu après une frappe car il est beaucoup plus facile de faire une clé sur un bras tendu.
Donner des coups de poings permet de surprendre quelqu'un qui vient plutôt faire des prises.

Une mesure de sécurité est de toujours garder les doigts serrés (et surtout le pouce) lorsque l'on dévie ou bloque des attaques à mains nues : le risque est très grand de se tordre le pouce.
Ce n'est qu'une fois que le contrôle est réalisé que l'on peut ouvrir les doigts et exécuter une saisie.


\begin{exercice}

\A met une cible et \D vient donner un coup très rapide.

Utilité : si l'adversaire a l'habitude d'adopter une garde de boxe parce qu'il a des gants, venir mettre un coup de poings dans le poignets (cela fait très mal sans protection).

% Romain
\end{exercice}


Quand un adversaire veut donner un coup au visage, le réflexe naturel est de se couvrir la tête avec les mains~\cite{enzi:dijon:messer_inner:2015}.
Les exercices qui suivent explorent ce principe.
À chaque fois il y a une tension lors de la parade puis un relâchement, et une nouvelle explosion au moment de la suite de la riposte.


\begin{exercice}
\label{mains-nues:ex:enzi-1}

\begin{enumerate}
	\item \A donne un coup de poing au visage.
	\item \D se baisse (sans bouger les pieds) et se protège le visage du bras gauche, mains vers le haut.
\end{enumerate}

% \source{\cite{enzi:dijon:messer_inner:2015}}
\end{exercice}


\begin{exercice}
\label{mains-nues:ex:enzi-2}

\begin{enumerate}
	\item \A donne un coup de poing au visage.
	\item \D se baisse (sans bouger les pieds) et se protège le visage du bras droit, coude vers le haut.
	\item Écarter le bras de \A avec la main gauche et de la droite le frapper au visage.
\end{enumerate}

% \source{\cite{enzi:dijon:messer_inner:2015}}
\end{exercice}


\begin{exercice}
\label{mains-nues:ex:enzi-3}

\begin{enumerate}
	\item \A donne un coup de poing au visage.
	\item \D se baisse (sans bouger les pieds) et se protège le visage du bras droit, main vers le haut.
	\item \D contrôle le poignet de \A avec la main gauche et frappe au visage de la main droite.
\end{enumerate}

% \source{\cite{enzi:dijon:messer_inner:2015}}
\end{exercice}


\begin{exercice}
\label{mains-nues:ex:enzi-4}

\begin{enumerate}
	\item \A donne un coup de poing au visage.
	\item \D se baisse (sans bouger les pieds) et se protège le visage du bras droit, coude vers le haut.
	\item Éjecter le bras de \A avec un mouvement circulaire dans le sens horaire.
\end{enumerate}

% \source{\cite{enzi:dijon:messer_inner:2015}}
\end{exercice}
\bigskip

Les exercices suivants permettent de préparer l'épée longue italienne (en particulier le jeu court).


\begin{exercice}

\begin{enumerate}
	\item \A et \D sont en garde, dos de la main droite en contact.
	
	\item Quand il le sent, \A passe sur la gauche et vient poser son avant-bras contre l'épaule de \D.
	
	\item \D vérifie si \A est stable en le poussant un peu.
\end{enumerate}

À faire en se déplaçant.

\A doit veiller à ne pas se prendre le coude de \D en passant.
De plus \A ne doit pas vraiment tirer ou attraper le bras de \D afin de ne pas donner d'indications.

L'intérêt de l'exercice est d'entrer dans la distance quand \A sent une poussée de la part de \D.

% Source : Romain (ENS).
\end{exercice}


\begin{exercice}

\begin{enumerate}
	\item \A et \D sont en garde, dos de la main droite en contact.
	
	\item Au moment de son choix, \D exerce une poussée ou retire sa main pour mettre une cible.
	
	\item Si \D pousse, \A entre dans la distance comme dans l'exercice précédent. Si \D met une cible, \A vient toucher en fente.
\end{enumerate}

À faire en se déplaçant.

L'intérêt de l'exercice est que \A ne peut pas prévoir la réaction de \D : ainsi même s'il est décidé à entrer en lutte, il doit garder en seconde intention de frapper si \D lui en donne la possibilité.

% Source : Romain (ENS).
\end{exercice}


\begin{exercice}

\begin{enumerate}
	\item \A et \D sont en garde.
	\A attaque \D à l'épaule.
	
	\item \D laisse passer court en reculant un peu sa jambe avant.
	
	\item \D fait un pas à gauche et vient heurter le coude ou l'épaule droite de \A avec sa main gauche.
	
	\item \D avance le pied droit et vient frapper l'épaule gauche de \A avec sa main droite.
\end{enumerate}

À faire en se déplaçant.

Au temps (3) \D doit pousser dans une direction à 45°, sinon \A peut facilement changer de garde et se repositionner (a fortiori dans les cas extrêmes de \ang{0} et \ang{90} -- faire le test à l'épée longue).
De plus \D doit pousser au niveau du coude ou de l'épaule, mais pas entre, car l'effet de levier serait beaucoup trop faible et \A pourrait résister.

% Source : Romain (ENS).
\end{exercice}


\begin{exercice}

\begin{enumerate}
	\item \A donne un coup circulaire.
	
	\item \D vient stopper le coup en plaçant ses deux avant-bras contre son bras.
	
	\item \D frappe à l'épaule droite.
	
	\item \D saisit le bras de la main gauche et se décale en tirant (pour déséquilibrer) et frappe en même temps l'autre épaule de sa main libre.
\end{enumerate}
\end{exercice}


\begin{technique}
\begin{enumerate}
	\item \A attaque l'épaule gauche.
	
	\item \D vient couvrir avec son bras gauche et pivote sur ses hanches pour donner un coup de poing à l'épaule.
	
	\item \D avance ensuite en revenant frapper/contrôler de sa main gauche le bras de \A (au niveau du creux du coude) et frappe à nouveau.
\end{enumerate}

\D exécute le dernier mouvement jusqu'à être bien positionné pour enchaîner avec autre chose (comme une prise).
\end{technique}



\section{Coups de pied}


\index{coup!de pied}
Pour donner un coup de pied il faut en premier monter le genou (en gardant la jambe pliée), puis frapper en déroulant la jambe, en "fouettant".
Il en existe quatre types.


\begin{coup}[Coup de pied montant]

De face, lever le genou et donner un coup de pied montant
\end{coup}


\begin{coup}[Coup de pied latéral]

Pivoter sur la jambe avant et donner un coup de pied sur le côté.
\end{coup}


\begin{coup}[Coup de pied poussé]

De face, lever le genou et pousser avec le pied.
\end{coup}


\begin{coup}[Coup de pied balayé]

Lever le genou et effectuer un mouvement semi-circulaire avec le pied.
\end{coup}


Le coup de pied balayé est utilisé pour écarter l'arme de l'adversaire.

À chaque fois il faut être bien stable pour ne pas perdre l'équilibre et être prêt à enchainer. De même une fois le coup donné il faut revenir en garde (par exemple en avançant et en changeant de pied).


\section{Clés}


\begin{technique}[Torsion du poignet]
\label{mains:tech:torsion-poignet}

\A tend la main devant lui, paume vers le ciel.
\D vient placer ses pouces gauche et droit respectivement dans le prolongement de l'index et du majeur (sur les métacarpes), et les autres doigts sur la face antérieure du poignet.
Le but est d'imprimer une torsion au poignet (pour en principe le briser).
Deux variantes sont possibles :
\begin{itemize}
	\item appuyer avec les pouces pour plier le poignet en arrière, puis tourner les mains sur le côté ;
	\item maintenir l'avant-bras avec les doigts et tordre la main avec les pouces, tout en faisant descendre le bras.
\end{itemize}
Dans les deux cas il ne faut pas tirer ou pousser le bras, afin que seul la main bouge par rapport à l'avant-bras.

Dans le cadre d'un enchaînement, le défenseur peut aussi donner un coup de pied.

Avec de la pratique il est même possible d'exécuter la clé d'une seule main.

Attention : cette torsion et particulièrement puissante et \D doit arrêter dès que \A le demande.
\end{technique}


\begin{technique}

\begin{enumerate}
	\item \A donne un coup de poing du bras droit.
	
	\item \D esquive et couvre avec le bras gauche.
	
	\item \D attrape la main de \A avec sa main gauche et exécute une torsion du poignet (technique~\ref{mains:tech:torsion-poignet}).
\end{enumerate}

Au point 3) la torsion s'accompagne d'un mouvement des hanches.
\end{technique}


\section{Sources}


La \emph{Fleur du combat} de Fiore possède une section entière consacrée à la lutte~\cite{deiLiberi:Conan:2013:FleurCombat:Lutte}.


\subsection*{Remerciements}


Mon approche de la dague a été influencée Samuel Baumard et Romain Wenz.
