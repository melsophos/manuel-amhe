\section{Épée longue}


\subsection{Variations sur la distance}


\base{Cor Kronenburg~\cite{kronenburg:dijon:going_distance:2015}}

L'idée est d'exécuter toujours les deux mêmes mouvements – oberhau à droite puis attaque à gauche – en adaptant les détails en fonction de la distance.
Il s'agit d'une étude visant à améliorer la prise de décision.
De plus cela permet d'unifier plusieurs techniques de Liechtenauer qui ont l'air différentes prises une par une.


\begin{itemize}
	\item Notions : distances en escrime allemande : zufechten, fechten, kriegen, ringen (section~\ref{chap:épée-longue-liechtenauer}).
	\item Exercices~\ref{struct:ex:contact:frappe-épaules}, \ref{struct:ex:frappe-gauche-droite}
	\item Exercices~\ref{ex:frappe-dist:approche-croix-aleat}, \ref{ex:frappe-dist:approche-croix-aleat-garde} : \D avance et place une cible avec la garde de son épée, \A porte deux attaques, une dans chaque angle ; sans puis avec retour en garde de \A.
	\item Exercice~\ref{ex:frappe-dist:approche-double-frappe} (à gauche en changeant de pied) : \D approche, \A attaque quand il est à distance, \D recule et \A attaque à nouveau.
	\item Techniques~\ref{épée-longue:tech:dg-zufechten-abnemmen}, \ref{épée-longue:tech:dg-fechten-duplieren}, \ref{épée-longue:tech:dg-kriegen-absetzen}, \ref{épée-longue:tech:dg-armringen-einlauffen}, \ref{épée-longue:tech:dg-leibringen-einlauffen} : abnemmen, duplieren, absetzen, et einlauffen (simple et projection).
	\item Pratique : tout le monde en colonne face à \D, une personne \A s'avance et attaque, \D choisit une des cinq réactions précédentes, et \A doit réagir correctement.
\end{itemize}

