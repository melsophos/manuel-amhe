\chapter{Ateliers}

Le but de ce chapitre est de présenter diverses idées d'ateliers (plus ou moins originales).
Chacun fait appel à un ensemble de notions et est composé d'exercices de préparation suivis des techniques elles-mêmes.
À la fin il peut y avoir une mise en pratique plus poussée.


\section{Épée longue : variations sur la distance}
\label{app:ateliers:épée-longue-variations-distance}

Cet atelier est basé sur une présentation donnée par Cor Kronenburg~\cite{kronenburg:dijon:going_distance:2015}.
L'idée est d'exécuter toujours les deux mêmes mouvements – oberhau à droite puis attaque à gauche – en adaptant les détails en fonction de la distance.

\begin{itemize}
	\item Notions : distances en escrime allemande : zufechten, fechten, kriegen, ringen (section~\ref{sec:épée-longue:liechtenauer}).
	\item Préparation : exercices~\ref{struct:ex:contact:frappe-épaules}, \ref{struct:ex:frappe-gauche-droite}, \ref{ex:frappe-dist:approche-croix-aleat}, \ref{ex:frappe-dist:approche-croix-aleat-garde}, \ref{ex:frappe-dist:approche-double-frappe} (à gauche en changeant de pied).
	\item Techniques : techniques~\ref{épée-longue:tech:dg-zufechten-abnemmen}, \ref{épée-longue:tech:dg-fechten-duplieren}, \ref{épée-longue:tech:dg-kriegen-absetzen}, \ref{épée-longue:tech:dg-armringen-einlauffen}, \ref{épée-longue:tech:dg-leibringen-einlauffen}.
	\item Pratique : tout le monde en colonne face à \D, une personne \A s'avance et attaque, \D choisit une des cinq réactions précédentes, et \A doit réagir correctement.
\end{itemize}


\section{Messer : principes internes}


Cet atelier est basé sur celui de Martin Enzi donné à Dijon~\cite{enzi:dijon:messer_inner:2015}.

Il s'agit de développer les concepts décrits dans la section~\ref{sec:structure:général} à l'aide des exercices~\ref{mains-nues:ex:enzi-1}, \ref{mains-nues:ex:enzi-2}, \ref{mains-nues:ex:enzi-3} et \ref{mains-nues:ex:enzi-4}, afin d'appliquer des principes au messer (section~\ref{sec:messer:lekuchner}).
L'idée est de travailler d'abord à mains nues pour voir le réflexe naturel, et ensuite ajouter l'arme.

