\section{Structure}


\subsection{Déplacements -- leçons du kung-fu}


\auteur{Thomas Mainguy}


\begin{enumerate}
	\item Rappels
	\begin{itemize}
		\item marche : la jambe arrière pousse ;
		\item quatre changements de garde : avant/arrière, sur les côtés (page~\pageref{dep:def:changement-garde-avant}).
	\end{itemize}
	
	\item Décomposition du changement de garde en deux étapes (page~\pageref{def:texte:garde-kung-fu}) → plus de flexibilité (peut changer de direction au milieu).
	
	\item Exercice : pratique des différents changements
	\begin{itemize}
		\item avant (technique~\ref{att:tech:changement-garde-2-temps-avant}) : 1) pivot des hanches, poids sur la jambe avant, jambe arrière commence à avancer, 2) fin du mouvement ;
		\item latéral (technique~\ref{att:tech:changement-garde-2-temps-latéral}) : 1) pivot des hanches pour amener la jambe arrière sur le côté, 2) fin du mouvement.
	\end{itemize}
	
	\item Exercice~\ref{att:ex:marche-poing-détendu} : marche avant avec coup de poing, revenir détendu en garde pour absorber l'impact.
	
	\item Ajout du haut du corps : sentir que sur un changement de garde on peut effectuer deux mouvements.
	
	\item Exercice~\ref{att:ex:changement-garde-2-temps-vide} : faire des séries de changement de garde dans le vide.
	
	\item Exercice~\ref{att:ex:changement-garde-2-temps-test} : \A et \D sont légèrement hors distance, \A effectue le premier mouvement du changement de garde.
	Si \A n'est pas bien protégé \D attaque, sinon \A termine son mouvement.
	
	\item Exercice~\ref{att:ex:changement-garde-2-temps-projection} : \A donne un coup de poing à \D qui couvre du bras en avançant.
	\D passe la jambe derrière \D tandis que la main droite pousse sur la poitrine (projection).
	
	\item Techniques~\ref{att:tech:changement-garde-2-temps-prime-haute} et \ref{att:tech:changement-garde-2-temps-latéral-prime}
\end{enumerate}

