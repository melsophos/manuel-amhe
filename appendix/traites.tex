\chapter{Traités et maîtres}


\section{Tradition allemande}


\subsection{Liechtenauer}
\label{app:maitres:liechtenauer}

Johannes Liechtenauer~\cite{wiktenauer:liechtenauer} était un maître allemand du 13\ieme{} ou 14\ieme{} siècle.
Le sujet de son traité est l'épée longue et servit de base à toute une tradition.
Le texte est versifié et extrêmement difficile à comprendre.
Son étude se fait principalement à l'aide des explications données par quatre glossateurs : Sigmund Ringeck (entre 1438 et 1452), Peter von Danzig (1452), Juden Lew (vers 1450) et Hans von Speyer (1491).
Son œuvre est prédominante dans les AMHE.

Traductions :~\cite{ardamhe:tetraptyque, farrell:ringeck, lindholm:ringeck_longsword:2008}.

% Mscr.Dresd.C487 / Dresden, Sächsische Landesbibliothek
% Cod.44 A 8 (Cod. 1449) 1452 / Bibliotheca dell ́Academica Nazionale dei Lincei e Corsiniana
% Cod. I.6.4°.3 / Universitätsbibliothek, Augsburg
% Handschrift M I 29 / Universitätsbibliothek Salzburg

\subsection{Liegniczer}
\label{app:maitres:liegniczer}

Andre Liegniczer était un maître de la fin du 14\ieme{} ou du début du 15\ieme{} siècle~\cite{wiktenauer:liegniczer}.

Traductions :~\cite{ardamhe:liegniczer, lindholm:ringeck_others:2006} (voir aussi~\cite{youtube:sala_armi:liegniczer, youtube:memag:liegniczer}).


\subsection{Talhoffer}
\label{app:maitres:talhoffer}

Hans Talhoffer était un maître allemand du 15\ieme{} siècle~\cite{wiktenauer:talhoffer}.
Il a écrit au moins trois traités en 1443, en 1459 et en 1467.

Traductions :~\cite{gaurin:talhoffer:2005}.


\subsection{Lecküchner}
\label{app:maitres:lekuchner}

Johannes Lecküchner était un maître allemand du 15\ieme{} siècle (ca. 1430s–1482)~\cite{wiktenauer:leckuchner}.
Son œuvre est entièrement dédiée au messer.

% L'un des spécialistes du messer est Martin Enzi.

Traductions :~\cite{ardamhe:leckuchner}.


\section{Tradition italienne}

% TODO: Capo Ferro


\subsection{Fiore}
\label{app:maitres:fiore}

Fiore Furlano de'i Liberi (ca. 1340s–1420s) était un maître italien du 14\ieme{}.
Son œuvre \emph{Fior di Battaglia} (on peut aussi trouver \emph{Florius de Arte Luctandi} ou \emph{Flos Duellatorum}) est l'une des plus étudiées dans les AMHE.

Traductions :~\cite{conan:fiore, exiles:fiore:getty}.


\subsection{Vadi}
\label{app:maitres:vadi}

Philippo di Vadi était un maître italien du 15\ieme{} siècle~\cite{wiktenauer:vadi}.
Son traité s'intitulait \emph{De Arte Gladiatoria Dimicandi} (\emph{On the Art of Swordsmanship}) (ca. 1480) et couvre de nombreuses armes.

Traductions :~\cite{chaize:vadi, patrouix:vadi:2013, petit:vadi:longword}.


\subsection{Marozzo}
\label{app:maitres:marozzo}

Achille Marozzo (1484–1553)~\cite{wiktenauer:marozzo}, actif au 16\ieme{} siècle, est un des grands maîtres italiens de la tradition bolognaise.
Dans son libre l'\emph{Opera Nova} (\emph{A new work}) il expose de nombreuses armes : épée longue, épée de côté (avec et sans bocle), armes d'hast…


\subsection{Capo Ferro}
\label{app:maitres:capo_ferro}

Ridolfo Capo Ferro da Cagli était un maître italien du 17\ieme{} qui a écrit sur la rapière~\cite{wiktenauer:capo_ferro}.
