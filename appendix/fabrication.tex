\chapter{Fabrication}

\section{Écu XIII\ieme{}}

% http://modelclub.forumactif.fr/t889-bouclier-normand-1066
% http://www.guerriersma.com/contenu/Articles_tutos/bouclier/bouclier.htm
% 

Matériel :
\begin{itemize}
	\item bois contreplaqué ;
	\item anneaux ;
	\item cavalier (du type pour accrocher un grillage) ;
	\item boucle de sangle ;
	\item os à mâcher pour chiens ;
	\item boucles pour sangle ;
	\item trois lanières de cuir (50 cm, 1 m, 2 m) ;
	\item colle à bois ;
	\item toile de jute.
\end{itemize}
\bigskip

Étapes :
\begin{enumerate}
	\item Cintrer le bouclier.
	\item Découper la forme du bouclier.
	\item Dessiner la marque du bras sur l'arrière du bouclier, et indiquer où attacher les anneaux. Quatre forment un carré autour de la main, deux entourent l'avant-bras juste avant le coude.
	\item Fixer les anneaux avec les cavaliers.
	\item Replier les pattes des cavaliers, et poncer le bois et les pointes métalliques.
	\item Découper (grossièrement) la forme du bouclier dans la jute (deux ou trois épaisseurs, un peu plus grande que le bouclier). Idéalement découper les formes à 90° pour avoir un maillage dans différent sens (plus résistant).
	\item Placer une première couche de jute sur le bouclier, et l'accorcher sur le bord arrière autour (avec un pistolet à agraphes).
	\item Recouvrir la jute de colle à bois, et placer la seconde couche de jute. Agrapher les deux couches à l'arrière. Laisser sécher.
\end{enumerate}
\bigskip

La taille du bouclier et la position du bras dépendent de la morphologie : le haut du bouclier doit arriver sous le nez, et le bas au niveau du genou.
Le bras doit être placé environ à 45°.
