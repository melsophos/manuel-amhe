\section*{Introduction}
\addcontentsline{toc}{section}{Introduction}


Ce document contient des idées d'ateliers.
Certains ateliers ont été construits pour le Chapitre des armes ou le club de l'\textsc{Ens} tandis que d'autres s'inspirent d'ateliers présentés lors de stages.
Dans ce dernier cas des modifications auront été apportées pour adapter l'atelier aux concepts présentés dans le manuel et en fonction de notre interprétation ; dans tous les cas l'auteur original est indiqué.
Les ateliers sont parfois organisés par cycle.

Ce document sert de compagnon à l'introduction aux principes généraux ainsi qu'au recueil de techniques (qui reprend le premier).
En particulier toutes les références (page, définition, exercice, technique) se font par rapport à ce dernier : les liens (en cyan) permettent d'ouvrir le recueil directement à la bonne page si celui-ci est situé dans le même dossier que le ficher actuel.

Les remarques, critiques et suggestions sont extrêmement bienvenues et peuvent être soumises sur Github~\footnotemark{} (méthode privilégiée) ou envoyées par mail.
\footnotetext{\url{https://github.com/melsophos/manuel-amhe/issues}}

L'intégralité des sources Latex peut être téléchargées sur Github~\footnotemark{}.%
\footnotetext{\url{https://github.com/melsophos/manuel-amhe}}
De cette manière n'importe quelle partie du texte peut être réutilisée, par exemple pour un document plus spécialisé ou pour préparer un atelier (à condition de respecter la licence Art libre).
Les modifications entre deux versions peuvent être trouvées dans le fichier \texttt{CHANGES} qui se trouve sur Github.

% Chacun fait appel à un ensemble de notions et est composé d'exercices de préparation suivis des techniques elles-mêmes.
% À la fin il peut y avoir une mise en pratique plus poussée.

% TODO: durée approximative


\subsection{Mise en pratique}


\noindent
Quelques idées pour mettre en pratique les ateliers :
\begin{itemize}
	\item Commencer par exécuter les exercices/techniques en statique (sauf mention contraire) et, quand cela s'y prête, les faire en se déplaçant puis en donnant quelques frappes avant.
	
	\item Décomposer les techniques en morceaux.
% 	TODO: mettre une note quand il est bien de le faire
\end{itemize}

