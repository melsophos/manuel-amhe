\section{Structure et principes}


\subsection{Introduction}


Ce cycle vise à introduire les bases des arts martiaux historiques.
Du fait de la généralité des concepts étudiés il est possible d'utiliser n'importe quelle arme apparentée à une épée (épée à une main, épée longue, rapière) ou tout simulateur approchant (bokken, canne de combat).


\subsubsection{Principes 0}


Cette première séance a pour but de donner goût à l'escrime en enseignant les bases et en donnant quelques exemples de techniques.
Elle est relativement longue et peut être faite sur deux séances, où la première omet certaines techniques finales et où la seconde passe plus rapidement sur l'introduction.


\begin{enumerate}
	\item Consignes de sécurité (annexe~\ref{app:sécurité}) et organisation des cours.
	
	\item \emph{Position du corps} (définition~\ref{struc:def:position-garde}) : jambes fléchies, pied avant vers l'adversaire, pied arrière à \ang{30}--\ang{90}, pieds légèrement séparés, coudes non tendus, dos droit, buste tourné sur le côté, bassin basculé vers l'avant, épaules décontractées.
	
	\item \emph{Déplacements simples} (sans croisement) : le pied du côté où l'on va se lève et l'autre jambe pousse.
	Quand le premier pied touche le sol le second se lève et se rapproche.
	
	Exemples : \emph{marche}, \emph{retraite}, \emph{déplacement latéral} (définitions~\ref{dep:def:marche},~\ref{dep:def:retraite},~\ref{dep:def:marche-latérale}).
	
	\item Exercice~\ref{dep:ex:réalisation-déplacements} : les élèves se mettent en ligne, l'instructeur indique quand faire des marches/retraites.

	\item \emph{Typologie de l'arme} : poignée, garde, vrai/faux tranchant, fort/faible.
	
	\item \emph{Tenue de l'arme} : angle du poignet, rester assez souple.
	
	\item Exercice~\ref{dep:ex:déplacement-contact-meneur} avec arme : \A et \D sont en garde, les armes en contact.
	\A mène les déplacements et \D maintient la distance.
	
	\item \emph{Changement de garde avant} (définition~\ref{dep:def:changement-garde-avant}) : grâce à une rotation des hanches le pied arrière passe devant, le pied avant pivote.
	
	Idem avec le changement de garde arrière.
	
% 	\item attaque avec le poing,

% 	\item mains nues : coup de poing avec déplacement des pieds
	
	\item \emph{Effectuer une attaque} (définition~\ref{att:coup:attaque-normale}) : pied droit en arrière, coup depuis le côté droit.
	Déplacement lors de l'attaque, avancer l'arme avant le corps, frappe avec le dernier tiers.
	Insister sur la \emph{précision} et l'\emph{intention}, parler des \emph{doubles touches}.
	
	\item Exercice : \D se tient immobile, \A exécute des frappes diagonales à gauche et à droite.
	
	\item \emph{Esquive} : se déplacer en diagonale dans la direction d'où vient la frappe, éventuellement en se baissant.

	\item Exercice : \A et \D se font face.
		\A donne un coup de poing (gauche ou droit) et \D esquive.
	
	\item Technique : \D commence en garde basse, \A attaque à l'épaule, \D esquive et vient couvrir avec l'épée sur la lame de \A.
	\D attaque \A (penser au mouvement de hanche).
	
	\item Technique : \D est en garde (basse), \A attaque à l'épaule, \D esquive en se couvrant avec l'épée.
	\D attaque \A (penser au mouvement de hanche), \A se protège en ramenant son épée.
	\D change l'angle de son épée (pronation → supination) pour menacer \A.
	
% 	TODO: fonctionne avec une épée à deux mains ?
	\item Technique : \D est en garde, \A attaque à l'épaule gauche, \D interpose son épée (pointe vers le bas) en avançant vers \A (si l'arme est à une main elle vient en appui sur le bras gauche).
	\D utilise son bras gauche pour faire une clé à \A et le menace du pommeau ou du tranchant.
\end{enumerate}



\subsubsection{Principes 1 : structure, déplacements et attaques}


Durée : 2 heures.

% att:ex:série-garde

\begin{enumerate}
	\item Consignes de sécurité (annexe~\ref{app:sécurité}) et organisation des cours.
	
	\item \emph{Position du corps} (définition~\ref{struc:def:position-garde}) : jambes fléchies, pied avant vers l'adversaire, pied arrière à \ang{30}--\ang{90}, pieds légèrement séparés, coudes non tendus, dos droit, buste tourné sur le côté, bassin basculé vers l'avant, épaules décontractées.
	
	\item Exercice~\ref{struc:ex:test-position} : tester les positions faibles et fortes.
	\D adopte une position (jambes tendues/fléchies, de profil/face, etc.) et \A pousse pour tester la résistance.
	
	\item \emph{Déplacements simples} (sans croisement) : le pied du côté où l'on va se lève et l'autre jambe pousse.
	Quand le premier pied touche le sol le second se lève et se rapproche.
	
	Exemples : \emph{marche}, \emph{retraite}, \emph{déplacement latéral} (définitions~\ref{dep:def:marche},~\ref{dep:def:retraite},~\ref{dep:def:marche-latérale}).
	
	\item Exercice~\ref{dep:ex:réalisation-déplacements} : les élèves se mettent en ligne, l'instructeur indique quand faire des marches/retraites.
	
	\item \emph{Changement de garde avant} (définition~\ref{dep:def:changement-garde-avant}) : le pied arrière passe devant, le pied avant pivote.
	
	Idem avec le changement de garde arrière.
	
	\item Exercice~\ref{dep:ex:réalisation-déplacements} : ajout des changements de garde.
	
	\item Exercice~\ref{dep:ex:déplacement-contact-meneur} : \A et \D sont en garde, bras en contact.
	\A mène les déplacements et \D maintient la distance.
	
	\item Exercice~\ref{struc:ex:lutte} : lutte simple avec saisie aux bras.
	\A et \D sont en position de garde, \A saisit l'intérieur du coude gauche de \D avec sa main droite et l'extérieur du bras droit avec la main gauche, et \D fait de même.
	\A et \D se déplacent en essayant de diriger l'autre.
	Objectif : ressentir quelle position permet d'être le plus efficace.
	
% 	\item attaque avec le poing,
% 	Insister sur la précision et l'intention, parler des doubles touches.

% 	\item mains nues : coup de poing avec déplacement des pieds

	\item \emph{Typologie de l'arme} : poignée, garde, vrai/faux tranchant, fort/faible.
	
	\item \emph{Tenue de l'arme} : angle du poignet, rester assez souple.

	\item Types d'attaques (définitions~\ref{att:coup:taille}, \ref{att:coup:estoc}, \ref{att:coup:entaille}) : taille, estoc, entaille.
	On se concentre sur la taille.
	
	\item Effectuer une attaque (définition~\ref{att:coup:attaque-normale}) : pied droit en arrière, coup depuis le côté droit.
	Déplacement lors de l'attaque, avancer l'arme avant le corps, frappe avec le dernier tiers.
	
	\item 5 cibles : tête, haut/bas gauche/droite (attaques descendantes).
	
	\item Exercice~\ref{att:ex:série-5} : série de 5 avec partenaire immobile : bas droite (1), bas gauche (2), haut gauche (3), haut droite (4), tête (5).
\end{enumerate}


\subsubsection{Principes 2 : défenses}


\begin{enumerate}
	\item Tenue de l'arme : pronation et supination.

	\item 4 gardes pronation (tête, haut droite, bas gauche/droite), 1 garde supination (haut gauche).
	
	\item Exercice~\ref{att:ex:série-défense} : série de 5 avec partenaire qui se défend : bas droite (1), bas gauche (2), haut gauche (3), haut droite (4), tête (5).
\end{enumerate}


\subsection{Explorations}


\subsubsection{Déplacements -- leçons du kung-fu}


\auteur{Thomas Mainguy}


\begin{enumerate}
	\item Rappels
	\begin{itemize}
		\item marche : la jambe arrière pousse ;
		\item quatre changements de garde : avant/arrière, sur les côtés (page~\pageref{dep:def:changement-garde-avant}).
	\end{itemize}
	
	\item Décomposition du changement de garde en deux étapes (page~\pageref{def:texte:garde-kung-fu}) → plus de flexibilité (peut changer de direction au milieu).
	
	\item Exercice : pratique des différents changements
	\begin{itemize}
		\item avant (technique~\ref{att:tech:changement-garde-2-temps-avant}) : 1) pivot des hanches, poids sur la jambe avant, jambe arrière commence à avancer, 2) fin du mouvement ;
		\item latéral (technique~\ref{att:tech:changement-garde-2-temps-latéral}) : 1) pivot des hanches pour amener la jambe arrière sur le côté, 2) fin du mouvement.
	\end{itemize}
	
	\item Exercice~\ref{att:ex:marche-poing-détendu} : marche avant avec coup de poing, revenir détendu en garde pour absorber l'impact.
	
	\item Ajout du haut du corps : sentir que sur un changement de garde on peut effectuer deux mouvements.
	
	\item Exercice~\ref{att:ex:changement-garde-2-temps-vide} : faire des séries de changement de garde dans le vide.
	
	\item Exercice~\ref{att:ex:changement-garde-2-temps-test} : \A et \D sont légèrement hors distance, \A effectue le premier mouvement du changement de garde.
	Si \A n'est pas bien protégé \D attaque, sinon \A termine son mouvement.
	
	\item Exercice~\ref{att:ex:changement-garde-2-temps-projection} : \A donne un coup de poing à \D qui couvre du bras en avançant.
	\D passe la jambe derrière \D tandis que la main droite pousse sur la poitrine (projection).
	
	\item Techniques~\ref{att:tech:changement-garde-2-temps-prime-haute} et \ref{att:tech:changement-garde-2-temps-latéral-prime}.
\end{enumerate}

