\chapter{Combat libre}


% TODO: parler de Boorman
% \cite{boorman:dijon:applied_combatives:2014}

Le combat libre permet de mettre en pratique l'étude des techniques : cela permet de voir comment cela se passe en vrai, lorsque le partenaire n'est plus coopératif et que les actions se déroulent plus rapidement.
Ainsi une technique qui a l'air de marcher quand on l'essaie lentement sur un partenaire qui se place exactement comme il faut risque d'échouer lors d'un vrai combat : dans ce cas il faut revenir au travail de la technique et comprendre ce qu'il faut améliorer.
Ainsi le combat libre est une occasion unique de vraiment comprendre ce que l'on a étudié et de s'assurer que l'on n'est pas dans l'idéalisation intellectuelle.

Un risque important du combat libre est de se laisser entraîner par l'esprit de compétition et de vouloir gagner à tout prix.
Cela consiste souvent à chercher les touches faciles, voire à oublier de se défendre pour pouvoir porter une attaque plus létale -- même comme nous l'avons indiqué à plusieurs endroits l'objectif principal de l'escrime est de survivre, tandis que la "mort" de l'adversaire est un gain secondaire.
% pour un duel judiciaire les deux combattants passaient un très long moment à s'observer car ils ne pouvaient pas prendre le risque d'être touchés
Pour éviter de tomber dans ce jeu il faut se fixer des objectifs concrets (placer une technique particulière…), éventuellement en adaptant les règles du combat avec son partenaire.
Par exemple on peut limiter les touches à la tête, s'accorder pour réduire la vitesse, etc.
De cette manière on peut chercher à travailler des techniques plus avancées (et donc plus difficiles à réaliser) que l'on n'oserait pas utiliser dans un contexte purement compétitif.

\index{double touche}
L'un des fléaux du combat libre est la double touche : il s'agit du cas où les deux opposants se touchent soit en même temps, soit dans un intervalle de temps réduit.
Plusieurs raisons pour l'occurrence de double touches sont les suivantes :
\begin{itemize}
	\item les deux opposants veulent gagner à tout prix et cherchent à toucher sans penser à se protéger ;
	
	\item l'un des deux opposants est dominant et voit qu'il va gagner, tandis que l'autre tente un coup "désespéré" : si le gagnant se protège mal, il prend le coup alors qu'il n'aurait pas dû ;
	
	\item le gagnant hésite un peu trop pour effectuer l'action ce qui laisse le temps à l'autre de porter un dernier coup.
\end{itemize}
Les deux partenaires, autant le premier à toucher que le second sont à blâmer : il n'y a aucun honneur à avoir « quand même touché » si l'on est mort.

Il peut être nécessaire de réduire l'équipement afin que l'on hésite plus à se laisser toucher.
Il ne s'agit pas de se battre tout le temps sans protection, mais il faut réfléchir au degré de protection souhaiter : par exemple il n'est pas nécessaire de porter l'équipement complet pour un simple drill, alors qu'au contraire il est nécessaire lors d'un tournoi international où les gens risquent de frapper fort.
Dans tous les cas il ne sert à rien de prendre des risques inutiles en ne se protégeant pas assez.
Une autre raison pour avoir des protections correctes est de permettre à son partenaire de travailler sereinement : si l'on porte beaucoup moins de protections que lui alors il ne se comportera pas de la même manière par peur de nous blesser.


\section{Préparation}


\begin{exercice}

On forme une ligne d'attaquants face à un défenseur \D (plus ou moins) protégé.
Au signal, le premier de la ligne \A s'avance et dispose d'un temps limité pour porter une frappe sur \D (uniquement sur une zone protégée).
Ce dernier n'a que le droit de se défendre.
Si \A parvient à toucher \D, il s'arrête et se remet dans la file.

Le temps peut être plus ou moins court, une dizaine de seconde est correct, et il est possible de faire plusieurs files pour faire tourner plus vite.
Cet exercice prépare à la fois les attaquants à être rapide dans leur décision et dans leur attaque, mais aussi elle renforce les défenseurs en les forçant à être toujours sur la défensive et à être confronté à de nombreux styles.
Les attaquants peuvent être protégés ou non, s'il n'y a pas assez de masques, mais cela reste préférable.
\end{exercice}


\begin{exercice}
Même exercice que le précédent, mais cette fois-ci \D a le droit de riposter (là où est protégé \D).
Si \A est touché, il arrête le combat et se remet dans la file.

Pour augmenter la difficulté les zones autorisées peuvent être différentes (en particulier si les attaquants sont moins protégés).

Variante : au moment où le signal est donné, \D annonce la seule cible que \A a le droit de toucher.
\end{exercice}


\begin{exercice}
\begin{enumerate}
	\item \A et \D se tiennent à distance, e.g.\ hors d'un cercle.
	
	\item Ils avancent ensemble et \A porte une attaque.
	
	\item \D choisit une défense et une contre-attaque.
\end{enumerate}

Au début \A peut faire cinq fois de suite la même attaque, avant de varier.

% Source : Romain.
\end{exercice}


\section{Entrée en lutte}


L'idée de vouloir entrer en lutte peut venir du fait que l'adversaire a tendance à avoir les bras très tendus, ou encore à se pencher en avant.
Toutefois il ne faut pas essayer à tout prix de passer une technique d'entrée en lutte, car cela pourrait impliquer de passer à côté d'autres occasions pour frapper à l'épée, et aussi de prendre plus de risques.
% Ainsi sur le plan cognitif il faut rester prêt aux éventualités possibles.


\section{Variantes}


Voici quelques idées pour explorer de nouvelles situations en combat libre :
\begin{itemize}
	\item affronter $N$ personnes à la suite ;
	\item combat en groupe : $M$ contre $N$ personnes.
\end{itemize}

