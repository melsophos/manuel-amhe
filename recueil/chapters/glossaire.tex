\chapter{Glossaires}


Dans cette annexe nous donnons la traduction en français et en anglais~\footnotemark{} des termes techniques, ainsi que la définition de certains termes.
\footnotetext{L'anglais est souvent utiliser lors des stages.}


% TODO: italien, latin
% TODO: glossaire général français vers anglais

% faire un tableau par style ? plus simple pour s'y retrouver peut-être


\section{Escrime allemande}


\begin{table}[h]
	\centering
	\begin{tabular}{lll}
		allemand &
			français &
			anglais
			\\
		\hline
		hau (haw) &
			coup &
			strike
			\\
		wechselhau &
			coup changeant &
			changing strike
			\\
		mittelhau &
			coup intermédiaire &
			middle strike
			\\
		oberhau &
			coup supérieur (par en haut) &
			over strike (from above)
			\\
		unterhau &
			coup inférieur (par en bas) &
			under strike (from below)
			\\
		zwerchau (twerhau) &
			coup de travers &
			cross(wise) strike
			\\
		krucke &
			crosse &
			crutch
			\\
		stich &
			estoc &
			thrust
			\\
		schnitt &
			entaille &
			\\
		vor &
			avant &
			before
			\\
		nach &
			après &
			after
			\\
		indes &
			instant &
			meanwhile
			\\
		hut/leger &
			garde &
			guard
			\\
		ochs &
			bœuf &
			ox
			\\
		pflug &
			charrue &
			plough
			\\
		alber &
			fou &
			fool
			\\
		(vom) tag &
			du jour &
			from the day \\
		langort &
			longue pointe &
			long point
	\end{tabular}
	\caption{Glossaire allemand.}
	\label{app:tab:glossaire-allemand}
\end{table}


\section{Définitions}


% TODO: faire un vrai glossaire
\begin{description}
	\item[Pronation] La main est tournée de telle sorte que que la paume est dirigée vers le sol, et le dos vers le haut.
	\index{pronation|textbf}
	
	\item[Supination] La main est tournée de telle sorte que que la paume est dirigée vers le haut, et le dos vers le sol.
	\index{supination|textbf}
\end{description}

