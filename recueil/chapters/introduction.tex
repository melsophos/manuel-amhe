\chapter{Introduction}


%%%%%%%%%%%%%%%%%%
\section{Objectif}
%%%%%%%%%%%%%%%%%%


\index{\textsc{Amhe}}
\index{arts martiaux traditionnels}

Ce manuel est une compilation de réflexions, d'outils, de techniques et d'exercices pour pratiquer les \textsc{Amhe}.
Il s'agit avant tout d'un document de travail personnel en constante évolution.
Ainsi je m'en sers pour prendre des notes sur les cours/ateliers que j'ai suivis et sur les traités que j'ai étudiés.
Pour cette raison il n'y a pas de logique particulière derrière la présentation des différents styles et je complète au fur et à mesure, et en particulier les références aux sources manquent par endroit.
De plus les analyses que je présente sont personnelles (et en travail) et je ne prétends pas être suffisamment qualifié sur les styles dont je parle pour servir d'autorité.

Malgré tout je pense que ce recueil peut être utile : bien qu'il existe de nombreux livres décrivant tel ou tel style les ressources en accès libre sur internet manquent (on trouve généralement des transcriptions/traductions sans commentaires).
J'espère ainsi offrir des pistes pour aborder les différents styles, et en particulier je cherche à être aussi clair que possible dans la description des techniques afin qu'elles puissent être reproduites sans ambiguïtés.


%%%%%%%%%%%%%%%%%%%
\section{Ce manuel}
%%%%%%%%%%%%%%%%%%%


Ce recueil sert de compagnon au manuel associé sur les principes généraux et il en reprend tous les chapitres afin de réunir en un seul document l'ensemble des exercices et des techniques.
De même les ateliers proposés dans un autre document reprennent la numérotation de ce recueil.

Ce manuel a commencé à croître à partir des cours donnés au club d'escrime ancienne de l'\textsc{Ens} (Paris) par Romain Wenz, Thomas Mainguy et Samuel Baumard.
À cette base se sont ajoutées des idées provenant de mes recherches personnelles, de discussions, de stages et de lectures (blog, livres…).
Pour cette raison il est très difficile d'attribuer précisément l'origine de chaque idée et de donner des références exhaustives, en particulier à tous les articles de blog.
Un autre problème se pose lorsqu'une même technique est présentée par différents auteurs avec des variations : dans ce cas j'ai adopté l'interprétation qui me paraît le plus juste, parfois en faisant une synthèse des différentes approches.
Même lorsqu'un exercice ou une technique ne provient d'un autre auteur j'ai tâché de l'adapter en fonction de mon approche et du contenu de ce manuel : ainsi il ne s'agit plus de la version proposée originellement, si bien que l'auteur pourrait éventuellement ne pas être d'accord avec mon interprétation.
Finalement un dernier argument est de ne pas alourdir le texte.
Toutefois je me suis efforcé d'indiquer quand un instructeur m'a particulièrement inspiré pour le contenu d'une section.
% Malgré tout je me suis efforcé d'apporter des références précises aux divers auteurs dont je me suis inspiré, en particulier dans les cas suivants : description d'un exercice spécial ou d'un atelier, étude (technique, historique…) détaillée.
De plus la source historique est indiquée pour toute technique où j'ai pu l'identifier.

Les remarques, critiques et suggestions sont extrêmement bienvenues et peuvent être soumises sur Github~\footnotemark{} (méthode privilégiée) ou envoyées par mail.
\footnotetext{\url{https://github.com/melsophos/manuel-amhe/issues}}

L'intégralité des sources Latex peut être téléchargées sur Github~\footnotemark{}.
\footnotetext{\url{https://github.com/melsophos/manuel-amhe}}
De cette manière n'importe quelle partie du texte peut être réutilisée, par exemple pour un document plus spécialisé ou pour préparer un atelier (à condition de respecter la licence Art libre).
Les modifications entre deux versions peuvent être trouvées dans le fichier \texttt{CHANGES} qui se trouve sur Github.


%%%%%%%%%%%%%%%%%%%%%%%
\section{Remerciements}
%%%%%%%%%%%%%%%%%%%%%%%


Ce manuel doit beaucoup aux membres du Chapitre des armes, du club de l'\textsc{Ens} et du club de Katori, et en particulier aux personnes suivantes : Samuel Baumard, Jean-Paul Blond, Arthur Boutillon, Lionel Cambos, Agnès Daipra, Thomas Mainguy, Paul Melotti, Jan Orkisz, Serge Rajevic, Raphaël Rigal, Léo Vallet, Romain Wenz.
