\chapter{Rapière et dague}


Il existe principalement deux gardes à la rapière combinée à une dague : pied droit en avant, la main droite tient la rapière devant tandis que la main gauche est
\begin{itemize}
	\item soit montée au niveau de la tête et pointe la dague vers l'ennemi (le dos de la main se trouve du côté de la tête) ;
	\item soit à côté de la main droite, la dague étant alignée avec l'épée.
\end{itemize}

\noindent
Voici les différentes attaques simples :
\begin{itemize}
	\item fente pied droit en avant et estoc ;
	\item parade en prime en ramenant le pied droit en arrière, puis attaque sur la quinte (en faisant passer l'épée par la gauche, près de la dague) ;
	\item parade avec la dague (au niveau du torse, tenue vers le haut), en avançant le pied gauche, puis estoc en fente sur le le pied droit ;
	\item avancer le pied gauche en venant placer la dague devant l'épée, puis attaque du tranchant sur la prime
	en fente jambe droite en avant, et enfin appel de pied pour donner un estoc.
\end{itemize}
La dernière attaque peut être inversée.
