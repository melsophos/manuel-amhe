\chapter{Principes -- divers}


Nous proposons deux exercices qui permettent d'adopter facilement une garde correcte.


\begin{exercice}[Position des pieds après avoir marché]
\tags{solitaire}

\obj{Trouver la position correcte des pieds dans la position standard.}

Marcher naturellement et s'arrêter (pied droit devant).

Au moment de s'arrêter les pieds se trouvent naturellement dans une position de garde.
Il ne reste plus qu'à se baisser sur ses appuis.

\source{\cite{guidoux:dijon:thibault:2015}.}
\end{exercice}


\begin{exercice}[Trouver une bonne garde]

Se tenir pied joint, sauter en l'air et retomber jambes écartées (droite devant, gauche en arrière).

\source{\cite{enzi:dijon:messer_inner:2015}.}
\end{exercice}


\begin{exercice}
En position de garde, se pencher en avant et laisser le bras ballant du côté de la jambe arrière (type lancer de pétanque).
Grâce à un mouvement de hanches laisser le bras se balancer d'avant en arrière, puis de plus en plus vite jusqu'à faire des cercles.

% \source{\cite{enzi:dijon:messer_inner:2015}}
\end{exercice}
