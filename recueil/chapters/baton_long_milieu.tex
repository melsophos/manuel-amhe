\chapter{Bâton long -- prise milieu}


Les deux mains sont placées de manière symétrique par rapport au centre.
L'avantage de cette prise est de permettre une plus grande mobilité lors des parades et des frappes.
Par exemple il est possible de parer d'un côté et d'attaquer immédiatement de l'autre.

Les cibles sont les mêmes que pour la tenue longue.
De même il existe trois gardes, dont deux sont identiques à la prise longue.


\begin{garde}[Garde de la barrière]
\index{garde!bâton long}

Le bâton est tenu verticalement devant soi, (presque) posé sur le sol, les bras sont tendus.
\end{garde}


\begin{technique}

\begin{enumerate}
	\item Depuis la garde, \A ramène un peu ses bras vers lui pour faire passer son arme au-dessus de celle de l'adversaire afin d'attaquer le poignet, en faisant un appel de pied.
	
	\item \D, d'un même mouvement, ramène la partie avant de son bâton vers la gauche pour écarter le bâton adversaire et fait tourner son bâton en changeant de pied pour attaquer le coup avec la partie arrière.
	
	\item \A bloque verticalement en avançant, la partie arrière du bâton se retrouvant en haut.
	
	\item \A abaisse le bâton pour frapper le coude.
	
	\item \A frappe de l'autre côté (verticalement, donc le bâton reste du même côté et se retrouve sous le bras).
	
	\item \A frappe horizontalement la nuque en faisant un cercle horaire.
	
	\item \A frappe le cou verticalement.
\end{enumerate}

Pour parer le premier coup \D peut baisser son bâton pour le faire passer sous celui de l'ennemi afin de le ramener devant.
Lorsque \A bloque il peut le faire en reculant, auquel cas la parade la plus "sûre" se fait avec la partie arrière du bâton en haut.

% TODO: technique séparée
Une variante consiste à parer en reculant sans retourner le bâton, ce qui permet de se retrouver dans la bonne position pour insérer son bâton entre le bras et le bâton de l'adversaire, ce qui permet de le contrôler (torsion, désarmement, etc.).
\end{technique}


% à partir de la garde verticale, quand on se retrouve à l'intérieur de la garde adverse (i.e. entre lui et son bâton), il faut alors avancer rapidement, ce qui permet de casser le poignet et d'entrer en lutte

% une parade consiste à reculer en lâchant le bâton de la main avant, puis le faire tourner pour le saisir comme une lance puis à l'abaisser

