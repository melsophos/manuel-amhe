\chapter{Introduction à l'escrime}


\section{Construire les bases}


Dans cette partie introductive liée à la structure nous allons étudier les concepts qui se trouvent à la base de l'escrime.
Certains sont plus compliqués que d'autres : dans ce cas le lecteur ne doit pas hésiter à continuer sa lecture et à pratiquer les exercices simples pour revenir plus tard sur les points difficiles.

Les idées de cette partie ne représentent pas la vérité absolue de l'escrime : pour chaque affirmation il est possible de trouver une situation qui la contredit.
L'objectif est plutôt de construire un socle commun à toutes les armes en donnant des principes qui fonctionnent globalement et qui permettent de limiter les erreurs tout en permettant d'acquérir certains automatismes.
Il s'agit d'une sorte de corde de sûreté qui permet de se retenir à des concepts qui ont fait leurs preuves.
Par la suite nous rencontrerons des situations où un principe ne sera pas respecté : cela ne signifie pas qu'il soit faux, mais plutôt que nous avons besoin d'agir autrement afin d'obtenir un effet donné.
Ainsi un escrimeur peut se permettre de ne pas respecter un principe quand il en comprend la raison.

On pourrait faire l'analogie avec l'apprentissage de la musique : on commence par répéter les mêmes gammes un grand nombre de fois, on travaille au métronome et on apprend les rythmes classiques.
Puis avec l'expertise on s'aperçoit que l'on peut se passer de tout cela car les principes de la musique ont été intériorisés, et l'on peut même prendre des libertés avec les règles afin d'obtenir certains effets.


\section{À la recherche des principes}


Les notions qui sont développées dans cette partie forment le cœur de l'escrime : une fois que ces principes généraux ont été intégrés -- au niveau corporel et non intellectuel -- il devient possible de s'adapter rapidement à n'importe quelle arme.
En effet l'utilisation d'une arme est globalement déterminée par l'interaction entre ses caractéristiques et le corps humain selon ces principes généraux : il faut apprendre à sentir l'arme et comment elle et le corps s'influencent mutuellement.
Les techniques sont un raffinement : elles aident tout d'abord à construire cette compréhension avant d'ajouter une finesse supplémentaire à l'exécution.
Les notions décrites dans cette partie peuvent se retrouver modifiées en fonction de l'arme étudiée (par exemple la même garde ne sera pas utilisée avec une rapière ou une lance) -- et cela sera expliqué en temps voulu --, mais le principe reste le même.
% ne pas se concentrer sur la forme

% TODO: déplacer ailleurs ?
À terme l'intérêt est de développer une compréhension des principes généraux qui vont fonctionner avec n'importe quelle arme, juste en adaptant la distance.
Parmi ces principes nous pouvons indiquer~\cite{enzi:dijon:messer_inner:2015} :
% cf Katori
\begin{enumerate}
	\item tension/détente : la réaction doit être rapide et explosive, puis on est relâché jusqu'à la prochaine action ;
	\item mouvements circulaires : la direction est donnée en partant d'un angle de 90° par rapport à la trajectoire de l'autre ;
	\item équilibre/déséquilibre : après un mouvement le corps peut être déséquilibré et il veut alors retourner dans une meilleur position ; en profiter pour le laisser faire efficacement (voir aussi~\cite{guidoux:dijon:thibault:2015}).
\end{enumerate}
L'objectif n'est pas d'avoir une technique parfaite, car en combat réel on aura rarement l'angle idéal, etc.


\section{Vocabulaire}


Le vocabulaire employé dans les sections qui suivent, et en particulier pour les déplacements et les positions de garde, est en grande partie inspiré de l'escrime moderne (pour un glossaire de l'escrime moderne voir~\cite{FIE:2014:BrefsGlossairesLescrime}).
Cela tient du fait que celle-ci a développé un langage précis -- hérité de l'escrime historique -- qui peut s'avérer utile pour décrire certains mouvements.
