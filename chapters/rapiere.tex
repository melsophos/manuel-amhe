\chapter{Rapière}


%%%%%%%%%%%%%%%%%%%%%
\section{Généralités}
%%%%%%%%%%%%%%%%%%%%%


\begin{technique}

\A se laisse tomber au sol tout en mettant un estoc en mettant sa jambe gauche loin en arrière et en s'appuyant par terre avec sa main gauche.

La main droite doit se trouver au-dessus de la tête pour la protéger.

Source : Romain.

\end{technique}


%%%%%%%%%%%%%%%%%%
\section{Destreza}
%%%%%%%%%%%%%%%%%%


Le concept principal est celui de premier plan : il s'agit du plan qui joint les deux adversaires.
Le pied avant est toujours pointé vers l'adversaire.
Il y a parfois des postures bizarres (avec peu d'équilibre...) mais elles ne sont pas faites pour y rester.

Trois dégagements sont possibles pour reprendre le centre :
\begin{itemize}
	\item faire passer la lame sous celle de l'autre, avec un mouvement du poignet ;
	\item demi-taille : ramener l'épée en arrière et la faire passer par dessus l'autre dès que possible, encore avec le poignet ;
	\item comme précédent mais avec un tour complet par le bas, et là le bras est plus impliqué.
	% couronné ?
\end{itemize}


\begin{garde}

L'épée est horizontale, à hauteur de visage, les quillons perpendiculaires au sol.
La coque protège bien le visage.

\end{garde}


\begin{garde}

Le bras est un peu plié, la lame se trouve dans le prolongement de l'avant-garde.

\end{garde}

Cette seconde garde permet de placer le fort contre le faible.


\begin{technique}
\A et \D sont en garde 1.

\begin{itemize}
	\item \A passe en garde 2 afin de prendre le faible.
	\item \A passe le pied avant sur le côté où se trouve la lame de \D, et le tourne vers \D.
	\item \A ramène le pied arrière.
	\item \A fente pour estoquer \D.
	\item \D fait un dégagement en se déplaçant dans le même sens que \A.
	\item \D estoque.
\end{itemize}

Source : Jan.

\end{technique}


%%%%%%%%%%%%%%%%%%%%
\section{Capo Ferro}
%%%%%%%%%%%%%%%%%%%%

Le contrôle de la lame adverse est obtenu en étant très près de lui.
Ainsi tout attaque de termine en avançant vraiment sur l'autre.
Lever la main et estoquer vers le bas permet d'être bien couvert.

Pour dérouter l'adversaire on peut passer d'une ligne d'attaque haute à une ligne basse.
Par exemple en boxe : si \A donne des coups de poings sans s'arrêter, \D ne peut pas espérer contrer en frappant des poings.
Mais il peut donner un coup de pied là où l'attention de \A n'est pas, et après il peut frapper du poing.


\begin{technique}

\begin{enumerate}
	\item \A avance et estoque.
	\item \D passe sous la lame de \D si elle est à gauche, et vient battre en quarte, tout en faisant un quarter du pied.
	\item \D estoque le côté droit de \A en ayant la main haute.
\end{enumerate}

En combat libre, le coup du berger pour \A est d'attendre loin avant d'avancer en accélérant pour estoquer : la technique actuelle est un bon contre.

Source : Romain.

\end{technique}


\begin{technique}

\begin{enumerate}
	\item \A attaque \D à la tête avec un coup tranchant.
	\item \D se protège en quinte.
	\item \D fait passe sa pointe par dessus le bras de \A et laisse tomber sa pointe en appuyant sur la lame.
	\item \D monte sa pointe en baissant sa main et estoque \A à l'aisselle droite.
	\item Pour se protéger, \A peut exécuter la technique précédente.
\end{enumerate}

Au temps 3) la pointe de \D doit se trouver légèrement derrière la main de \A pour éviter les quillons, et avoir le fort bien placé. Mais il ne faut pas être trop loin non plus.

Au temps 4) \D doit garder un contact avec la lame de \A pour la contrôler, sinon il pourra caver. Il ne faut pas non plus appuyer trop sinon À aura de l'élan pour donner un coup de taille.

Une autre défense plus risquée est de décaler en quadrant en se protégeant en prime.
La parade franche en 2) semble du pain béni pour A, car il est très facile de reprendre le dessus.

Source : Romain.

\end{technique}


\begin{technique}

\begin{enumerate}
	\item \A est en garde type destreza et empêche \D d'avancer (plus grande allongé, etc.).
	\item \D fait une feinte en avançant comme pour un estoc.
	\item \D prend le contact en quarte (ou se protège si \A a réagi).
	\item \D décale à gauche et passe en seconde.
	\item \D baisse la main et estoque en montant.
\end{enumerate}

Source : Romain.

\end{technique}
