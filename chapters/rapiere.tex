\chapter{Rapière}


%%%%%%%%%%%%%%%%%%%%%
\section{Généralités}
%%%%%%%%%%%%%%%%%%%%%


La rapière~\footnotemark{} se tient à une main et est principalement une arme d'estoc, même s'il reste tout à fait possible de porter des coups de taille.
\footnotetext{Mon approche de la rapière a été influencée Romain Wenz ainsi que par Sébastien Romagnan~\cite{Romagnan:Dijon:2014:Destreza}.}


\begin{technique}

\A se laisse tomber au sol tout en mettant un estoc en mettant sa jambe gauche loin en arrière et en s'appuyant par terre avec sa main gauche.

La main droite doit se trouver au-dessus de la tête pour la protéger.

% Source : Romain.
\end{technique}


%%%%%%%%%%%%%%%%%%
\section{Destreza}
%%%%%%%%%%%%%%%%%%


Le concept principal est celui de premier plan : il s'agit du plan qui joint les deux adversaires.
Le pied avant est toujours pointé vers l'adversaire.
Il y a parfois des postures bizarres (avec peu d'équilibre...) mais elles ne sont pas faites pour y rester.
Sébastien Romagnan a écrit plusieurs livres de référence sur la Destreza~\cite{Romagnan:2013:DestrezaManuel, Romagnan:2016:DestrezaTailleRevers}.

Trois dégagements sont possibles pour reprendre le centre :
\begin{enumerate}
	\it
	\item faire passer la lame sous celle de l'autre, avec un mouvement du poignet ;
	\item demi-taille : ramener l'épée en arrière et la faire passer par dessus l'autre dès que possible, encore avec le poignet ;
	\item comme précédent mais avec un tour complet par le bas, et là le bras est plus impliqué.
	% couronné ?
\end{enumerate}


\begin{garde}

L'épée est horizontale, à hauteur de visage, les quillons perpendiculaires au sol.
La coque protège bien le visage.
\end{garde}


\begin{garde}

Le bras est un peu plié, la lame se trouve dans le prolongement de l'avant-garde.
\end{garde}


Cette seconde garde permet de placer le fort contre le faible.


\begin{technique}
\A et \D sont en garde 1.

\begin{enumerate}
	\item \A passe en garde 2 afin de prendre le faible.
	\item \A passe le pied avant sur le côté où se trouve la lame de \D, et le tourne vers \D.
	\item \A ramène le pied arrière.
	\item \A fente pour estoquer \D.
	\item \D fait un dégagement en se déplaçant dans le même sens que \A.
	\item \D estoque.
\end{enumerate}

% Source : Jan.
\end{technique}


%%%%%%%%%%%%%%%%%%%%
\section{Capo Ferro}
%%%%%%%%%%%%%%%%%%%%


Le contrôle de la lame adverse est obtenu en étant très près de lui.
Ainsi tout attaque de termine en avançant vraiment sur l'autre.
Lever la main et estoquer vers le bas permet d'être bien couvert.

Pour dérouter l'adversaire on peut passer d'une ligne d'attaque haute à une ligne basse.
Par exemple en boxe : si \A donne des coups de poings sans s'arrêter, \D ne peut pas espérer contrer en frappant des poings.
Mais il peut donner un coup de pied là où l'attention de \A n'est pas, et après il peut frapper du poing.


\begin{technique}

\begin{enumerate}
	\item \A avance et estoque.
	\item \D passe sous la lame de \D si elle est à gauche, et vient battre en quarte, tout en faisant un quarter du pied.
	\item \D estoque le côté droit de \A en ayant la main haute.
\end{enumerate}

En combat libre, le coup du berger pour \A est d'attendre loin avant d'avancer en accélérant pour estoquer : la technique actuelle est un bon contre.

% Source : Romain.
\end{technique}


\begin{technique}

\begin{enumerate}
	\item \A attaque \D à la tête avec un coup tranchant.
	\item \D se protège en quinte.
	\item \D fait passe sa pointe par dessus le bras de \A et laisse tomber sa pointe en appuyant sur la lame.
	\item \D monte sa pointe en baissant sa main et estoque \A à l'aisselle droite.
	\item Pour se protéger, \A peut exécuter la technique précédente.
\end{enumerate}

Au temps 3) la pointe de \D doit se trouver légèrement derrière la main de \A pour éviter les quillons, et avoir le fort bien placé. Mais il ne faut pas être trop loin non plus.

Au temps 4) \D doit garder un contact avec la lame de \A pour la contrôler, sinon il pourra caver. Il ne faut pas non plus appuyer trop sinon À aura de l'élan pour donner un coup de taille.

Une autre défense plus risquée est de décaler en quadrant en se protégeant en prime.
La parade franche en 2) semble du pain béni pour A, car il est très facile de reprendre le dessus.

% Source : Romain.
\end{technique}


\begin{technique}

\begin{enumerate}
	\item \A est en garde type destreza et empêche \D d'avancer (plus grande allongé, etc.).
	\item \D fait une feinte en avançant comme pour un estoc.
	\item \D prend le contact en quarte (ou se protège si \A a réagi).
	\item \D décale à gauche et passe en seconde.
	\item \D baisse la main et estoque en montant.
\end{enumerate}

% Source : Romain.
\end{technique}


%%%%%%%%%%%%%%%%%%%%%%%%%%%
\section{Thibault d'Anvers}
%%%%%%%%%%%%%%%%%%%%%%%%%%%


Il faut prendre une posture détendue~\footnotemark{}.%
\footnotetext{Cette section s'inspire d'Alexandre Guidoux~\cite{guidoux:dijon:thibault:2015}.}
Les déplacements se font en marchant normalement, la lame est tenue tranquillement.
L'arme se tient légèrement tournée par rapport à la prise classique, le plat de la lame est parallèle au sol, l'arceau se trouve à droite. Le pouce se trouve sur le plat de la lame au delà des quillons et forme un crochet avec l'index.
À la rapière la distance est bonne si le bout de la lame touche la coquille de l'autre quand les armes sont tenues parallèles au sol.


\begin{technique}

En garde pied droit devant, lames au contact.

\begin{enumerate}
	\item \A tend la rapière devant lui.
	\item \A estoque, tandis que \D écarte très légèrement la lame de \A et fait un léger pas sur la gauche (environ \SI{5}{cm}) tout en tendant le bras pour estoquer au visage.
\end{enumerate}

% \source{\cite{guidoux:dijon:thibault:2015}}
\end{technique}


\begin{technique}

En garde pied droit devant, lames au contact.
\D possède une rapière et une dague.

\begin{enumerate}
	\item \A tourne légèrement le poignet (pouce vers la gauche) et tourne dans le sens horaire autour de \D.
	\item Quand il se trouve à 90°, \A estoque \D au niveau de l'aisselle (ou de la poitrine droite).
\end{enumerate}

Il est très difficile de parer un coup à cet endroit avec la dague.
\A doit exercer une très légère pression, sinon \D aura envie de caver.

% \source{\cite{guidoux:dijon:thibault:2015}}
\end{technique}


\begin{technique}

\A est en garde pied droit devant, \D est pied gauche devant, les lames sont en contact.
\D possède une rapière et une dague, et les tient avec la pointe au même niveau.

\begin{enumerate}
	\item \A tourne légèrement le poignet (pouce vers la gauche) et tourne dans le sens horaire autour de \D.
	\item \D donne un coup de dague vers la droite pour chasser l'épée de \A.
	\item \A laisse sa lame partir en un cercle autour de sa tête, tout en tournant autour de \D par la droite (inversion du sens).
	\item Quand sa lame se trouve à sa droite, \A resserre les doigts et effectue un estoc montant.
\end{enumerate}

Pour un cercle efficace \A doit détendre ses doigts.

% \source{\cite{guidoux:dijon:thibault:2015}}
\end{technique}


\begin{technique}

En garde pied droit devant, lames au contact.
\D possède une rapière et une dague.

\begin{enumerate}
	\item \A tourne légèrement le poignet (pouce vers la gauche) et tourne dans le sens horaire autour de \D.
	\item \D écarte légèrement la lame de \A vers la gauche avec sa dague.
	\item \A avance légèrement et tend le bras pour menacer \D et le faire accentuer son geste d'écart.
	\item \A tourne autour de \D par la droite, en quartant du pied et lève la main pour faire un estoque descendant.
	\item \A ramène sa main vers le bas et pousse vers la gauche afin d'estoquer \D par en-dessous.
\end{enumerate}

% \source{\cite{guidoux:dijon:thibault:2015}}
\end{technique}


%%%%%%%%%%%%%%%%
\section{Divers}
%%%%%%%%%%%%%%%%


Les quatre techniques suivantes sont inspirées du messer~\cite{kleinau:dijon:rapier_messer:2015}.
Il s'agit de la variation d'une même technique où la conclusion dépend de :
\begin{itemize}
	\item de la distance entre \A et \D ;
	\item si \A possède ou non une dague.
\end{itemize}


\begin{technique}

\A n'a pas de dague et reste loin au temps 2).
\D est en prime.

\begin{enumerate}
	\item \A lance un oberhau et \D pare en quinte de pied ferme, tout en gardant la main gauche à proximité.
	
	\item \D vient saisir le poignet de \A.
	
	\item \D tire par le poignet et percute le coude avec le pommeau (côté ou extérieur).
	
	\item \D pousse \A (qui a le bras tendu et donc suit juste), et ensuite D franche.
\end{enumerate}

Au temps 3) pour bloquer l'articulation, \D peut aussi venir serrer avec le pommeau contre son avant-bras.

% Source : \source{Jan, d'après~\cite{kleinau:dijon:rapier_messer:2015}}
\end{technique}


\begin{technique}

\A n'a pas de dague et se trouve près au temps 2).
\D est en prime.

\begin{enumerate}
	\item \A lance un oberhau et \D pare en quinte de pied ferme, tout en gardant la main gauche à proximité.
	
	\item \D vient saisir le poignet de \A.
	
	\item \D remonte la main de l'autre au dessus de la tête voire derrière, puis recule bien la main pour estoquer visage.
\end{enumerate}

% Source : \source{Jan, d'après~\cite{kleinau:dijon:rapier_messer:2015}}
\end{technique}


\begin{technique}

\A a une dague et reste loin au temps 2).
\D est en prime.

\begin{enumerate}
	\item \A lance un oberhau et \D pare en quinte de pied ferme, tout en gardant la main gauche à proximité.
	
	\item \D vient saisir le poignet de \A.
	
	\item \D repousse \A, puis part vers la gauche et estoque sous le bras droit.
\end{enumerate}

% Source : \source{Jan, d'après~\cite{kleinau:dijon:rapier_messer:2015}}
\end{technique}


\begin{technique}

\A a une dague et se trouve près au temps 2).
\D est en prime.

\begin{enumerate}
	\item \A lance un oberhau et \D pare en quinte de pied ferme, tout en gardant la main gauche à proximité.
	
	\item \D vient saisir le poignet de \A.
	
	\item \D lève la main de \A puis se déplace sur la gauche pour trancher derrière le genou droit.
\end{enumerate}

% Source : \source{Jan, d'après~\cite{kleinau:dijon:rapier_messer:2015}}
\end{technique}

