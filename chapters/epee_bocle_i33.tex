\chapter{Épée et bocle -- I.33}


% type d'épée : Cinato : XVI, clubs allemands : XIV

% Royal Armouries, Tower Fechtbuch
Le traité MS I.33 (appelé aussi \emph{Walpurgis Fechtbuch} ou \emph{Liber de arte dimicatoria}) est le plus vieux manuel qui nous soit parvenu – il a été rédigé dans la décennie de 1320.
L'auteur serait un prêtre nommé Liutger.

La référence pour l'étude du I.33 est la traduction et le commentaire de Cinato et Suprenant~\cite{cinato:I33:2009}.
Kenner a écrit un manuel sur le sujet~\cite{kenner:I33:2014}.
Les images en couleur peuvent être trouvées sur le site de Wiktenauer~\cite{wiktenauer:I33}.


% TODO: custodia, contraria, liages supérieur/inférieur gauche/droit (inférieur = par en dessous)

\begin{definition}[Assiègement]
\index{assiègement}
\index{obsessio|see{assiègement}}
\index{contraria|see{assiègement}}

Un assiègement (lat. \emph{obsessio}~\cite{cinato:I33:2009}, \emph{contraria}~\cite{kenner:I33:2014}) est une position qui permet de briser une garde (lat. \emph{custodia}).

Il s'agit d'une position relativement hermétique et un assiègement adapté permet de se couvrir des attaques provenant de la garde adoptée par l'opposant.
\end{definition}

% liste des assiègement : krucke


\begin{definition}[Garde – \emph{Custodia}]
\index{garde!I.33}
\index{custodia|see{garde I.33}}

Une garde (lat. \emph{custodia}) est une position permettant de préparer une attaque.
Elle offre généralement une faible protection.
\end{definition}

% liste des gardes : 1 à 7

% TODO: déplacer en épée longue ?
\index{liage}
Il existe quatre liages possible : supérieur/inférieur et gauche/droite.
La personne qui lie est celle qui possède le centre et qui possède donc un avantage (plus ou moins important) sur l'autre.
Lorsque l'épée de celui qui lie est au-dessus on parle de liage supérieur, et de même si l'épée est en-dessous on parle de liage inférieur.
% cf Kenner

Le liage doit être franc en écartant vraiment la lame adverse vers le bas. Si on prend juste le centre sans bien éloigner la lame l'autre peut revenir facilement au moment où l'autre avance et attaquer gorge.

Rappelons aussi que les tranchants d'épées affûtées collent et il faut garder ceci à l'esprit en effectuant les mouvements (voir section~\ref{sec:armes-tranchantes:tranchant-collant}).


\begin{exercice}[Liages]

\begin{enumerate}
	\item \A fait un oberhau.
	\item \D vient prendre un liage.
	\item \A inverse le liage.
\end{enumerate}

L'oberhau est de hauteur variable pour encourager \D à varier les liages.
Le but n'est pas d'aller vite mais de sentir qui a le centre qui lie/est lié, de voir à quel point on est gêné et trouver le meilleur reliage.
\end{exercice}


\begin{exercice}
\A et \D sont en garde.
\D réunit ses deux mains devant lui (bras presque tendus) et \A vient placer son poing dedans.
\A essaie de repousser \D en poussant avec ses jambes.

Cet exercice doit être effectuer avec les deux mains.
Il prépare à donner des coups de bocles.

% Source : Thomas.
\end{exercice}


\begin{technique}

\A et \D démarrent au contact, pris en milieu de lame.

\begin{enumerate}
	\item \A prend le centre en liant l'épée vers la droite.
	\item \A donne un coup de bocle en avançant le pied gauche et vient trancher gorge.
\end{enumerate}

Test : en 1) \A ne repousse pas assez sur le côté, et alors en 2) \D revient au ventre et tranche gorge.

\source{\cite{fuhrmann:dijon:I33_liage:2015}}
\end{technique}


\begin{exercice}[Liage et \emph{Krucke}]
\label{épée-bocle:I33:ex:krucke-liage}

\A et \D démarrent au contact, pris en milieu de lame.

\begin{enumerate}
	\item \A prend le centre en liant l'épée vers la droite.
	\item \D accepte le liage et monte son coude (\emph{Krucke}).
	\item \A continue d'avancer pour voir si \D tient bien.
\end{enumerate}

En 2) le mouvement est bon si \D relâche le poignet pour faire tomber l'épée, tout en montant le coude (comme en prime).
\D ne peut être ferme que s'il a placé son pouce sur le bord de la fusée, sous les quillons.
\D met sa bocle au dessus de sa main au cas où.

\source{\cite{fuhrmann:dijon:I33_liage:2015}}
\end{exercice}


\begin{technique}

\A et \D démarrent au contact, pris en milieu de lame.

\begin{enumerate}
	\item \A prend le centre en liant l'épée vers la droite.
	\item \D accepte le liage (\emph{Krucke}).
	\item \A libère son tranchant et vient couper sur le plat de la lame de \D, en ramenant la main vers la gauche et en repoussant la lame.
	\item \A remonte les deux bras tendus dans l'ouverture pour frapper les mains de \D et estoquer au visage.
	\item Si \D réagit assez tôt, il abaisse les mains afin de bloquer la lame.
	\item \A fait un coup de bouclier et frappe au visage.
\end{enumerate}

En 3) \A casse son poignet. \A doit laisser sa pointe au centre.
Tout le long \A garde sa bocle bien sur sa main.
Au dernier temps si \A est bien placé il va ouvrir naturellement les bras de \D.

\source{\cite{fuhrmann:dijon:I33_liage:2015}}
\end{technique}


\begin{technique}

\A et \D démarrent au contact, pris en milieu de lame.

\begin{enumerate}
	\item \A prend le centre en liant l'épée vers la droite.
	\item \D accepte le liage (\emph{Krucke}).
	\item \D ramène son bras gauche en arrière.
	\item \D arrête de reculer mais laisse son épée continuer derrière.
		Sous l'ouverture créée il donne un coup de bouclier, tout en faisant pivoter la lame (paume vers l'intérieur).
	\item \D laisse l'épée continuer son chemin et donne un coup de faux tranchant sur la tête de \A grâce à un arc vertical.
\end{enumerate}

Cette technique ne marche que si le contact du fer est assez haut.
Note que le fait que les lames sont collées permet à \D de tirer \A vers l'arrière alors même que lui n'avance plus.

\source{\cite{fuhrmann:dijon:I33_liage:2015}}
\end{technique}


\begin{technique}

\A et \D démarrent au contact, pris en milieu de lame.

\begin{enumerate}
	\item \A prend le centre en liant l'épée vers la droite.
	\item \D accepte le liage (\emph{Krucke}).
	\item Tandis que \A continue d'avancer, \D décale son pied droit vers la droite et se retourner en entourant les bras de \A pour les bloquer.
\end{enumerate}

% Marche mieux que la technique précédente quand le contact est plus bas (?).

\source{\cite{fuhrmann:dijon:I33_liage:2015}}
\end{technique}


\begin{technique}

\A et \D démarrent au contact, pris en milieu de lame.

\begin{enumerate}
	\item \A prend le centre en liant l'épée vers la droite.
	\item \D accepte le liage (\emph{Krucke}).
	\item \D tourne les hanches et retourne le bras pour frapper sur la lame de \A.
\end{enumerate}

% Marche mieux que la technique avant la précédente quand le contact est plus bas (?).
Si cela ne marche pas, remonter pour trancher les bras.

\source{\cite{fuhrmann:dijon:I33_liage:2015}}
\end{technique}
