\chapter{Épée longue italienne}


\section{Italienne}


\begin{definition}[Mezza spada]

L'engagement au milieu de la lame est appelé mezza spada.
\end{definition}


\subsection{Général}


% poids typique : 1.7 kg

% Romain
Les parades se font toujours en se déplaçant sur le côté afin d'adoucir le choc : les épées étant lourdes il n'est pas possible de prendre une parade franche sans se déplacer.

% Romain
Une attaque peut être précédée d'un battement.
Ce battement peut envoyer l'épée dans la même direction que celle du déplacement, mais la position de l'attaque est telle que le défenseur ne peut pas revenir.

% Romain
L'escrime italienne à l'épée longue se fait typiquement en armure de plates.
Sur une amure trois zones absorbent très bien les chocs du fait de la largeur de la plaque de métal : les avant-bras et bras, les cuisses et le ventre (côtés compris).
Il est donc possible de rabattre volontairement l'arme adverse sur ces zones, cf les trois premiers techniques.
Cela provoque un effet de surprise, et de plus l'opposant perd l'amplitude qui donne de la force à sa frappe, ce qui augmente l'intérêt d'avancer vraiment.

% Romain
Quand la distance est faible et que l'on tient l'épée à une main, il faut avoir le dos de la main sur le dessus (pronation).
En effet si la lame est chassée dans cette position il sera beaucoup plus facile de revenir que si la main est tournée dans l'autre sens.

% Romain
Si \A et \D ont le fer en contact et que \A ne menace pas \D avec sa pointe en la gardant bien entre les deux, \D peut tourner ses poignets d'un côté en se déplaçant dans la même direction, ce qui permet d'estoquer.
Par exemple si son l'épée est placée pour attaquer l'épaule gauche de \A, \D tourne ses mains dans le sens anti-horaire – avec la main droite passant de supination à pronation – et va vers la droite.
De même si \A essaie de faire une entrée en lutte alors qu'il se trouve trop loin il est possible d'utiliser cette technique.


\begin{coup}[Estramaçon]
\index{coup!estramaçon}
\index{estramaçon}

L'estramaçon est un grand coup donné verticalement avec le tranchant.
\end{coup}


\begin{technique}

\begin{enumerate}
	\item \A attaque depuis sa droite l'épaule gauche de \D.
	\item \D pare en quarte en se déplaçant sur le côté droit.
	\item \D lâche sa main gauche et vient heurter la lame de \A avec son avant-bras (par-dessus) pour la rabattre sur son bras, en changeant de jambe.
	\item \D tourne sa main en pronation et vient placer sa main gauche sous la lame.
\end{enumerate}

Au point (3) il est important que \D s'accroupisse très bas et entre dans la distance.
De plus \D doit prendre garde à toujours garder sa pointe menaçante, bien au centre.

% Source : Romain, d'après Fiore.
\end{technique}


\begin{technique}

\begin{enumerate}
	\item \A attaque depuis sa droite l'épaule gauche de \D.
	\item \D pare en quarte en se déplaçant sur le côté droit.
	\item \D lâche sa main gauche et vient passer son bras entre les deux lames, pour ensuite enrouler son bras autour de la lame, en changeant de jambe. \D attrape le quillon (du bas) de \A.
\end{enumerate}

Cette technique est moins efficace que la technique précédente.
Il est important de tenir le quillon car sinon \A peut dégager sa lame en tirant dessus, et en profiter pour couper sous l'aisselle (sauf si \D est suffisamment près et menaçant avec sa pointe).

% Source : Romain, d'après Fiore.
\end{technique}


\begin{technique}

\begin{enumerate}
	\item \A attaque depuis sa droite l'épaule gauche de \D.
	\item \D pare en quarte en se déplaçant sur le côté droit.
	\item \D pose sa paume gauche sur le plat de sa lame et s'en sert pour venir abattre la lame de \A contre sa cuisse (à \D).
\end{enumerate}

L'intérêt de poser sa main sur sa propre lame est de ne pas prendre le risque d'attraper le tranchant de \A.

% Source : Romain, d'après Fiore.
\end{technique}


\begin{technique}

\begin{enumerate}
	\item \A attaque depuis la droite.
	\item \D pare en quarte puis monte en bœuf (mutation) pour estoquer.
	\item \A lève les mains, fait faire un tour avec son épée derrère lui et vient frapper derrière le genou droit de \D en faisant un grand saut sur le côté.
\end{enumerate}

L'idée pour \A est de finir juste à porter du genou, qui est situé un peu devant l'épaule, et ainsi \D est trop loin pour frapper.
Le coup doit arrivée derrière le genou car le côté est protéger par des disques de métal.

Le coup en (3) est l'équivalent italien d'un Zwerchau, qui est beaucoup plus ample.

% Source : Romain, d'après Marozzo.
\end{technique}


\begin{technique}

\begin{enumerate}
	\item \A attaque depuis sa droite.
	\item \D pare en quarte.
	\item \D vient en seconde.
	\item \D peut venir piquer la jambe ou, mieux, venir frapper au visage avec le faux tranchant.
\end{enumerate}
\end{technique}


\begin{technique}

\begin{enumerate}
	\item \A attaque avec un estoc.
	\item \D pare en quarte.
	\item \D chasse l'épée de \A vers la droite avec le faux tranchant et il finit en octave.
	\item \D laisse son épée aller en arrière et profite de l'élan pour venir frapper.
\end{enumerate}

Un bon estoc se fait en laissant d'abord tomber l'épée horizontalement, et ensuite en poussant la pointe.
Le pied se pose juste après l'impact, ce qui permet de pouvoir continuer d'avancer si l'autre recule.
Cette manière de faire est aussi plus précise.
% ça diminue le chance de coup double car l'autre voit que \A a décidé de prendre l'initiative

Au point (3) \D peut attendre que \A décide d'avancer/reculer, ou bien encore qu'il relâche un peu sa menace, pour lancer la suite.

% Source : Romain.
\end{technique}


Les trois techniques qui suivent sont efficaces contre un adversaire qui cherche à avoir la tête, et en particulier en combat libre où il y a de nombreux coups doubles.
Au dernier temps la position de la lame empêche \A de pouvoir toucher \D.
Dans tous les cas, au moment de la parade, \D doit maintenir sa pointe en direction du visage de \A pour garder le contrôle du centre.
Ainsi si \A recule \D peut le suivre facilement, et \A ne peut pas non plus attaquer.


\begin{technique}
\label{épée-longue:italien:fiore:tech:faux-tranchant-droite}

\begin{enumerate}
	\item \A attaque \D sur son épaule droite.
	
	\item \D pare avec le faux tranchant (en sixte).
	
	\item \D avance en diagonale vers la gauche et vient frapper \A avec le vrai tranchant.
\end{enumerate}

% Source : Romain, d'après Marozzo.
\end{technique}


\begin{technique}

\begin{enumerate}
	\item \A attaque \D sur son épaule droite ou sur la tête.
	
	\item \D pare avec le faux tranchant (en quinte, pointe vers la gauche).
	
	\item \D avance en diagonale vers la gauche et vient frapper \A à la tête avec le vrai tranchant, l'épée totalement dans l'axe.
\end{enumerate}

% Source : Romain, d'après Marozzo.
\end{technique}


\begin{technique}

\begin{enumerate}
	\item \A attaque \D sur son épaule gauche.
	
	\item \D pare avec le vrai tranchant (en quarte).
	
	\item \D avance en diagonale vers la droite et vient frapper \A avec le faux tranchant.
\end{enumerate}

% Source : Romain, d'après Marozzo.
\end{technique}


\begin{technique}[Coup de travers italien]

\begin{enumerate}
	\item \A menace \D dans le cadrant supérieur droit en passant en bœuf.
	
	\item \A déplace son pied gauche loin sur le côté mais en gardant son buste et son épée aux mêmes endroits.
	
	\item \A décroche son épée et exécute un coup de travers en ramenant son pied arrière, en ciblant l'arrière de la jambe de \D.
\end{enumerate}

Le fait de déplacer le pied gauche sans le reste du corps induit une torsion qui va donner la force au coup en rotation.

La première menace permet de fixer l'épée adverse, et il faut donc l'encourager à rester de ce côté en menaçant vraiment, sinon dès qu'il sentira que l'épée décroche il pourra revenir prendre le centre facilement.

% Source : Romain, d'après Marozzo.
\end{technique}


\begin{technique}

\begin{enumerate}
	\item \A et \D démarre en longue pointe.
	
	\item \A donne un coup vertical en visant la tête.
	
	\item \D se décale sur le côté et frappe en diagonale pour intercepter la lame de \A tout en le frappant à la tête.
\end{enumerate}

Comme \D se décale en second il a un léger avantage au niveau de l'axe.
La frappe de \D doit être diagonale pour briser la symétrie : si elle était aussi verticale alors il pourrait y avoir un coup double.

% Source : Romain.
% travail avec un bokken pour mieux sentir
\end{technique}


\begin{technique}

\begin{enumerate}
	\item \A attaque \D à la tête.
	
	\item \D se protège en se baissant, et monte son épée en tierce haute.
	
	\item \A monte les mains pour estoquer.
	
	\item \D passe sous la lame et en demi-épée, pour escroquer dans la visière.
\end{enumerate}

\D a besoin de l'ouverture fournie par \A pour briser la distance.
La position en 2) est bonne car si \A part \D peut donner un grand coup.

% Romain
\end{technique}


\begin{technique}

\begin{enumerate}
	\item \A attaque \D à la tête.
	
	\item \D se protège en se baissant, et monte son épée en tierce haute.
	
	\item \A reste menaçant en longue pointe.
	
	\item \D tourne autour de sa jambe droite et abat violemment son épée sur celle de \A, tout en lançant son pied gauche en arrière pour qu'il ne soit pas sur la trajectoire.
	
	\item \D revient avec la pointe devant pour estoquer.
\end{enumerate}

Variante : dernier coup en revers.
% Romain
\end{technique}


\begin{technique}

\begin{enumerate}
	\item \A se prépare à faire une attaque puissante (avec un appel, pour les besoins de l'entraînement).
	
	\item \D laisse passer court en reculant la jambe avant, et vient couvrir par dessus l'épée de \A.
	
	\item \D remonte son arme pour la placer contre le flanc de \A, sous son bras.
\end{enumerate}

% Romain
\end{technique}


\begin{technique}
Il s'agit d'une variante d la technique précédente (Marozzo).

\begin{enumerate}
	\item \A se prépare à faire une attaque puissante (avec un appel, pour les besoins de l'entraînement).
	
	\item \D laisse passer court en reculant la jambe avant, et donne un coup très puissant sur l'épée de \A.
	
	\item \D remonte verticalement pour frapper les bras -- idéalement l'intérieur des mains non protégés -- puis redescend en frappant à la tête.
\end{enumerate}

% Romain
\end{technique}


\begin{technique}

\begin{enumerate}
	\item \A décale le pied droit sur le côté droit, il avance les bras, puis il frappe en ramenant le pied gauche.
	
	\item Si \D défend, alors \A passe en bœuf en continuant de se décaler vers la droite, et estoque sous/sur le bras.
\end{enumerate}

Variante : \D avance.

% Romain
\end{technique}


\begin{technique}

\A et \D sont en garde en contact.

\begin{enumerate}
	\item \A monte en bœuf.
	
	\item \D frappe \A au ventre avec une attaque montante, du côté de la garde.
\end{enumerate}

Application : contre un adversaire qui monte en bœuf sans raison en étant près.
Il faut viser le ventre et le bras afin de le gêner au maximum. Se placer du côté des mains permettent d'être à un endroit où \A a moins de visibilité.

% Romain
\end{technique}


\begin{technique}

\begin{enumerate}
	\item \A attaque (estoc par exemple).
	
	\item \D recule sa jambe en avant et pare en prime (pointe menaçante).
	
	\item \D enroule et frappe à la tête.
\end{enumerate}

% Romain
\end{technique}


\begin{technique}

\begin{enumerate}
	\item \A attaque sur le côté droit.
	
	\item \D pare en seconde en reculant, fait un battement et vient estoquer.
\end{enumerate}

% Romain
\end{technique}


\begin{technique}

\begin{enumerate}
	\item \D fait une parade en sixte, décale le pied à gauche et frappe tête.
\end{enumerate}

Exemple : fréquent en parade instinctive.

% Romain
\end{technique}



\subsection{Fiore}


En général Fiore va au contact de l'épée~\cite{campo:dijon:posta_frontale:2015}.
Ainsi sur un oberhau dirigé sur sa gauche, Fiore conseille de venir en posta frontale en faisant un pas à gauche.
Une explication est que cela permet de gagner du temps : sinon il faudrait aller à droite en partant, puis repartir à gauche ; là on arrive directement à gauche.
La garde se prend comme si on visait les mains (cf un blocage à mains nues).
Il faut prendre garde à ne pas être accessible à une mutation de l'autre.
Si l'on prend bien la garde et que l'autre tape fort, son faible ira dans les quillons.
% Thomas
Si Fiore cherche le contact c'est pour pouvoir dégager la lame de l'autre.
Si l'autre est fort alors cela permet de prendre le liage (sachant que s'il est assez fort pour résister il ne pourra pas passer en dessous).


\begin{technique}

\begin{enumerate}
	\item \A fait un oberhau à droite.
	\item \D pare en posta frontale en se décalant à droite.
	\item \D lève le bras gauche pour placer sa pointe dans le visage de \A.
\end{enumerate}

Au temps 3) il ne s'agit pas d'un bœuf.

Si \A attaque à gauche la technique fonctionne en symétrique.

\source{\cite{campo:dijon:posta_frontale:2015}}
\end{technique}


\begin{technique}
\label{épée-longue:italien:fiore:tech:posta-frontale-faux-tranchant}

\begin{enumerate}
	\item \A fait un oberhau à droite.
	\item \D pare avec le faux tranchant en posta frontale en se décalant à gauche, en passant sous l'épée.
	\item \D tranche les bras de \A en avançant.
\end{enumerate}

Si \A attaque à gauche la technique fonctionne en symétrique.

La différence avec la technique~\ref{épée-longue:italien:fiore:tech:faux-tranchant-droite} est le côté de la première coupe.

% Romain
Une interprétation est la suivante : \A espère que \D va prendre une parade franche en quinte pour enchaîner sur un estoc.
\D fait croire ça en prenant une quinte avec le faux tranchant, mais il se baisse pour esquiver sur le côté gauche, passe en sixte puis frappe.

\source{\cite{campo:dijon:posta_frontale:2015}}
\end{technique}


\subsection{Vadi}


\subsubsection{Jeu court}


Dans cette section nous présentons les techniques de jeu court de Vadi~\cite{Vadi:Petit:2013:EpeeLongue}.
% La majorité des techniques qui suivent ont été présentées par Gautier Petit lors du stage Dijon 2015~\cite{petit:dijon:close_longword:2015}
% Prix Philippe Errard


\begin{exercice}
\A et \D sont en contact pointe longue.
\A pousse \D vers l'arrière en essayant de garder le centre.

\source{\cite{petit:dijon:close_longword:2015}}
\end{exercice}


\begin{technique}

\begin{enumerate}
	\item \A fait un estoc.
	\item \D dévie en tierce et se déplace sur la droite.
	\item \D avance la jambe gauche et estoque.
\end{enumerate}
\end{technique}


\begin{technique}

\begin{enumerate}
	\item \A fait un estoc.
	\item \D esquive vers la gauche et couvre avec son épée.
	\item \D percute avec sa main gauche le bras de \A (au niveau du coude) pour dégager la ligne.
	\item \D laisse revenir son épée pour porter un coup de tranchant ou il donne un coup de pommeau.
\end{enumerate}

En 4) le retour d'épée est très naturel.
En 3) \A doit bien tourner les hanches pour que le dégagement soit efficace.

Variante : percussion avec l'avant bras.
\end{technique}


\begin{technique}

\A est en dent du sanglier.

\begin{enumerate}
	\item \A fait un estoc en feinte pour forcer \D à réagir.
	\item \D fait un oberhau.
	\item \D enchaîne sur un estoc.
	\item \A avance et pare en prime haute.
	\item \A lâche sa main gauche et entoure les bras de \D avec son bras.
	\item \A soulève sa main pour sa clé, et peut frapper l'épée.
\end{enumerate}
\end{technique}


\begin{technique}

\A et \D sont en mezza spada.

\begin{enumerate}
	\item \A pince la lame avec la main gauche, paume vers la gauche, pouce en bas.
	\item \A fait un tour sec pour écarter la lame.
	\item Si \D résiste, À donne un coup de pied dans la jambe.
\end{enumerate}
\end{technique}


\begin{technique}

\A et \D sont en mezza spada.

\begin{enumerate}
	\item \A lâche sa main gauche, et chasse l'épée de \D en crochetant avec les quillons, en avançant pied gauche.
	\item \A prend sa lame en demi-épée et appuie sur le torse de \D pour le faire tomber.
\end{enumerate}

En principe la main gauche est à gauche, la droite de l'autre côté, mais l'inverse peut marcher.
Au moment du chasser on a le plat de la lame contre la main.
\end{technique}


\begin{technique}

\A et \D sont en mezza spada.

\begin{enumerate}
	\item \A lâche sa main gauche, et chasse l'épée de \D en crochetant avec les quillons ou le pommeau, en avançant pied gauche.
	\item \A passe sa jambe droite derrière \D, et attrape son poignet (étranglement) ou son épée (projection).
	\item \A fait tomber \D par terre.
\end{enumerate}
\end{technique}


\begin{technique}

\A et \D sont en mezza spada.

\begin{enumerate}
	\item \A pince la lame avec la main gauche, paume vers la droite, pouce en haut.
		\A abaisse l'épée de \D.
	\item \A lance son épée sur \D.
	\item En profitant du mouvement de recul de \D, \A passe son pied droit derrière et balance \D au sol.
\end{enumerate}

Pour travailler cette technique sans abîmer les armes, poser l'épée par terre.
\end{technique}


\begin{technique}

\begin{enumerate}
	\item \A pince l'épée de \D.
	\item \A lâche son épée et avance pour attraper la fusée de \D avec la main droite.
	\item \A se retourne, jambe gauche contre \D, et se sert de son épée pour le mettre à terre.
\end{enumerate}
\end{technique}


\begin{technique}

\begin{enumerate}
	\item \A passe en demi épée (une main de chaque côté de l'épée de \D) en se déplaçant à gauche et plaque l'épée de \D sur son épaule.
	\item \A appuie sur la lame de \D vers le bas et l'avant afin de l'amener la terre et de le faire lâche.
	\item Si \D ne lâche pas, \A peut donner un coup de pommeau.
\end{enumerate}
\end{technique}


\begin{technique}

\begin{enumerate}
	\item \A va attraper la fusée de \D avec la main gauche, pouce vers le bas.
	\item Au moment où \A passe sa main, \D passe la sienne pour attraper la main de \A (entre autres le pouce).
	\item \D tord la main de \A.
\end{enumerate}

En fait il vaut mieux briser la symétrie (bon avis général, sauf si un est beaucoup plus fort que l'autre).
La technique peut passer en symétrique grâce au temps d'avance de \D, mais ce n'est pas évident.
\D peut aussi se décaler à droite et appuyer avec le pommeau.
\end{technique}

