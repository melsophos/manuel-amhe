\chapter{Mains nues contre armes}


Dans ce chapitre nous explorons le combat à mains nues quand l'opposant est armé.
Ce type de situation est particulièrement dangereux et requiert une grande attention car l'adversaire est fortement avantagé, ne serait-ce que par la distance.


%%%%%%%%%%%%%%%%%%%%%%%%%%%%%%%%%%%
\section{Mains nues contre couteau}
%%%%%%%%%%%%%%%%%%%%%%%%%%%%%%%%%%%


% TODO: dire que la section est uniquement sur la dague, et dans la section sur le couteau; éventuellement déplacer certains dans la partie sur le close-combat

% TODO: référence au chapitre sur la dague
La majorité des techniques présentées dans cette section sont adaptées aussi à la dague qu'au couteau : l'avantage de penser aux techniques en terme de couteau oblige à se méfier du tranchant (les dagues n'étaient pas tranchantes).

Dans son traité \emph{La Fleur du combat} Fiore traite du combat à mains nue contre une dague~\cite{deiLiberi:Conan:2014:FleurCombat:Dague}.

Nous commençons par décrire des techniques générales.


\begin{exercice}

\begin{enumerate}
	\item \A donne un coup de couteau au ventre de \D.
	
	\item \D esquive sur le côté en venant couvrir (main droite ou gauche).
\end{enumerate}
\end{exercice}

% Marrozzo : accompagne le bras dans son mouvement


\begin{technique}

\begin{enumerate}
	\item \A donne un coup de couteau au ventre de \D.
	
	\item \D esquive sur le côté gauche et vient contrôler l'arme de la main gauche.
	\D se retrouve en garde face à \A.
	
	\item \D a plusieurs solutions pour poursuivre, par exemple :
	\begin{enumerate}
		\item frapper du poing droit dans le flanc ;
		
		\item frapper \A sous le menton avec la paume droite (passer sous son bras) ;
		
		\item tourner la tête de \A vers sa gauche en appuyant sur le côté droit de son visage avec la paume de la main droite (passer sous son bras), et passer la jambe droite derrière \A pour le projeter au sol.
	\end{enumerate}
\end{enumerate}

L'intérêt de contrôler l'arme de la main gauche est d'avoir la main droite à distance de frappe, ainsi que pour généraliser au cas où la main droite tient une arme.

Au point (3a) il est important d'engager l'épaule pour protéger le visage des coups de \A.

% Source : Romain.
\end{technique}


\begin{technique}

\begin{enumerate}
	\item \A donne un coup de couteau au ventre de \D.
	
	\item \D esquive sur le côté gauche et vient contrôler l'arme de la main droite.
	
	\item \D avance sa jambe gauche et place sa main gauche dans le creux du coude (droit) de \A pour plaquer le bras contre son corps.
	
	\item \D repousse en arrière le bras de \A grâce à sa main droite, tandis que sa main gauche vient attraper son propre coude droit afin de verrouiller la prise.
\end{enumerate}

Au point (4) \D doit tenir son coude et pas autre chose : cela permet de bloquer le seul angle d'attaque où \A aurait pu frapper.

% Source : Romain.
\end{technique}


\begin{technique}

\begin{enumerate}
	\item \A donne un coup de couteau au ventre de \D.
	
	\item \D saute sur le côté gauche et vient contrôler l'arme de la main droite.
	
	\item \D donne un coup de pied (droit) au ventre de \A.
\end{enumerate}

Le coup de pied au point (3) peut être exécuté plus rapidement si \D n'a pas posé le pied droit au point (2).

% Source : Romain.
\end{technique}


\begin{technique}

\begin{enumerate}
	\item \A donne un coup de couteau au ventre de \D.
	
	\item \D saute sur le côté gauche et vient contrôler l'arme de la main droite.
	
	\item \D donne un coup de pied/tibia (gauche) derrière le genou droit de \A.
\end{enumerate}

Après le dernier temps \D peut profiter de sa position pour faire une clé avec sa main gauche.
L'intérêt d'avoir frapper dans le creux du genou est de pouvoir appuyer dessus pour emmener facilement \A à terre.

% Source : Romain.
\end{technique}


\begin{technique}

\begin{enumerate}
	\item \A donne un coup de couteau au ventre de \D.
	
	\item \D esquive sur le côté gauche et vient contrôler l'arme de la main droite.
	
	\item \D donne frappe du poing gauche sur le flanc (droit) de \A (en avançant le pied gauche).
	En même temps il ferme sa prise avec la main droite sur le bras droit de \A.
	
	\item \D se retourne et vient percuter \A avec son épaule gauche.
	
	\item \D peut :
	\begin{enumerate}
		\item soit mettre son bras gauche en travers de la poitrine de \A et tirer avec la main droite pour faire basculer \A par-dessus son bras gauche ;
		
		\item soit attraper le bras de \A avec les deux mains et tirer.
	\end{enumerate}
\end{enumerate}

Le coup de poing et la frappe de l'épaule aux temps (3) et (4) servent à déséquilibrer \A afin de le faire tomber.
En particulier le foie se trouve vers le bas du flanc droit et fait une excellente cible.

Au point (4) \D doit veiller à garder l'arme de \A loin de lui, à un endroit où \A ne peut pas se dégager ni la retourner.

Si \A est trop grand pour être mis à terre, \D peut ramener ses mains pour lui planter son couteau dans les jambes.

% Source : Romain.
\end{technique}


