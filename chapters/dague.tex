\chapter{Dague}


La dague que nous considérons ici n'est pas tranchante : il s'agit d'une sorte de long pic qui sert à estoquer afin de pénétrer profondément dans la peau.
Elle faisait partie de l'armement standard du combattant.
Elle fournit un moyen efficace pour continuer le combat dès que la distance est trop courte pour continuer à utiliser l'arme principale.
Elle est aussi utile pour porter un coup létal (par exemple en combat en armure, après projection au sol) car elle est beaucoup plus précise pour atteindre les cibles non protégées.

\index{prise!marteau}
\index{prise!inversée}
Deux prises sont possibles pour la dague : la pointe est dirigée soit vers le haut (prise normale, ou marteau) soit vers le bas (prise inversée, ou "pic à glace").
% https://en.wikipedia.org/wiki/Icepick_grip
Typiquement la première prise correspondra à des attaques montantes tandis et la seconde à des attaques descendantes.
Les cibles respectives sont les suivantes (pour un droitier) :
\begin{itemize}
	\item prise marteau (attaques montantes, figure~\ref{dague:fig:cibles-marteau}) : abdomen gauche, aisselle gauche, sous le plexus, mains~\footnote{Les paumes ne sont pas protégées par les gantelets.} et, dans une moindre mesure, abdomen et aisselles droits ;
	\item prise inversée (attaques descendantes, figure~\ref{dague:fig:cibles-pic}) : gorge, sous le plexus, épaules gauche et droite (entre la clavicule et l'omoplate, pour atteindre le cœur).
\end{itemize}
Il faut appuyer de tout son corps pour bien enfoncer la lame.
Il ne faut pas juste donner de petits coups : la lame n'étant pas tranchante il n'est pas possible de blesser comme avec un couteau, et de plus chacun était formé à la lutte et pouvait ainsi parer et/ou désarmer facilement si les attaques n'étaient pas bien faites.


\begin{figure}[ht]
	\centering
	\subfloat[Prise normale.\label{dague:fig:cibles-marteau}]{\includegraphics[scale=1]{combat_rapproche/cibles_dague_marteau.pdf}}
	\hspace{3cm}
	\subfloat[Prise inversée.\label{dague:fig:cibles-pic}]{\includegraphics[scale=1]{combat_rapproche/cibles_dague_pic.pdf}}
	\caption{Comparaison entre une prise correcte et une prise avec le poignet cassé.}
	\label{dague:fig:cibles}
\end{figure}


\begin{exercice}[Assassins dans une rue]

Délimiter un espace sur le sol (une "ruelle") où se trouve un grand nombre de personnes, toutes armées de dague.
Certaines sont désignée comme étant des assassins par un "maître du jeu" (elles ne se connaissent pas entre elles) : leur but est d'assassiner les autres personnes, qui doivent esquiver l'attaque en faisant attention à ce qui se passe autour, et sans sortir de l'espace délimité.
\end{exercice}


\begin{technique}

\begin{enumerate}
	\item \A tient la dague en prise inversée et attaque l'épaule.
	
	\item Pied gauche en avant, le défenseur bloque en tenant sa dague horizontalement et des deux mains.
	
	\item \D attrape la main de \A avec sa main gauche, et tire vers lui en reculant le pied gauche, en essayant de le déséquilibrer
	
	\item \D avance en retournant la dague de l'ennemi contre lui-même.
\end{enumerate}
\end{technique}


\begin{technique}

\begin{enumerate}
	\item \A tient la dague en prise inversée et attaque l'épaule.
	
	\item Pied gauche en avant et en passant légèrement sur la gauche, \D bloque en tenant sa dague horizontalement et des deux mains.
	
	\item \D avance en levant les bras afin de ramener les bras de l'adversaire dans son dos et afin de les bloquer.
\end{enumerate}

Si \A résiste à la prise, alors il est avantageux de changer pour la méthode précédente.
\end{technique}

