\chapter{Messer}


Le faux tranchant d'un messer est tranchant sur la toute dernière partie (\SI{10}{cm}). La prise se fait comme le sabre.
Après une attaque on casse le poignet pour étendre l'arme et trancher.
Globalement la main gauche reste en arrière.


%%%%%%%%%%%%%%%%%%%
\section{Leküchner}
%%%%%%%%%%%%%%%%%%%
\label{sec:messer:lekuchner}


Les techniques qui suivent sont similaires aux exercices~\ref{mains-nues:ex:enzi-1}, \ref{mains-nues:ex:enzi-2}, \ref{mains-nues:ex:enzi-3} et \ref{mains-nues:ex:enzi-4}.


\begin{technique}

\D démarre en garde du fou.

\begin{enumerate}
	\item \A attaque à la tête.
	\item \D recule et lève son messer pour couper sous le bras de \A avec le faux tranchant.
	\item \D revient en garde haute en reculant.
	\item \D recule encore en laissant bien son messer.
\end{enumerate}

Le but final est d'avoir un grand angle avec la direction de l'attaque afin de heurter le bras plus efficacement (sinon risque d'être parallèle à la lame).
En principe le coup est très fort, et on peut tester en visant le messer de l'autre plutôt que le bras. Pour que le pouce n'ait pas mal on abandonne la prise sabre, en le faisant glisser sur le côté.
Si \D ne revient pas en garde et qu'il a raté alors \A a la possibilité de revenir en estoc.
Fonctionne quelque soit le pied de départ et le côté où l'on part.

Source :~\cite{enzi:dijon:messer_inner:2015}.

\end{technique}


\begin{technique}

\D démarre en garde du fou.

\begin{enumerate}
	\item \A attaque à la tête.
	\item \D esquive sur le côté et coupe le poignet de \A avec le vrai tranchant.
	\item \D revient en garde haute et recule.
\end{enumerate}

La main se trouve du côté où l'on sort.

Source :~\cite{enzi:dijon:messer_inner:2015}.

\end{technique}


\begin{technique}

\D démarre en garde du fou.

\begin{enumerate}
	\item \A attaque à la tête.
	\item \D esquive sur le côté et frappe le poignet de \A avec le vrai tranchant.
	\item \D avance et appuie sur le poignet en coupant.
\end{enumerate}

Source :~\cite{enzi:dijon:messer_inner:2015}.

\end{technique}


\begin{technique}

\D démarre en garde du fou.

\begin{enumerate}
	\item \A attaque à la tête.
	\item \D pare en bœuf à gauche, en se déplaçant légèrement à gauche.
	\item De la main gauche \D pousse la main de \A un peu vers la droite.
	\item \D défend des doigts et envoie son pommeau dans la tête de \A.
	\item \D se sert du choc pour faire tourner son messer en reculant et frappe \A au visage du tranchant.
\end{enumerate}

En fait au temps 2) ce n'est pas un vrai bœuf : il s'agit juste de lever l'arme droit pour accrocher l'arme de \A dans le troisième clou.
Avant 3) \D regardait \A à droite de son bras, après 3) il le regarde à gauche.

Source :~\cite{enzi:dijon:messer_inner:2015}.

\end{technique}


\begin{technique}

\D démarre en garde du fou.

\begin{enumerate}
	\item \A attaque à la tête.
	\item \D se protège en quinte en avançant, pied gauche devant.
	\item \D heurte la main droite de \A avec sa main gauche pour écarter sa lame, tandis que de la main droite il tranche au visage.
\end{enumerate}

Le mouvement en 2) est l'éternel écartement des épaules cher à Romain.
En 3) Enzi conseille de reculer.

Le mouvement du corps est le même que l'exercice~\ref{mains-nues:ex:enzi-2}.

Source :~\cite{enzi:dijon:messer_inner:2015}.

\end{technique}


\begin{technique}

\D démarre en garde du fou.

\begin{enumerate}
	\item \A attaque à la tête.
	\item \D se protège en quinte en avançant, pied gauche devant.
	\item Du tranchant de la main \D percute l'avant bras de \A.
	\item \D ramène son bras gauche, en accrochant éventuellement le pommeau de \A, et pousse avec le bras droit pour écraser le messer de \A et trancher au visage de \A.
\end{enumerate}

Il est important de ne pas chercher à saisir le pommeau : le placement est tel que le messer adversaire sera soit hors portée, soit retenu par la main. Le coup se fait entre le poignet et le coude.
Au dernier temps \A va en général se faire trancher aussi par le faux tranchant de sa propre arme.
% TODO: Le mouvement d'écartement est toujours celui de Romain.

Source :~\cite{enzi:dijon:messer_inner:2015}.

\end{technique}


\begin{technique}

\D n'a pas d'arme.

\begin{enumerate}
	\item \A attaque à la tête.
	\item \D part sur la gauche et couvre avec son bras droit, main vers le haut.
	\item \D éjecte le bras de \A avec un mouvement circulaire dans le sens horaire.
	\item \D frappe \A.
\end{enumerate}

L'objectif au temps 2) est de parvenir à arriver avec l'avant-bras contre le plat de la lame.
Voir l'exercice~\ref{mains-nues:ex:enzi-4}.

Source :~\cite{enzi:dijon:messer_inner:2015}.

\end{technique}


