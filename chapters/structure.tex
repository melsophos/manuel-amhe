\chapter{Structure}


Dans ce chapitre nous présentons des explications générales sur le positionnement du corps.
Nous donnons aussi des exercices pour aider à prendre conscience des positions et des sensations ainsi qu'à construire certains automatismes.
La position de garde est aussi discutée.


%%%%%%%%%%%%%%%%%%%%%%%%%%%%%%%%%%%%%%
\section{L'importance de la structure}
%%%%%%%%%%%%%%%%%%%%%%%%%%%%%%%%%%%%%%


La structure du corps est un élément clé de l'escrime : bien plus que la technique la structure détermine comment on va réagir à une attaque adverse.
Avec une structure faible on aura du mal à résister, mais contre une attaque mal exécutée.
De plus il est aussi plus de réaliser correctement les techniques.
Enfin une mauvaise position du corps entraîne des tensions inutiles qui peuvent accroître le risque de blessures et contribuer à l'apparition de douleurs sur le long terme (tendinite, douleur au genou…).
Pour ces différentes raisons il est important de prêter dès que possible attention à la position du corps et de garder cette question toujours en tête, même après plusieurs années de pratique.

Il peut être frustrant de chercher à améliorer la position du corps car c'est quelque chose qui prend énormément de temps : nous avons été habitués pendant des années à avoir une certaine position (souvent mauvaise), et là il faut en apprendre une autre, qui paraît souvent moins naturelle.
Une première étape est d'apprendre à écouter les sensations du corps : l'escrime est logique et efficace, et de ce fait l'on n'est pas censé être mal à l'aise dans une position.
Ainsi si l'on ressent une gêne quelconque en effectuant une technique ou en prenant une certaine position alors cela signifie que l'on a un problème de structure (ou bien que l'on se met dans une position incorrecte).
Il peut être utile, régulièrement, de s'arrêter -- par exemple à la fin d'une technique ou même d'un coup (en ayant prévenu son partenaire) -- afin de tourner son attention vers son corps.

Les notions qui sont développées dans ce chapitre (et dans les autres chapitres de cette partie, dans une moindre mesure) forment le cœur de l'escrime : une fois que ces principes généraux ont été intégrés -- au niveau corporel et non intellectuel -- il devient possible de s'adapter rapidement à n'importe quelle arme.
En effet l'utilisation d'une arme est globalement déterminée par l'interaction entre ses caractéristiques et le corps humain selon ces principes généraux : il faut apprendre à sentir l'arme et comment elle et le corps s'influencent mutuellement.
Les techniques sont un raffinement : elles aident tout d'abord à construire cette compréhension avant d'ajouter une finesse supplémentaire à l'exécution.
Les notions décrites dans cette partie peuvent se retrouver modifiées en fonction de l'arme étudiée (par exemple la même garde ne sera pas utilisée avec une rapière ou une lance) -- et cela sera expliqué en temps voulu --, mais le principe reste le même.


%%%%%%%%%%%%%%%
\section{Corps}
%%%%%%%%%%%%%%%


\subsection{Position}


% valable dans la plupart des cas
Les genoux sont déverrouillés, les jambes fléchies.
Le pied arrière est tourné entre 45° et 90° sur le côté, tandis que le pied avant est droit.
Les coudes ne sont jamais tendus, le buste est droit.

Expliquons maintenant certains des points précédents :
\begin{itemize}
	\item jambes fléchies : cela permet d'abaisser le centre de gravité et d'être plus stable, et ainsi de mieux résister en cas de chocs ou de poussée (le pire étant d'avoir les jambes totalement tendues) ;
	\item coudes fléchis : des coudes tendus offrent une bonne cible pour placer une clé au corps-à-corps, augmentent le risque de se faire mal aux articulations (à force de buter) et enfin diminuent la rapidité de mouvement ;
	\item pied avant vers l'adversaire : l'alignement du pied définit la direction d'attaque et d'avancée : ainsi si le pied pointe sur le côté, une marche entraînera du mauvais côté.
\end{itemize}

% Thomas
Un bon repère est d'avoir le genou au-dessus du pied, afin que le mollet soit bien vertical, tandis que la cuisse est dans l'axe du pied.
Cela permet de préserver les genoux et d'être plus stable.

Il n'est pas toujours facile au début de trouver la bonne position.
Le fait de se sentir à l'aise est un critère important.
Nous proposons deux exercices qui permettent d'adopter facilement une garde correcte.

\begin{exercice}[Trouver une bonne garde]

Marcher naturellement et s'arrêter.

Source :~\cite{guidoux:dijon:thibault:2015}.

\end{exercice}


\begin{exercice}[Trouver une bonne garde]

Se tenir pied joint, sauter en l'air et retomber jambes écartées (droite devant, gauche en arrière).

Source :~\cite{enzi:dijon:messer_inner:2015}.

\end{exercice}


\subsection{Respiration}

La respiration est un élément important.
Une bonne pratique est d'expirer au moment de porter le coup.
Par exemple en escrime japonaise les attaques sont accompagnées de kiais.


\subsection{Hanches}

Un défaut important du débutant (et qui peut perdurer longtemps) est de dissocier les différentes parties du corps.
On a souvent tendance à utiliser uniquement les bras et les épaules pour les mouvements, alors qu'un mouvement efficace mobilise le corps entier.
En particulier les hanches sont très importantes pour unifier tout le corps, et les mouvements devraient partir de celles-ci en priorité.


%%%%%%%%%%%%%%%%%
\section{Général}
%%%%%%%%%%%%%%%%%
\label{sec:structure:général}


À terme l'intérêt est de développer une compréhension des principes généraux qui vont fonctionner avec n'importe quelle arme, juste en adaptant la distance.
Parmi ces principes nous pouvons indiquer~\cite{enzi:dijon:messer_inner:2015} :
\begin{enumerate}
	\item tension/détente : la réaction doit être rapide et explosive, puis on est relâché jusqu'à la prochaine action ;
	\item mouvements circulaires : la direction est donnée en partant d'un angle de 90° par rapport à la trajectoire de l'autre ;
	\item équilibre/déséquilibre : après un mouvement le corps peut être déséquilibré et il veut alors retourner dans une meilleur position ; en profiter pour le laisser faire efficacement (voir aussi~\cite{guidoux:dijon:thibault:2015}).
\end{enumerate}
L'objectif n'est pas d'avoir une technique parfaite, car en combat réel on aura rarement l'angle idéal, etc.

Il est intéressant de travailler sans gants pour vraiment sentir la poignée et le sentiment de la lame~\cite{enzi:dijon:messer_inner:2015}.
Idem travailler sans chaussures permet de mieux sentir le sol (historiquement ils se battaient sans chaussures ou avec des semelles de \SI{1}{mm}).

