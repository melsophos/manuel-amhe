\chapter{Structure}


Dans ce chapitre nous présentons des explications générales sur le positionnement du corps.
Nous donnons aussi des exercices pour aider à prendre conscience des positions et des sensations ainsi qu'à construire certains automatismes.
La position de garde est aussi discutée.


%%%%%%%%%%%%%%%%%%%%%%%%%%%%%%%%%%%%%%
\section{L'importance de la structure}
%%%%%%%%%%%%%%%%%%%%%%%%%%%%%%%%%%%%%%
\label{sec:structure:général}


La structure du corps est un élément clé de l'escrime : bien plus que la technique la structure détermine comment on va réagir à une attaque adverse.
Il sera plus compliqué de résister avec une structure faible, même contre une attaque mal exécutée.
De plus il est aussi plus difficile de réaliser correctement les techniques.
Enfin une mauvaise position du corps entraîne des tensions inutiles qui peuvent accroître le risque de blessures et contribuer à l'apparition de douleurs sur le long terme (tendinite, douleur au genou…).
Pour ces différentes raisons il est important de prêter dès que possible attention à la position du corps et de garder cette question toujours en tête, même après plusieurs années de pratique.

Il peut être frustrant de chercher à améliorer la position du corps car c'est quelque chose qui prend énormément de temps : nous avons été habitués pendant des années à avoir une certaine position (souvent mauvaise), et là il faut en apprendre une autre, qui paraît souvent moins naturelle.
Une première étape est d'apprendre à écouter les sensations du corps : l'escrime est logique et efficace, et de ce fait l'on n'est pas censé être mal à l'aise dans une position.
Ainsi si l'on ressent une gêne quelconque en effectuant une technique ou en prenant une certaine position alors cela signifie que l'on a un problème de structure (ou bien que l'on a mal compris la position demandée).
Il peut être utile, régulièrement, de s'arrêter -- par exemple à la fin d'une technique ou même d'un coup (en ayant prévenu son partenaire) -- afin de tourner son attention vers son corps.

Les notions qui sont développées dans ce chapitre (et dans les autres chapitres de cette partie, dans une moindre mesure) forment le cœur de l'escrime : une fois que ces principes généraux ont été intégrés -- au niveau corporel et non intellectuel -- il devient possible de s'adapter rapidement à n'importe quelle arme.
En effet l'utilisation d'une arme est globalement déterminée par l'interaction entre ses caractéristiques et le corps humain selon ces principes généraux : il faut apprendre à sentir l'arme et comment elle et le corps s'influencent mutuellement.
Les techniques sont un raffinement : elles aident tout d'abord à construire cette compréhension avant d'ajouter une finesse supplémentaire à l'exécution.
Les notions décrites dans cette partie peuvent se retrouver modifiées en fonction de l'arme étudiée (par exemple la même garde ne sera pas utilisée avec une rapière ou une lance) -- et cela sera expliqué en temps voulu --, mais le principe reste le même.
% ne pas se concentrer sur la forme

À terme l'intérêt est de développer une compréhension des principes généraux qui vont fonctionner avec n'importe quelle arme, juste en adaptant la distance.
Parmi ces principes nous pouvons indiquer~\cite{enzi:dijon:messer_inner:2015} :
% cf Katori
\begin{enumerate}
	\item tension/détente : la réaction doit être rapide et explosive, puis on est relâché jusqu'à la prochaine action ;
	\item mouvements circulaires : la direction est donnée en partant d'un angle de 90° par rapport à la trajectoire de l'autre ;
	\item équilibre/déséquilibre : après un mouvement le corps peut être déséquilibré et il veut alors retourner dans une meilleur position ; en profiter pour le laisser faire efficacement (voir aussi~\cite{guidoux:dijon:thibault:2015}).
\end{enumerate}
L'objectif n'est pas d'avoir une technique parfaite, car en combat réel on aura rarement l'angle idéal, etc.

Il peut être intéressant de travailler sans gants (en prenant les précautions nécessaires) pour vraiment sentir la poignée et le sentiment de la lame~\cite{enzi:dijon:messer_inner:2015}.
Idem travailler sans chaussures permet de mieux sentir le sol (historiquement ils se battaient sans chaussures ou avec des semelles de \SI{1}{mm}).


%%%%%%%%%%%%%%%
\section{Corps}
%%%%%%%%%%%%%%%


Porter une attaque en escrime ne se réduit pas à un simple mouvement du bras : pour que le coup soit efficace il faut que tout le corps contribue à l'action.
Un coup qui est porté seulement avec le bras paraîtra mécanique et peu naturel, tandis qu'une action sollicitant tout le corps sera fluide et harmonieuse.


\subsection{Position de garde}


La position de garde est la position que le corps adopte lors de l'attente et entre chaque attaque.
Il s'agit d'une position robuste, qui permet d'absorber les coups, et dynamique, qui permet de réagir rapidement (aussi bien en attaque qu'en défense) tout en améliorant l'efficacité des mouvements.

% TODO: images
\noindent
La position de garde est la suivante :
\begin{itemize}
	\item genoux déverrouillés et jambes fléchies ;
	\item pied arrière tourné de 45° sur le côté (parfois jusqu'à 90°), pied avant est pointé droit devant ;
	\item les coudes ne sont jamais tendus ;
	\item le dos est droit, le buste est tourné à 45°.
\end{itemize}
Expliquons maintenant certains des points précédents.
\begin{itemize}
	\item jambes fléchies : cela permet d'abaisser le centre de gravité et d'être plus stable, et ainsi de mieux résister en cas de chocs ou de poussée (le pire étant d'avoir les jambes totalement tendues) ;
	
	\item pied avant vers l'adversaire : l'alignement du pied définit la direction d'attaque et d'avancée : ainsi si le pied pointe sur le côté, une marche entraînera du mauvais côté ;
	
	\item coudes fléchis : des coudes tendus offrent une bonne cible pour placer une clé au corps-à-corps, augmentent le risque de se faire mal aux articulations (à force de buter) et enfin diminuent la rapidité de mouvement ;
	
	\item buste de profil : cela permet d'offrir une cible plus réduite à l'opposant tout en augmentant la portée de l'arme tenue dans la main avant.
\end{itemize}
Il faut noter que la position décrite dans cette section correspond à celle utilisée avec de nombreuses armes, mais il ne s'agit pas de la seule.
Par exemple le buste est penché vers l'avant dans la garde à l'épée-bocle (style I.33), tandis que le buste est beaucoup plus de profil à la rapière ou encore le second pied peut se trouver presque vers l'avant en combat rapproché.
Nous introduirons les gardes adaptées aux différentes armes dans les chapitres traitant de celles-ci.

% Thomas
Un bon repère est d'avoir le genou au-dessus du pied, afin que le mollet soit bien vertical, tandis que la cuisse est dans l'axe du pied.
Cela permet de préserver les genoux et d'être plus stable.

Il n'est pas toujours facile au début de trouver la bonne position.
Le fait de se sentir à l'aise est un critère important.
Nous proposons deux exercices qui permettent d'adopter facilement une garde correcte.


\begin{exercice}[Trouver la position des pieds]

Marcher naturellement et s'arrêter (pied droit devant).

Au moment de s'arrêter les pieds se trouvent naturellement dans une position de garde.
Il ne reste plus qu'à se baisser sur ses appuis.

\source{\cite{guidoux:dijon:thibault:2015}.}

\end{exercice}


\begin{exercice}[Trouver une bonne garde]

Se tenir pied joint, sauter en l'air et retomber jambes écartées (droite devant, gauche en arrière).

\source{\cite{enzi:dijon:messer_inner:2015}.}

\end{exercice}


\subsection{Hanches}


Ainsi que nous l'avons vu plus haut, un défaut important du débutant (et qui peut perdurer longtemps) est de dissocier les différentes parties du corps.
On a souvent tendance à utiliser uniquement les bras et les épaules pour les mouvements, alors qu'un mouvement efficace mobilise le corps entier.
Les hanches sont très importantes pour unifier tout le corps, et les mouvements devraient partir de celles-ci en priorité.


\begin{exercice}[La roue (en avant)]

Dans la description de cet exercice nous donnons d'abord la position de départ, puis le mouvement qui permet d'effectuer la transition vers la position suivante, elle-même décrite dans un point séparé.
Cette position sert alors de départ pour la transition suivante, et ainsi de suite : cet exercice est cyclique et peut être travaillé en faisant des longueurs dans le gymnase.

\begin{enumerate}
	\item Position verticale : de face, talons joints, léger angle entre les pieds, bras tendus parallèles au corps (le gauche vers le haut, le droit vers le bas).
	
	\item Transition : avancer le pied droit en levant le bras droit vers l'avant et en basculant le bras gauche vers l'arrière.
	
	\item Position horizontale : de profil, position de profil, pieds en position de garde (jambe droite en avant), bras tendus parallèles au sol (le droit vers l'avant, le gauche vers l'arrière).
	
	\item Transition : avancer le pied gauche à côté du pied droit, le bras gauche bascule vers le bas, le bras droit se lève vers le haut.
	
	\item Position verticale (symétrique par rapport à 1.).
\end{enumerate}

Pendant tous l'enchaînement (en particulier dans la position verticale) les genoux sont légèrement pliés.
Chaque partir du corps reste à la même hauteur : en particulier la tête et les épaules tracent une ligne parallèle au sol (i.e.\ le corps ne fait pas une vague en bougeant -- on retrouve cette idée dans la valse).

Après une certaine pratique on pourra chercher à effectuer la transition dynamiquement, avec l'idée que l'on est en train d'attaquer vers l'avant.

\obj{La position du corps fait qu'il est nécessaire de mobiliser les hanches lors du mouvement.}

% Jean-Paul

\end{exercice}


\begin{exercice}[La roue (en arrière)]

\begin{enumerate}
	\item Position verticale : de face, talons joints, léger angle entre les pieds, bras tendus parallèles au corps (le gauche vers le haut, le droit vers le bas).
	
	\item Transition : reculer le pied gauche en levant le bras droit vers l'avant et en basculant le bras gauche vers l'arrière.
	
	\item Position horizontale : de profil, position de profil, pieds en position de garde (jambe droite en avant), bras tendus parallèles au sol (le droit vers l'avant, le gauche vers l'arrière).
	
	\item Transition : avancer le pied gauche à côté du pied droit, le bras gauche bascule vers le bas, le bras droit se lève vers le haut.
	
	\item Position verticale (symétrique par rapport à 1.).
\end{enumerate}

La différence avec le mouvement principale est que l'on recule le pied gauche lors du premier mouvement.
Toutefois la position horizontale est la même que dans l'exercice précédent.

% Jean-Paul

\end{exercice}


% TODO: déplacer ailleurs
\begin{exercice}[La roue (à deux)]

\A et \D se font face.
\A et \D exécutent respectivement une roue vers l'avant et une roue vers l'arrière, en simultané.

Du fait que seul le mouvement des pieds est inversé \A et \D terminent dans une position symétrique, avec leurs bras parallèles.

% on y reviendra
\obj{Cet exercice permet de prendre conscience de l'occupation du centre de la ligne ainsi que de s'entraîner à agir en même temps que son partenaire.}

% Jean-Paul

\end{exercice}


\subsection{Respiration}


Dans un premier lieu il faut savoir gérer sa respiration afin de ne pas se retrouver à court de souffle, surtout lors du combat libre et lorsqu'il fait chaud (et ce d'autant plus lorsque l'on porte un masque ou d'autres protections).
De plus le souffle permet de rythmer efficacement les mouvements, et en particulier les frappes, en expirant au moment de porter le coup.
Cette dernière idée a été particulièrement développée dans l'escrime japonaise où (presque) chaque coup est accompagnée d'un kiai.


\begin{exercice}[Frapper en expirant]

Choisir une frappe et s'entraîner à expirer en portant le coup.

\end{exercice}

