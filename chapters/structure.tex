\chapter{Structure}


\section{Respiration}

La respiration est un élément important. Une bonne pratique est de souffler au moment de porter le coup. Par exemple en escrime japonaise les attaques sont accompagnées de kiais.

\section{Hanches}

Un défaut important du débutant (et qui peut perdurer longtemps) est de dissocier les différentes parties du corps. On a souvent tendance à utiliser uniquement les bras et les épaules pour les mouvements, alors qu'un mouvement efficace mobilise le corps entier. En particulier les hanches sont très importantes pour unifier tout le corps, et les mouvements devraient partir de celles-ci en priorité.

\section{Général}

% TODO: exercice d'Ingulf
\cite{kohlweiss:hemac:ringen:2014}
