\chapter{Structure}


\section{Respiration}

La respiration est un élément important. Une bonne pratique est de souffler au moment de porter le coup. Par exemple en escrime japonaise les attaques sont accompagnées de kiais.


\section{Hanches}

Un défaut important du débutant (et qui peut perdurer longtemps) est de dissocier les différentes parties du corps. On a souvent tendance à utiliser uniquement les bras et les épaules pour les mouvements, alors qu'un mouvement efficace mobilise le corps entier. En particulier les hanches sont très importantes pour unifier tout le corps, et les mouvements devraient partir de celles-ci en priorité.


\section{Général}

% TODO: exercice d'Ingulf
\cite{kohlweiss:hemac:ringen:2014}

\begin{exercice}
\label{ex:general:miroir}

\A et \D se font face et se déplacent en miroir.
\end{exercice}


\section{Frappes et distance}

\begin{exercice}
\label{ex:frappe-dist:approche-frappe}

\begin{enumerate}
	\item \A et \D démarre hors distance.
	\item \D s'approche de \A.
	\item Quand \A pense qu'il est à la bonne distance, il porte une frappe.
\end{enumerate}

Le but de \A est de parvenir exactement à la bonne distance pour que sa frappe soit efficace, donc ni trop près, ni hors distance.
La frappe de \A peut se faire en avançant ou en se décalant sur le côté, selon le type d'arme.
Au début on peut choisir de faire pratiquer toujours la même frappe (e.g. un oberhau), puis ensuite de laisser le choix.
Finalement il est possible d'utiliser la croix formée par les quillons et le pommeau d'une épée en nylon pour donner une cible.

\end{exercice}


\begin{exercice}

Même exercice que \ref{ex:frappe-dist:approche-frappe}, mais \D place son pommeau (plus ou moins tôt) vers l'une des quatre directions nord/sud/est/ouest, et \A doit effectuer deux frappes consécutives dans les deux angles.

Source : Jan.
\end{exercice}


\begin{exercice}

Même exercice que \ref{ex:frappe-dist:approche-frappe}, mais juste après son coup \A doit reculer tout en revenant en garde.

Pour y parvenir \A doit être prêt à se déplacer rapidement, donc il doit être fléchi et souple pour pouvoir enchaîner les deux actions.

Cet exercice aide à préparer le combat libre.

% combat libre
\end{exercice}


\begin{exercice}
\A se met en équilibre sur une jambe et pratique n'importe quel exercice de frappe sur cible fixe.

Dans cet exercice il est très facile de voir si \A utilise uniquement ses bras ou tout son corps pour porter ses frappes.
\end{exercice}
