\chapter{Concepts généraux}


Les décisions prises lors d'un combat dépendent d'un certain nombre de paramètres que nous allons étudier dans ce chapitre.
En effet l'action juste ne sera pas la même selon la distance, les informations que nous obtenons de l'adversaire (via le contact avec son arme par exemple)


\section{Distance}


La distance est un concept important car les techniques utilisées vont varier en fonction de celle-ci.
De plus il faut s'habituer à être capable d'évaluer rapidement les différentes distances en fonction de l'arme utilisée : être hors distance ne signifie pas la même chose quand on se bat à mains nues ou avec une lance.

Dans la définition ci-dessous nous indiquons entre parenthèses le nom allemand correspondant~\cite{kronenburg:dijon:going_distance:2015}.
L'attribution de ces mots est sujette à débat mais nous les donnons car certains les utilisent.
"Ringen" signifie "corps-à-corps", "arm" signifie "bras" et "leib" signifie "corps".
% cf Forgeng Glossary

\begin{definition}[Distances]
\label{conc:def:distances}
\index{distance}

\noindent
La distance entre deux opposants peut être définie selon les actions possibles :
\begin{itemize}
	\item \emph{hors distance} : les deux combattants sont éloignés et ne peuvent mener aucune action offensive directement ;
	
	\item \emph{approche} (zufechten) : les deux combattants sont éloignés et il suffit d'un temps pour arriver au contact des lames ;
	
	\item \emph{engagement} (fechten) : les armes des deux combattants peuvent se toucher, mais ils sont trop éloignés pour pouvoir toucher directement l'autre ;
	
	\item \emph{engagement proche} (kriegen) : les armes des deux combattants peuvent se toucher et l'opposant est à distance de touche ;
	
	\item \emph{combat rapproché} (armringen) : les deux opposants sont suffisamment proches pour pouvoir toucher l'autre en tendant le bras ou la jambe ;
	
	\item \emph{corps-à-corps} (leibringen) : les deux opposants sont très proches et peuvent travailler sur tout le corps de l'adversaire.
\end{itemize}
\end{definition}


\begin{definition}[Jeux long et court]
\index{distance!jeu court}%|textbf
\index{distance!jeu long}
\index{jeu|see{distance}}

\noindent
En escrime italienne les distances sont rassemblées de la manière suivante :
\begin{itemize}
	\item \emph{jeu long} : hors distance, approche, engagement ;
	
	\item \emph{jeu court} : engagement proche, combat rapproché, corps-à-corps.
\end{itemize}
\end{definition}

% À l'épée longue le jeu court est caractérisé par la position de \emph{mezza spada}, où les deux armes sont au contact au milieu de la lame.

La division du combat en six distances est relativement arbitraire et les frontières sont parfois floues, selon les armes et les styles.
Toutefois cela donne une première idée des actions possibles en fonction de la distance et peut servir de base au raisonnement.

Un combat débute typiquement avec les adversaires hors distance : plusieurs temps sont nécessaires avant de pouvoir lancer une attaque.
Il s'agit d'une phase d'observation où chacun essaie de se faire une idée de son opposant.
Pour cette raison les positions de garde sont peu offensives et défensives : on privilégiera des positions reposantes car l'adversaire est trop loin pour surprendre.
Il peut arriver pendant un combat que les adversaires s'éloignent pour se mettre hors distance afin de souffler.

La position devient plus hermétique dans la phase d'approche : l'adversaire est capable de porter rapidement un coup et il faut être prêt à réagir vite.
Dès qu'une attaque est portée on arrive à l'engagement : les armes sont généralement au contact afin de contrôler les mouvements de l'adversaire et de \emph{sentir} l'intention de l'adversaire.

Dès que l'engagement se rapproche il devient possible de blesser directement l'adversaire et il est donc important de ne pas perdre le contrôle et de ne pas laisser d'ouverture.
On commence aussi à disposer d'un manœuvre plus large sur l'arme de l'adversaire.
% escrime allemande
À noter que l'on passe souvent de la phase d'approche à la phase d'engagement rapproché directement : il suffit que la cible lors de l'attaque soit le corps de l'adversaire pour se retrouver très proche à la fin de l'attaque.

Finalement le combat rapproché permet de donner des coups de poing et de pied à l'adversaire.
En outre on se trouve suffisamment proche pour exécuter certaines clés sur les bras, typiquement.
Le corps-à-corps constitue la dernière étape du combat, et peut éventuellement se dérouler au sol : elle contient des saisies et des clés sur tout le corps (étranglements…) et des projections.

% TODO: continuer
Une précise évaluation de la distance est nécessaire pour porter des coups précis.
Si l'adversaire est trop loin alors le coup ne portera pas, et si l'adversaire est trop proche alors nous sommes ouvert pour une contre-attaque : dans les deux cas l'adversaire peut regagner l'initiative.


\section{Temps}


\index{temps}

Une action n'est adaptée que si elle s'inscrit correctement dans le temps : trop tard et l'on s'expose à une riposte, trop tôt et l'adversaire peut se protéger et voire même prendre l'initiative.
On compte trois temps : l'avant, l'après et l'instant (ou le même-temps).
Ces temps ont été particulièrement théorisés dans la tradition de Liechtenauer, et nous indiquons la traduction allemande entre parenthèses.
En particulier il n'existe pas de mot français rendant exactement le sens de indes, donc nous utiliserons parfois les mots en allemand.


\begin{definition}[Avant (Vor)]
\index{temps!Vor}
\index{temps!avant|see{Vor}}

Agir dans l'avant signifie que l'on prend initiative (par exemple en attaquant).
\end{definition}


\begin{definition}[Après (Nach)]
\index{temps!Nachès}
\index{temps!après|see{Nach}}

Agir dans l'après signifie que l'on réagit à une action de l'adversaire.
\end{definition}


Ainsi l'un des deux adversaires va prendre l'initiative (en étant dans l'avant) tandis que l'autre devra se contenter de réagir en fonction (en étant dans l'après).
D'une certaine manière on pourrait dire que le Vor et le Nach correspondent aux actions offensives et défensives, et on pourrait penser qu'agir dans le Nach est inférieur, mais cela n'est pas tout à fait correct.
En effet il est tout à fait possible de réagir en portant soi-même une attaque, et, si celle-ci est adaptée, il n'est pas forcément nécessaire de se protéger.
Prenons un exemple à l'épée-bocle : \A attaque en laissant sa main à découvert (Vor), tandis que \D fait un pas sur le côté et frappe le bras (Nach).

% TODO: interpréter ce passage
% 14v-16v, p 11 : "if you cannot come in the "Before", wait for the "After"."

% ainséité / pleine conscience
Le dernier temps, Indes, joue un rôle particulier : il rejoint la notion de conscience du moment.
Il s'agit de ressentir instantanément ce qui se passe et d'agir/réagir en fonction avec la méthode la plus appropriée ; cela est particulièrement lors du liage (voir la section suivante).
Ainsi à chaque moment se situe l'Indes, et en fonction des paramètres à notre disposition on choisit d'agir dans le Vor ou dans le Nach.

% traduction : à-propos (?)
\begin{definition}[Même-temps (instant, Indes)]
\index{temps!Indes}
\index{temps!instant|see{Indes}}

L'Indes consiste à être conscient des paramètres définissant le combat et à prendre une décision juste en conséquence.
\end{definition}


\index{vitesse}
Être capable d'agir rapidement s'avère crucial lorsque l'on participe à un combat, et il est clair qu'une personne beaucoup plus rapide que son opposant (que ce soit en vitesse de déplacement ou de réaction) aura un avantage certain.
Le corps évolue grâce à l'entraînement et il ne faut donc pas désespérer si les mouvements ou les réflexes apparaissent lents au début.

\index{rythme}
Le rythme est une autre manière d'organiser le le temps du combat : l'idée est de jouer sur les différentes vitesses qui nous sont accessibles afin de surprendre l'adversaire.
Par exemple on peut forcer l'adversaire dans un schéma en attaquant lentement plusieurs fois avant de conclure avec une attaque beaucoup plus rapide.
On peut aussi commencer une attaque lentement et accélérer sur la fin (éventuellement en changeant la trajectoire).


\section{Liage et sentiment du contact}


À partir du moment où l'un des deux adversaires choisit d'attaquer, la réaction de l'opposant a de fortes chances de conduire à un contact entre les deux armes : il s'agit du \emph{liage}.


\begin{definition}[Liage]
\label{conc:def:liage}
\index{liage}

Un liage est établit dès que les armes sont au contact.
\end{definition}


Le liage représente une opportunité autant qu'un danger, et de nombreuses actions décisives sont entreprises à partir du liage.
En effet on peut avoir une idée de l'intention de l'adversaire en fonction de la manière dont il manie son arme grâce au contact entre les deux armes -- le sentiment du fer.
De plus on possède alors une manière directe d'influencer l'arme, par exemple pour l'écarter.
Toutefois cette position est dangereuse et requiert une attention à chaque instant car il peut être très facile de se faire déborder, et du fait de la proximité cela signifie que l'on n'a pas de deuxième chance.
Il faut ainsi être capable de sentir précisément ce que souhaite faire l'adversaire, et de prendre la bonne décision en fonction (Indes).


\begin{definition}[Sentiment du fer]
\label{conc:def:sentiment-fer}
\index{sentiment du fer}

Le sentiment du fer (ou sentiment du contact) consiste à sentir l'intention de son adversaire à travers le contact que l'on a avec lui, que ce soit via les armes (bouclier compris) ou le corps.
\end{definition}


\begin{exercice}[Variante à l'échauffement d'Ingulf]
\label{att:ex:Ingulf-variantes}
\index{echauffement@échauffement!exercice}

\obj{Cet exercice travaille la structure, l'équilibre et le sentiment du contact.}

La disposition initiale est identique à celle de l'exercice~\ref{struc:ex:Ingulf}.
Lors du déplacement des actions supplémentaires peuvent être exécutées par \A et \D afin de travailler le sentiment du contact :
\begin{enumerate}
	\item Changement de garde : \A peut passer le bras à l'extérieur (gauche au début de l'exercice) sous le bras de \D pour venir attraper l'intérieur de son coude droit.
	Cela met \A dans une position avantageuse car il contrôle totalement le centre et oblige \D à se retrouver de face, avec les deux bras à l'extérieur.
	Au combat \D possèderait peu de cibles intéressantes, à l'opposé de \A.
	
	Afin de ne pas se retrouver en position dominée \D a intérêt à changer lui aussi de garde en même temps que \A, afin de rétablir l'équilibre.
	Dans ce cas la position est exactement opposée à celle de départ.
	
	\item \A essaie d'attraper le bras de \D.
	Pour ce faire il doit lâcher sa prise et passer ses deux bras sous le bras de \D afin de le plaquer contre sa poitrine avec ses avant-bras.
	Attention aux pouces !
	L'objectif de \D est de sentir le moment où \A lâche son autre bras afin de retirer le bras ciblé.
	
	En explorant on peut remarquer que deux moments sont particulièrement adaptés : quand \D change de garde et quand \D pousse vers l'avant.
	
	\item Si \A essaie d'attraper le bras et que \D parvient à se retirer alors \D essaie de se placer dans le dos de \A et de l'enserrer avec ses bras.
	
	\item \A peut lâcher une de ses mains afin de venir toucher la tête de \D.
	\D doit sentir à quel moment \A lâche sa main afin d'esquiver en se baissant.
	
	La difficulté se trouve dans le fait que \A, à ce point de l'exercice, \A peut lâcher sa main pour exécuter différentes actions et \D doit sentir son intention.
	
	\item Fermer les yeux.
\end{enumerate}
Ces divers éléments peuvent être ajoutés un par un (en particulier la première variante est simple et devrait être ajoutée rapidement).

\source{\cite{Kohlweiss:2014:Dijon:RingenSchwert}}
\end{exercice}


Il existe différentes manières de réagir lors du liage : soit en s'y opposant fortement et en poussant (liage fort), soit en se contentant de toucher la lame (liage faible), soit finalement en se contentant d'être présent et de sentir mais sans tomber dans les extrêmes.
La dernière approche est la meilleure car elle offre moins d'emprise à l'adversaire.
Par exemple si l'on est fort au liage l'adversaire peut facilement dérober son arme et attaquer pendant que notre arme est emportée par l'élan.
Au contraire il sera facile pour lui d'écarter notre lame (avec une percussion) si l'on ne met aucune force.
Il s'agit donc de trouver le dosage correct entre les deux.


\begin{definition}[Liage fort]
\index{liage!fort}

Un liage est dit fort si la personne pousse fermement avec son arme.
\end{definition}


\begin{definition}[Liage faible]
\index{liage!faible}

Un liage est dit faible si la personne est simplement au contact mais ne résiste pas.
\end{definition}


\section{Prendre le centre}


Une manière de demeurer protégé est de ne pas laisser d'ouvertures à l'adversaire, que l'on soit au liage ou non : il faut que chaque action qu'il entreprend ait un coût (en temps ou en sécurité).
Cela s'obtient en prenant le centre.


\begin{definition}[Prendre le centre]
\index{centre}

On parle de prendre le centre lorsque l'on occupe (avec l'arme et le corps) l'espace principal entre l'adversaire et soi-même.
\end{definition}


L'idée, en prenant le centre, est de bloquer la ligne centrale : ainsi l'adversaire ne peut attaquer directement aucune partie car il est gêné par l'arme et il s'expose à une riposte immédiate.
De plus le fait d'occuper le centre permet de menacer l'adversaire car l'on se trouve alors dans une position supérieure, avec un accès plus direct à ses ouvertures.
% important : I.33, Destreza (?), Katori
Arrêter la frappe là où se trouvait la cible (ainsi que nous l'avons discuté plus haut) permet de prendre le centre naturellement.
% (cf Liechtenauer)


\section{Résumé}


\noindent
Les concepts importants qui ont été abordés dans ce chapitre sont :
\begin{itemize}
	\item la distance, avec la distinction entre jeu long et jeu court (propice au corps-à-corps) ;
	
	\item les temps Vor (initiative), Nach (réaction) et Indes (prise de décision en fonction des paramètres) ;
	
	\item le liage et le sentiment du fer : lorsque les armes sont au contact on peut deviner l'intention de l'adversaire et agir en fonction ;
	
	\item le centre : prendre le centre revient à établir une position forte qui bloque nos ouvertures et permet d'attaquer plus directement.
\end{itemize}

