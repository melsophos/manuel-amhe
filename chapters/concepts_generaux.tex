\chapter{Concepts généraux}


Les décisions prises lors d'un combat dépendent d'un certain nombre de paramètres que nous allons étudier dans ce chapitre.



% TODO: prendre le centre



\section{Distance}


La distance est un concept important car les techniques utilisées vont varier en fonction de celle-ci.
De plus il faut s'habituer à être capable d'évaluer rapidement les différentes distances en fonction de l'arme utilisée : être hors distance ne signifie pas la même chose quand on se bat à mains nues ou avec une lance.

Dans la définition ci-dessous nous indiquons entre parenthèses le nom allemand correspondant~\cite{kronenburg:dijon:going_distance:2015}.
L'attribution de ces mots est sujette à débat mais nous les donnons car certains les utilisent.
"Ringen" signifie "corps-à-corps", "arm" signifie "bras" et "leib" signifie "corps".
% cf Forgeng Glossary

\begin{definition}[Distances]
\index{distance}

\noindent
La distance entre deux opposants peut être définie selon les actions possibles :
\begin{itemize}
	\item \emph{hors distance} : les deux combattants sont éloignés et ne peuvent mener aucune action offensive directement ;
	
	\item \emph{approche} (zufechten) : les deux combattants sont éloignés et il suffit d'un temps pour arriver au contact des lames ;
	
	\item \emph{engagement} (fechten) : les armes des deux combattants peuvent se toucher, mais ils sont trop éloignés pour pouvoir toucher directement l'autre ;
	
	\item \emph{engagement proche} (kriegen) : les armes des deux combattants peuvent se toucher et l'opposant est à distance de touche ;
	
	\item \emph{combat rapproché} (armringen) : les deux opposants sont suffisamment proches pour pouvoir toucher l'autre en tendant le bras ou la jambe ;
	
	\item \emph{corps-à-corps} (leibringen) : les deux opposants sont très proches et peuvent travailler sur tout le corps de l'adversaire.
\end{itemize}
\end{definition}


\begin{definition}[Jeux long et court]
\index{distance!jeu court}%|textbf
\index{distance!jeu long}
\index{jeu|see{distance}}

\noindent
En escrime italienne les distances sont rassemblées de la manière suivante :
\begin{itemize}
	\item \emph{jeu long} : hors distance, approche, engagement ;
	
	\item \emph{jeu court} : engagement proche, combat rapproché, corps-à-corps.
\end{itemize}
\end{definition}

% À l'épée longue le jeu court est caractérisé par la position de \emph{mezza spada}, où les deux armes sont au contact au milieu de la lame.

La division du combat en six distances est relativement arbitraire et les frontières sont parfois floues, selon les armes et les styles.
Toutefois cela donne une première idée des actions possibles en fonction de la distance et peut servir de base au raisonnement.

Un combat débute typiquement avec les adversaires hors distance : plusieurs temps sont nécessaires avant de pouvoir lancer une attaque.
Il s'agit d'une phase d'observation où chacun essaie de se faire une idée de son opposant.
Pour cette raison les positions de garde sont peu offensives et défensives : on privilégiera des positions reposantes car l'adversaire est trop loin pour surprendre.
Il peut arriver pendant un combat que les adversaires s'éloignent pour se mettre hors distance afin de souffler.

La position devient plus hermétique dans la phase d'approche : l'adversaire est capable de porter rapidement un coup et il faut être prêt à réagir vite.
Dès qu'une attaque est portée on arrive à l'engagement : les armes sont généralement au contact afin de contrôler les mouvements de l'adversaire et de \emph{sentir} l'intention de l'adversaire.

Dès que l'engagement se rapproche il devient possible de blesser directement l'adversaire et il est donc important de ne pas perdre le contrôle et de ne pas laisser d'ouverture.
On commence aussi à disposer d'un manœuvre plus large sur l'arme de l'adversaire.
% escrime allemande
À noter que l'on passe souvent de la phase d'approche à la phase d'engagement rapproché directement : il suffit que la cible lors de l'attaque soit le corps de l'adversaire pour se retrouver très proche à la fin de l'attaque.

Finalement le combat rapproché permet de donner des coups de poing et de pied à l'adversaire.
En outre on se trouve suffisamment proche pour exécuter certaines clés sur les bras, typiquement.
Le corps-à-corps constitue la dernière étape du combat, et peut éventuellement se dérouler au sol : elle contient des saisies et des clés sur tout le corps (étranglements…) et des projections.

% TODO: continuer
Une précise évaluation de la distance est nécessaire pour porter des coups précis.
Si l'adversaire est trop loin alors le coup ne portera pas, et si l'adversaire est trop proche alors nous sommes ouvert pour une contre-attaque : dans les deux cas l'adversaire peut regagner l'initiative.


\section{Temps}


\section{Sentiment du contact}


\begin{definition}[Sentiment du contact]
\index{sentiment du fer}

Le sentiment du contact, aussi appelé plus spécifiquement le sentiment du fer, consiste à sentir l'intention de son adversaire à travers le contact que l'on a avec lui, que ce soit via les armes (bouclier compris) ou le corps.
\end{definition}


\begin{exercice}[Variante à l'échauffement d'Ingulf]
\label{att:ex:Ingulf-variantes}
\index{echauffement@échauffement!exercice}

\obj{Cet exercice travaille la structure, l'équilibre et le sentiment du contact.}

La disposition initiale est identique à celle de l'exercice~\ref{struc:ex:Ingulf}.
Lors du déplacement des actions supplémentaires peuvent être exécutées par \A et \D afin de travailler le sentiment du contact :
\begin{enumerate}
	\item Changement de garde : \A peut passer le bras à l'extérieur (gauche au début de l'exercice) sous le bras de \D pour venir attraper l'intérieur de son coude droit.
	Cela met \A dans une position avantageuse car il contrôle totalement le centre et oblige \D à se retrouver de face, avec les deux bras à l'extérieur.
	Au combat \D possèderait peu de cibles intéressantes, à l'opposé de \A.
	
	Afin de ne pas se retrouver en position dominée \D a intérêt à changer lui aussi de garde en même temps que \A, afin de rétablir l'équilibre.
	Dans ce cas la position est exactement opposée à celle de départ.
	
	\item \A essaie d'attraper le bras de \D.
	Pour ce faire il doit lâcher sa prise et passer ses deux bras sous le bras de \D afin de le plaquer contre sa poitrine avec ses avant-bras.
	Attention aux pouces !
	L'objectif de \D est de sentir le moment où \A lâche son autre bras afin de retirer le bras ciblé.
	
	En explorant on peut remarquer que deux moments sont particulièrement adaptés : quand \D change de garde et quand \D pousse vers l'avant.
	
	\item Si \A essaie d'attraper le bras et que \D parvient à se retirer alors \D essaie de se placer dans le dos de \A et de l'enserrer avec ses bras.
	
	\item \A peut lâcher une de ses mains afin de venir toucher la tête de \D.
	\D doit sentir à quel moment \A lâche sa main afin d'esquiver en se baissant.
	
	La difficulté se trouve dans le fait que \A, à ce point de l'exercice, \A peut lâcher sa main pour exécuter différentes actions et \D doit sentir son intention.
	
	\item Fermer les yeux.
\end{enumerate}
Ces divers éléments peuvent être ajoutés un par un (en particulier la première variante est simple et devrait être ajoutée rapidement).

\source{\cite{Kohlweiss:2014:Dijon:RingenSchwert}}
\end{exercice}
