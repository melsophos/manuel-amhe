\chapter{Attaques et défenses}


Certains des exercices peuvent se faire en plaçant un banc entre les deux partenaires.

\begin{exercice}[Mouvement dans le vide]
Effectuer des attaques simples dans le vide, puis des enchaînements.

Cet exercice devrait être un des premiers à effectuer avec toute nouvelle arme afin de s'habituer à son poids, à sa taille, aux mouvements possibles avec le corps, etc.
\end{exercice}


\subsection{Frappes et distance}


\subsubsection{Mains nues}


Les exercices suivants peuvent se faire en statique ou en se déplaçant.


\begin{exercice}
\label{struct:ex:contact:frappe-signal}

\A et \D sont en garde, avant-bras en contact.
Quand \A baisse sa main, \D vient frapper à l'épaule.

Source : Romain.

\end{exercice}


\begin{exercice}

\A et \D sont en garde, avant-bras en contact.
\A donne un signal, \D vient frapper à l'épaule en enroulant un peu le bras et en gardant le contact.

Source : Romain.

\end{exercice}


\begin{exercice}
\label{struct:ex:contact:frappe-épaules}

\A et \D sont en garde, avant-bras en contact.
Quand \A le sent, il tend le bras droit pour toucher \D sur une épaule (\D ne cherche pas à se protéger).

Naturellement \A peut toucher \D à l'épaule droite juste en tendant le bras.
Mais si \A est parvenu plus dans l'intérieur de \D (par exemple en se déplaçant sur le côté plus vite, ou après un quarté du pied), alors il peut venir le toucher à l'autre épaule.

\end{exercice}


\begin{exercice}
\label{struct:ex:frappe-gauche-droite}

\A et \D se font face.

\begin{enumerate}
	\item \A frappe l'épaule gauche de \D de la main droite, en avançant la jambe droite.
	\item \D vient couvrir avec son bras droit en tournant les hanches.
	\item \A avance la jambe gauche et pose sa main gauche sur l'épaule droite de \D, en maintenant le contact avec la main droite.
\end{enumerate}

Dans l'intérêt de l'exercice, \D ne doit pas avancer sur \A.
Cet exercice est utile pour préparer certains enchaînements d'attaque droite puis gauche (e.g.
en épée longue allemande).

\end{exercice}


\subsubsection{Armes}


Il est intéressant de varier les armes avec lesquelles sont faits les trois exercices suivants afin de s'entraîner à estimer les distances dans divers cas.
À chaque fois l'attaque doit être normale par rapport à l'arme utilisée (ni forcée, ni exagérée, etc.).
La croix de la garde d'une épée longue (en nylon) renversée peut servir de cible.


\begin{exercice}
\A et \D choisissent une arme quelconque.

\begin{enumerate}
	\item \A avance et s'arrête quand il se juge juste hors distance d'une attaque de \D.
	
	\item \D porte l'attaque pour vérifier.
\end{enumerate}

Source : Romain.

\end{exercice}


\begin{exercice}
\A et \D choisissent une arme quelconque.

\begin{enumerate}
	\item \A avance et \D lui dit de s'arrêter quand il le juge juste hors distance.
	
	\item \A porte l'attaque pour vérifier.
\end{enumerate}

Source : Romain.

\end{exercice}


\begin{exercice}
\A choisit une arme quelconque.

\begin{enumerate}
	\item \D avance et \A lui dit de s'arrêter quand il le juge juste à portée d'attaque.
	
	\item \A porte l'attaque pour vérifier.
\end{enumerate}

Source : Romain.

\end{exercice}


\begin{exercice}
\label{ex:frappe-dist:approche-frappe}

\begin{enumerate}
	\item \A et \D démarrent hors distance.
	\item \D s'approche de \A.
	\item Quand \A pense qu'il est à la bonne distance, il porte une frappe.
\end{enumerate}

Le but de \A est de parvenir exactement à la bonne distance pour que sa frappe soit efficace, donc ni trop près, ni hors distance.
La frappe de \A peut se faire en avançant ou en se décalant sur le côté, selon le type d'arme.
Au début on peut choisir de faire pratiquer toujours la même frappe (e.g.
un oberhau), puis ensuite de laisser le choix.
Finalement il est possible d'utiliser la croix d'une épée en nylon pour donner une cible.

\end{exercice}


\begin{exercice}
\label{ex:frappe-dist:approche-double-frappe}

Suite de l'exercice~\ref{ex:frappe-dist:approche-frappe}.

\begin{enumerate}
	\item \A et \D démarrent hors distance.
	\item \D s'approche de \A.
	\item Quand \A pense qu'il est à la bonne distance, il porte une frappe.
	\item \D recule un peu et \A porte une nouvelle attaque.
\end{enumerate}

La seconde attaque peut être soit à gauche normalement (en changeant de pied), soit en contre-pied à droite.
\end{exercice}


\begin{exercice}
\label{ex:frappe-dist:approche-croix-aleat}

La croix d'une épée nylon sert de cible.

\begin{enumerate}
	\item \A et \D démarrent hors distance.
	\item \D s'approche de \A, et à un moment décide de placer son pommeau (plus ou moins tôt) vers l'une des quatre directions nord/sud/est/ouest.
	\item Quand \A pense qu'il est à la bonne distance, il porte deux frappes consécutives dans les deux angles.
\end{enumerate}

Source : Jan.

\end{exercice}


\begin{exercice}
\label{ex:frappe-dist:approche-croix-aleat-garde}

Même exercice que \ref{ex:frappe-dist:approche-frappe}, mais juste après son coup \A doit reculer tout en revenant en garde.

Pour y parvenir \A doit être prêt à se déplacer rapidement, donc il doit être fléchi et souple pour pouvoir enchaîner les deux actions.

Cet exercice aide à préparer le combat libre.

% combat libre
\end{exercice}


\begin{exercice}
\A se met en équilibre sur une jambe et pratique n'importe quel exercice de frappe sur cible fixe.

Dans cet exercice il est très facile de voir si \A utilise uniquement ses bras ou tout son corps pour porter ses frappes.
\end{exercice}


\begin{coup}[Attaque à contre-pied]
\label{struct:coup:contre-pied}
\index{coup!en contre-pied}
\index{contre-pied}

Une attaque à contre-pied consiste à avancer le pied opposé du côté où l'on attaque.
\end{coup}

Le contre-pied permet de gagner en distance et de surprendre l'adversaire.
Le gain de distance par rapport à la position normale est due à la position des épaules.


\begin{exercice}

\begin{enumerate}
	\item \D tient l'épée horizontalement et se déplace.
	
	\item \A vient couper dessus, en essayant d'être juste limite au niveau de sa pointe.
\end{enumerate}

Si \A réussit à toucher à chaque fois alors c'est trop facile.

Source : Romain.

\end{exercice}


% TODO: mieux expliquer
\begin{technique}[Changement de ligne]
\label{struct:tech:changement-ligne}

\A et \D ont les lames en contact.
En tournant les mains de 180° \A peut passer derrière la lame de \D et le toucher.

Cela implique de passer de supination à pronation, ou inversement (selon le côté).

Source : Romain.
\end{technique}


\begin{exercice}
\A et \D n'ont aucune protection.

\A porte une attaque et touche \D (juste un contact léger), à ce moment \D attaque \A (sans s'être protégé), et ainsi de suite.

L'idée de l'exercice est de perdre la peur d'être touché par l'arme de son équipier tout en apprenant à gérer sa force et la distance.
\end{exercice}




