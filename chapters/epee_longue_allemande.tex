\chapter{Épée longue allemande}


\section{Liechtenauer}
\label{sec:épée-longue:liechtenauer}


La référence pour l'escrime de Liechtenauer~\cite{wiktenauer:liechtenauer} (voir l'annexe~\ref{app:maitres:liechtenauer}) est le tétraptyque des quatre glossateurs, traduit par l'\textsc{Ardamhe}~\cite{ardamhe:tetraptyque}.


\subsection{Concepts généraux}


En épée longue allemande, le corps est divisé en quatre quadrants, et à chacun correspond une cible qui peut être attaquée avec un coup ascendant ou descendant.

Après chaque attaque il est important de finir dans une garde :
\begin{itemize}
	\item une attaque descendante termine dans la garde de la charrue ;
	\item une attaque ascendante termine dans la garde du bœuf.
\end{itemize}
Les attaques de base se font sur les diagonales, donc si l'attaque démarre à gauche, la garde sera du côté droit, et inversement.

% TODO: winden, schnappen, duplieren, zucken
% coup de maîtres
% http://www.amheonweb.net/forum/viewtopic.php?f=7&t=535


\begin{coup}[Oberhau (coup supérieur)]
\label{épée-longue:coup:oberhau}
\index{coup!allemand!oberhau}
\index{oberhau}

Un oberhau est un coup descendant à partir d'une garde haute.

\end{coup}

\begin{coup}[Unterhau (coup inférieur)]
\label{épée-longue:coup:unterhau}
\index{coup!allemand!unterhau}
\index{unterhau}

Un unterhau est un coup ascendant à partir d'une garde basse.

\end{coup}

\begin{coup}[Mittelhau (coup intermédiaire)]
\label{épée-longue:coup:mittelhau}
\index{coup!allemand!mittelhau}
\index{mittelhau}

Un mittelhau est un coup horizontal.

\end{coup}

L'oberhau et l'unterhau sont des termes génériques.
En général si aucune indication plus précise n'est donnée il s'agit d'une frappe diagonale : par exemple si l'on est en garde haute avec l'épée au niveau de l'épaule droite, alors un oberhau consistera en une frappe sur la diagonale droite.

\begin{garde}[Versetzen]
\index{versetzen}

Le versetzen est un simple mouvement de défense en opposant sa lame au coup.

\end{garde}


\subsection{Exercices généraux}


Les deux techniques suivants sont utiles pour gagner en fluidité au niveau des poignets, sur les attaques descendantes et ascendantes. Ils peuvent se faire dans le vide, ou avec un partenaire légèrement hors distance.

\begin{exercice}

\begin{enumerate}
	\item \A commence pied gauche en avant.
	\item \A frappe en diagonale droite.
	\item Quand son arme a dépassé le flanc de \D, \A ramène sa main droite en arrière et tourne le poignet gauche, de manière à croiser les poignets avec la lame vers l'arrière.
	\item \A lève les bras et attaque sur la diagonale inverse.
	\item Quand l'arme a dépassé le flanc de \D, \A tourne les deux poignets pour amener l'épée sur le côté.
\end{enumerate}

Au dernier point \A n'a plus qu'à lever les mains pour se retrouver dans la position de départ. L'intérêt de la technique est d'enchaîner rapidement plusieurs séries.

Source : CdA.
\end{exercice}


\begin{exercice}
Cet exercice est exactement comme le précédent mais avec des attaques ascendantes sur les diagonales. Cette fois-ci les poignets ne sont pas croisés à gauche, mais croisés à droite.

Source : CdA.
\end{exercice}


% TODO: comprendre
% \begin{exercice}
% Placer les pieds sur une ligne horizontale, écartés confortablement.
% En partant d'une garde, porter un coup et finir dans la garde opposée.
% 
% Les combinaisons sont les suivantes :
% \begin{itemize}
% 	\item garde du toit et oberhau ;
% 	\item garde du toit et unterhau ;
% 	\item garde du bœuf et estoc ;
% 	\item garde de la charrue et estoc.
% \end{itemize}
% 
% L'intérêt de l'exercice est de forcer à bouger les hanches.
% 
% Source : \cite{kronenburg:dijon:going_distance:2015}.
% \end{exercice}


\begin{technique}

\begin{enumerate}
	\item Départ pieds gauches en avant, \A lance un coup furieux qui est tout juste à la bonne distance.
	\item \D porte son poids sur la jambe arrière pour laisser passer court.
	\item \D passe le pied droit devant et attaque droit devant.
\end{enumerate}

Si le furieux était fait plus proche, alors il faudrait parer avec un furieux en allant sur le côté, mais ici comme le coup peut passer court il est plus économique de procéder ainsi.

Pour que le coup passe il faut que le coup de \D soit franc et termine la pointe loin sur le côté (pas menaçant d'estoc).

Finalement une variante est la suivante : \A avance le pied droit mais sans attaquer. Dans ce cas \D doit attaquer directement. Une manière de savoir si \A attaque ou non est de surveiller les hanches (plutôt que de regarder à la fois les jambes et les bras) : si \A attaque elles seront bien positionnées, sinon elles seront vrillées.

Cette technique peut se faire sur celui du miroir (ex.~\ref{ex:general:miroir}).
\end{technique}


% TODO: techniques séparés
\begin{technique}
Dans cette technique nous allons étudier les coups de maître comme des brisures de garde :
\begin{itemize}
	\item \D en garde du fou : coup crânien~\footnote{Sans masque le faire légèrement hors distance et menacer la poitrine.} ;
	\item \D en garde de la charrue ou en longue pointe : coup bigle ;
	\item \D en garde du bœuf : coup tordu~\footnote{Sans gants le faire sur le fort de la lame, en principe on vise les doigts.} ;
	\item \D en garde du jour : coup de travers.
\end{itemize}
La longue pointe peut aussi briser toute les gardes.

Note sur le coup tordu : il est important de le faire en croisant les poignets (pour \D en bœuf du côté gauche), et non pas en faisant un coup furieux, car dans ce dernier cas nous n'occupons plus le centre et l'autre peut facilement changer sa lame de côté. Avec les poignets croisés, on est vraiment face à l'adversaire, et on est plus offensif même si l'épée est sur le côté (il est possible de la ramener rapidement vers le centre).

De même on doit faire le coup tordu du même côté que la garde de \D (donc sortir à droite si \D est en garde à sa gauche), car sinon on n'occupe pas le centre et \D peut estoquer à la cuisse.

% TODO: principe valable plus généralement
\begin{figure}[ht]
	\centering
	\includegraphics[scale=1]{epee_longue/coup_cranien}
	\caption{Schéma de déplacement pour le coup crânien. Quand \A est sur le cercle, il se trouve hors distance par rapport à \D situé au centre. En sautant sur une ligne droite entre deux points du cercle, \A se retrouve, au milieu de son segment, à un endroit où il peut atteindre \D, et c'est à cet endroit qu'il porte le coup. Diagramme dû à Thomas.}
\end{figure}

Source : Raphaël (CdA).
\end{technique}



\subsection{Le coup furieux (Zornhau) et ses pièces}


% TODO: différence zornhau oberhau

\begin{coup}[Coup furieux – \emph{Zornhau}]
\index{coup!de maître (allemand)!zornhau}
\index{zornhau}

Le coup furieux (all. \emph{Zornhau}, ang. \emph{wrath strike}) est le coup de maître le plus simple.
Il consiste à avancer le pied arrière en portant un coup diagonal au niveau de l'épaule~\cite[fol.~19r-20v, p.~16]{farrell:ringeck}.
Si le coup n'a pas touché la cible, la pointe est menaçante, à hauteur de la poitrine ou du visage.

\end{coup}


Noter qu'à la fin l'attaquant ne se trouve pas en fente.
De plus les mains ne doivent pas être trop hautes (à peu près à la hauteur du nombril – pour le coup "normal").


\begin{technique}

\begin{enumerate}
	\item \A porte un coup furieux en restant sur la même ligne.
	
	\item En réaction \D porte aussi un coup furieux mais en se décalant sur le côté.
	La pointe de l'épée est dirigée vers le visage de \A.
\end{enumerate}

À la fin du temps (2), l'épée de \D se trouve alignée avec la ligne qui joignait originellement \A et \D.
Pour cette raison \D se trouve dans une position bien plus forte.
Cela montre l'intérêt de se décaler lors de l'attaque, et l'technique suppose que \A est naïf.

Source : Raphaël (CdA), d'après~\cite[fol.~19r-20v, §1, p.~16]{farrell:ringeck}.
\end{technique}


\begin{technique}[Abnemmen]
\label{épée-longue:tech:abnemmen}

\begin{enumerate}
	\item \A porte un coup furieux en se décalant sur le côté.
	
	\item En réaction \D porte un coup furieux en se décalant sur le côté.
	
	\item Si \A résiste, \D se baisse pour passer sous l'épée de \A et fait une fente sur la gauche.
	En même temps il fait glisser son épée le long de l'épée de \A (vers soi), la fait passer de l'autre côté et porte une attaque en finissant 
\end{enumerate}

L'attaque au point (3) peut se faire selon la même diagonale que la première attaque (temps (2)), ou bien sur la diagonale opposée.
Il est important de toujours garder le contact avec l'épée de \A, et de ne jamais mettre son épée en arrière.
En principe \D n'a pas le temps de décaler la jambe droite pour revenir d'une vraie garde, mais il doit le faire dès que possible pour redevenir stable.

Contrairement à la technique précédente, la position au temps (2) est symétrique car \A et \D se sont tous les deux décalés.
Pour cette raison les rôles peuvent être inversés au point (3) (\A peut attaquer s'il sent une résistance de la part de \D).

Cette attaque est à utiliser dès que l'on sent une forte résistance de la part de l'opposant.
Celle-ci donne alors le point de pivot nécessaire.

Source : Raphaël (CdA), d'après~\cite[fol.~19r-20v, §2, p.~16]{farrell:ringeck}.
\end{technique}


\begin{technique}

\begin{enumerate}
	\item \A porte un coup furieux en se décalant sur le côté.
	
	\item En réaction \D porte un coup furieux en se décalant sur le côté.
	
	\item \D lève les mains (soit dos de la main droite vers le haut – en bœuf – soit paume en haut) pour menacer la poitrine de \A.
	
	\item \A lève son épée verticalement dans un mouvement réflexe pour se protéger.
	
	\item \D contourne la garde et les bras de \A avec la pointe de son épée pour venir estoquer la poitrine, entre les bras.
\end{enumerate}

Cette technique est à utiliser quand aucun des deux opposants n'exerce de pression.
La réaction de \A au point (4) est mauvaise, la technique suivante montre la réponse correcte.

L'avantage de monter en bœuf au point (3) est d'offrir plus de possibilités si \A réagit (par exemple en enchaînant avec un coup de travers).
L'autre position est plus rapide à exécuter, mais il vaut mieux l'exécuter si \A ne pourra pas réagir.
Une fois les mains levées il s'agit d'une vraie garde (quillons environ horizontaux, etc.).
A priori le faible de \A se trouve accroché dans la garde.

Encore une fois cette technique est symétrique à partir du point (2).

Source : Raphaël (CdA), d'après~\cite[fol.~19r-20v, §3–4, pp.~16–17]{farrell:ringeck}.
\end{technique}


\begin{technique}[Mutation]

\begin{enumerate}
	\item \A porte un coup furieux en se décalant sur le côté.
	
	\item En réaction \D porte un coup furieux en se décalant sur le côté.
	
	\item \D lève les mains (soit dos de la main droite vers le haut – en bœuf – soit paume en haut) pour menacer la poitrine de \A.
	
	\item \A lève son épée verticalement pour placer son fort contre le faible de \D puis il ramène son épée vers le sol en gardant la pointe de \D dans sa garde. \A change de garde car \D pourrait estoquer dans la jambe à l'avant.
	
	\item \A peut estoquer la jambe de \D.
\end{enumerate}

Une mutation consiste à prendre le faible adverse dans son propre fort, et à amener sa pointe vers une cible (haute si on était bas avant, et inversement).

Source : Raphaël (CdA), d'après~\cite[fol.~23v-24v, §3, p.~23]{farrell:ringeck}.
\end{technique}


Les quatre techniques qui suivent s'enchaînent naturellement.
% (voir l'atelier~\ref{app:ateliers:épée-longue-variations-distance}).

\begin{technique}[Zufechten et abnemmen]
\label{épée-longue:tech:dg-zufechten-abnemmen}

\A commence pied gauche en avant.

\begin{enumerate}
	\item \A fait un oberhau à droite en avançant.
	\item \D recule d'un pas et se protège (versetzen).
	\item \A fait un pas sur la gauche en ramenant son épée en arrière, juste assez pour passer l'épée de l'autre côté, et frappe sur l'anti-diagonale (abnehmen).
\end{enumerate}

Il s'agit d'une autre manière de voir l'abnemmen décrit dans la technique~\ref{épée-longue:tech:abnemmen}.

Source : d'après~\cite{kronenburg:dijon:going_distance:2015}.

\end{technique}


\begin{technique}[Fechten et duplieren]
\label{épée-longue:tech:dg-fechten-duplieren}

\A commence pied gauche en avant.

\begin{enumerate}
	\item \A fait un oberhau à droite en avançant.
	\item \D ne bouge pas et se protège (versetzen).
	\item \A tourne sa lame et frappe à la tête sur l'anti-diagonale (duplieren).
\end{enumerate}

Plusieurs interprétations sont possibles.
Soit on peut rester sur place (ou avancer un peu en ligne droite), puis muter pour reprendre le liage.
Sinon on peut partir sur la droite avec le pied droit.
Enfin on peut partir sur la gauche avec le pied gauche à condition de se baisser et de finir en bœuf à droite, afin d'être couvert (ou bien si \D est faible on peut partir à gauche en appuyant sur sa lame).

Source : d'après~\cite{kronenburg:dijon:going_distance:2015}.

\end{technique}


\begin{technique}[Kriegen et absetzen]
\label{épée-longue:tech:dg-kriegen-absetzen}

\A commence pied gauche en avant.

\begin{enumerate}
	\item \A fait un oberhau à droite en avançant.
	\item \D fait un pas en avant et se protège (versetzen).
	\item \A tourne ses mains derrière l'épée de \D, accroche et écarte l'épée de \D puis frappe à la tête (anti-diagonale).
\end{enumerate}

L'écartement de l'épée peut se faire de plusieurs manières, selon la distance : en lâchant la main gauche et en donnant un coup de pommeau dans la lame (le dos de la main droite est en contact avec le plat de l'épée), en écartant les doigts de la main gauche pour pousser sur les quillons, en lâchant la main gauche et en crochetant le poignet de \D, ou bien encore en poussant les quillons avec le poignet gauche.

C'est en pivotant les hanches que l'on va à la fois bien placer les mains pour écarter l'épée et amener la lame au bon endroit.

Source : d'après~\cite{kronenburg:dijon:going_distance:2015}.

\end{technique}


\begin{technique}[Armringen et einlauffen]
\label{épée-longue:tech:dg-armringen-einlauffen}

\A commence pied gauche en avant.

\begin{enumerate}
	\item \A fait un oberhau à droite en avançant.
	\item \D fait un pas en avant et se protège (versetzen).
	\item \D repousse l'épée de A vers le haut en avançant le pied gauche (einlauffen).
	\item \D lâche sa main gauche, passe le bras derrière le bras droit de \A et attrape son propre coude (à la Fiore).
	\item Pour se défendre, \A lâche sa main gauche, crochette le poignet de \D avec le pommeau et du bras gauche pousse le coude de \D pour l'amener à terre.
\end{enumerate}

Au temps 4) il est aussi possible de passer sous le bras de \A : dans ce cas le mouvement se fait plus dans un plan vertical (par exemple en reculant ensuite pour laisser tomber l'épée sur \A), tandis que l'autre se fait sur un plan horizontal (en poussant \A vers la gauche ou vers la droite).

Source : d'après~\cite{kronenburg:dijon:going_distance:2015}.

\end{technique}


\begin{technique}[Leibringen et einlauffen]
\label{épée-longue:tech:dg-leibringen-einlauffen}

Idem que la technique précédente excepté :
\begin{enumerate}
	\item[5.] \A lâche la main gauche, fait un pas pour placer sa jambe devant \D, la hanche contre le centre de \D, et en passant le bras dans le dos de \D, \A le projette.
\end{enumerate}

Source : d'après~\cite{kronenburg:dijon:going_distance:2015}.

\end{technique}


% Krumphau
% deux versions : en rapproché, coup doux pour venir prendre le liage (et si la distance est un peu plus grande on peut venir couper sur la lame) ; de loin (par exemple ouverture d'un combat) : coup puissant (avec un grand saut) car on ne sait pas ce que l'autre va faire et on veut juste virer son épée dans tous les cas

