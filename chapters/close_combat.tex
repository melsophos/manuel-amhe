\chapter{Close combat}


% TODO: défense sur la saisie

\section{Combat à la chaise}

La chaise se tient à deux mains ; une tient le socle et l'autre le dossier.

Pour empêcher l'adversaire d'attraper la chaise il faut la faire tourner régulièrement.

% Source : Romain
Trois attaques sont possibles.

\begin{coup}
\label{coup:close-combat:chaise:frappe-armée}

Lâcher une main et armer un coup loin en arrière.
\end{coup}

\begin{coup}
\label{coup:close-combat:chaise:coincer}

Coincer l'adversaire entre les pieds de la chaise et pousser.

Cette attaque s'utilise si l'autre essaie d'attraper la chaise et s'avance trop.
\end{coup}

\begin{coup}
\label{coup:close-combat:chaise:coup-pied}

Faire semblant de frapper au visage et en tournant la chaise frapper au ventre.
\end{coup}


\begin{technique}

\begin{enumerate}
	\item \A vise les jambes de \D avec un coup de chaise armée.
	
	\item Quand \D recule pour éviter le coup, \A vient le coincer entre les pieds.
	
	\item Si \D commence à se dégager et à frapper haut, frapper les côtes avec les pieds.
\end{enumerate}

Le premier coup vise les jambes car \D peut plus facilement intercepter une attaque haute, quitte à avoir un bleu.
Ne pas laisser \D reculer trop car sinon il aura le temps d'attraper la chaise.

Source : Romain.
\end{technique}


\begin{technique}
Quand \D est coincé par la chaise, s'il cherche à dégager un bras (e.g. pour frapper avec un couteau), effectuer une torsion pour venir bloquer l'épaule avec un des pieds.
Ne pas hésiter à changer la prise, par exemple pour bloquer le bras.

Source : Romain.
\end{technique}


\section{Attaque de sentinelle}

\D est une sentinelle, \A a un couteau et vient attaquer \D par l'arrière.


\begin{technique}

\begin{enumerate}
	\item \A commence par tirer l'épaule gauche de \D pour déséquilibrer, et attirer l'attention de l'autre côté.
	
	\item \A frappe \D à la gorge puis l'étrangle de la main gauche, et avec la jambe droite frappe la jambe droite de \D et appuie dessus.
	
	\item \A poignarde \D à la gorge avec le couteau.
\end{enumerate}

Il y a deux points de pivot pour déséquilibrer \D.
Il est important d'amener \D à terre (moins visible des autres sentinelles).

Source : Romain.
\end{technique}


\begin{technique}

\D est armée d'une arme à feu, posée à terre et qui encombre la vue.

\begin{enumerate}
	\item \A commence par tirer l'épaule gauche de \D pour déséquilibrer, et attirer l'attention de l'autre côté.
	
	\item \A frappe \D à la gorge puis l'étrangle de la main gauche, et avec la jambe droite donne un coup de pied dans l'arme (tout en frappant la jambe de \D avec le genou si possible), avec un mouvement de balayette.
	
	\item \A poignarde \D à la gorge avec le couteau.
\end{enumerate}

Le coup dans l'arme est nécessaire pour empêcher D d'appuyer sur la gâchette ou d'éviter la baïonnette.
De plus si \D s'appuie sur l'arme il va être déséquilibrer.

Source : Romain.
\end{technique}


\begin{technique}

\begin{enumerate}
	\item \A commence par tirer l'épaule gauche de \D pour déséquilibrer, et attirer l'attention de l'autre côté.
	
	\item \A attrape \D par le cou avec sa main gauche et le tire en arrière, et de la main droite il plante dans le creux entre le cou et l'épaule.
\end{enumerate}

Cette technique s'utilise si \D est trop grand.

Source : Romain.
\end{technique}
