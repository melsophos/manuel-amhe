\chapter{Close combat}


L'idée du close combat~\footnotemark{} est de fournir des techniques d'autodéfense simples et efficaces.%
\footnotetext{Mon approche du close combat a été influencée par Samuel Baumard et Romain Wenz.}
Fairbairn constitue une référence sur le close combat~\cite{fairbairn:allin:2008}.

% TODO: défense sur la saisie

% à terre : technique du ciseau


\section{Étranglements}


Pour limiter l'efficacité d'un étranglement, il faut monter les épaules et rentrer le menton.
Ceci est sous-entendu sur toute défense d'étranglement.


\begin{technique}[Étranglement arrière]

\begin{enumerate}
	\item \A arrive derrière \D pour l'étrangler avec le bras droit.
	
	\item Dès qu'il sent le début de l'étranglement, \D vient placer sa main droite dans le creux du coude de \A, à l'intérieur.
	
	\item \D attrape de sa main gauche le poignet de \A, et, en se baissant et en tournant autour de sa jambe droite, se dégage sur le côté droit.
	
	\item \D remonte le bras de \A dans son dos pour passer une clé.
\end{enumerate}

Il faut que \D intervienne vite pour placer sa main dans le creux du coude : il est trop tard dès que le bras de \A est plaqué contre sa gorge.

% Source : Raphaël.
\end{technique}


\begin{technique}[Étranglement arrière]

\begin{enumerate}
	\item \A arrive derrière \D pour l'étrangler avec le bras droit.
	
	\item Dès qu'il sent le début de l'étranglement, \D vient placer sa main droite dans le creux du coude de \A, à l'intérieur.
	
	\item \alt{Si \D ne parvient pas à se dégager.} \D enroule sa jambe gauche autour de la jambe droite de \A, avant de se laisser tomber dessus.
\end{enumerate}

Il faut que \D intervienne vite pour placer sa main dans le creux du coude : il est trop tard dès que le bras de \A est plaqué contre sa gorge.

% Source : Raphaël.
\end{technique}


\begin{technique}[Étranglement latéral]

\begin{enumerate}
	\item \A se trouve à gauche de \D et vient étrangler avec son bras droit (\D a son bras gauche dans le dos de \A).
	
	\item \D percute le flanc de \A avec l'épaule.
	
	\item De la main droite \D attrape la jambe de \A (par en-dessous) tandis que la main gauche fait basculer le haut du corps (e.g. en poussant sur l'épaule gauche, le coup…). 
\end{enumerate}

% Source : Raphaël.
\end{technique}


\section{Mains nues contre couteau}


Dans cette section \A est armé du couteau.

Si les deux opposants sont proches alors \A doit prendre l'initiative car il est dans une position forte.


\begin{exercice}
Acheter une blouse blanche de peinture et des feutres.
Faire un combat libre et voir où sont les coups.

Cela permet de se rendre compte d'une nombre de fois où l'on aurait pris un coup de couteau en réalité.
% Romain
\end{exercice}


\begin{exercice}
\label{cc:ex:blocage-estoc}

\D est jambe gauche devant.

\begin{enumerate}
	\item \A attaque \D avec un estoc.
	
	\item \D vient arrêter le coup avec ses deux avant-bras verticaux, très serrés, jambe droite devant (\D pousse sa jambe gauche pour y parvenir).
	
	\item \D avance et maintient le bras de \A bloqués.
\end{enumerate}

Comme l'idée est d'avancer et de maintenir le couteau loin, \D ne doit pas tirer le bras de \A en arrière.

% Romain
\end{exercice}


\begin{technique}

\begin{enumerate}
	\item \A attaque \D avec un estoc.
	
	\item \D bloque le coup avec ses deux avant-bras verticaux (exercice~\ref{cc:ex:blocage-estoc}).
	
	\item \D enroule son bras droit autour du cou de \A et appuie vers le bas, tandis que son bras gauche tire la main de \A en arrière.
\end{enumerate}

% Romain
\end{technique}


\begin{technique}

\begin{enumerate}
	\item \A attaque \D avec un estoc.
	
	\item \D bloque le coup avec ses deux avant-bras verticaux (exercice~\ref{cc:ex:blocage-estoc}).
	
	\item Grâce à un mouvement de hanche, \D ramène son bras et vient frapper poitrine ou menton.
\end{enumerate}

% Romain
\end{technique}


\begin{exercice}

\D est jambe gauche devant.

\begin{enumerate}
	\item \A attaque.
	
	\item \D se baisse pour éviter le coup et vient toucher épaule.
\end{enumerate}

Application : coup de poing au menton.

% Romain
\end{exercice}


\begin{technique}

\begin{enumerate}
	\item \A attaque.
	
	\item \D se baisse pour éviter le coup.
	
	\item \D vient percuter \A au ventre.
	
	\item \D enroule son bras gauche par derrière l'épaule de \A et se redresse pour faire une clé.
\end{enumerate}

La main se trouve du côté droit de la tête de \A, pour avoir un levier maximal.

Variante : en présence d'un mur, venir plaquer \A contre le mur, contrôler du bras gauche et étrangler (en se décalant sur le côté gauche).
\A ne peut pas riposter (et peut être menotté, etc.).
Cette technique se fait typiquement dans un bar quand quelqu'un veut frapper, on réagit rapidement en se laissant tomber.

% Romain
\end{technique}


\begin{technique}

Un mur se trouve derrière \A.

\begin{enumerate}
	\item \A attaque.
	
	\item \D se baisse pour éviter le coup.
	
	\item \D vient percuter \A au ventre et le plaque contre le mur.
	
	\item \D contrôle du bras gauche et étrangle (en se décalant sur le côté gauche).
\end{enumerate}

\A ne peut pas riposter (et peut être menotté, etc.).
Cette technique se fait typiquement dans un bar quand quelqu'un veut frapper, on réagit rapidement en se laissant tomber.

% Romain
\end{technique}


\begin{exercice}

\begin{enumerate}
	\item \A attaque assez haut.
	
	\item \D pousse sur la jambe droite pour esquiver sur la gauche (la droite reste au même endroit) en venant placer le dos de la main droite sur le coude de \A.
	
	\item \D avance la jambe droite et vient plier le coude de \A contre lui.
\end{enumerate}

La position de la garde ressemble à une sixte.
Il faut garder le coude très proche.
Avec un couteau \D viendrait directement couper au creux du coude.

% Romain
\end{exercice}


\begin{technique}

\begin{enumerate}
	\item \A attaque avec un estoc.
	
	\item \D esquive sur la gauche, en couvrant avec une main ou les deux, puis il vient donner un coup de pied gauche derrière le genou de \A.
	
	\item \D place sa main derrière le manche du couteau pour ensuite tirer vers l'avant de le retourner et de planter \A.
\end{enumerate}

Le coup de pied est là pour déséquilibrer \A et lui donner envie de reculer, et il est alors plus simple de retourner le couteau.

Cette technique est plutôt du close-combat moderne, mais elle peut se faire en médiéval, sans le coup de pied et avec une dague en prise inversée.

% Romain
\end{technique}


\begin{technique}

\begin{enumerate}
	\item \D avance le pied droit juste devant celui de \A (en vrai : l'écrase)
	
	\item \D vient mettre sa jambe gauche entre les deux jambes de \A et tire sur son bras droit avec la main gauche, tandis que le bras droit vient s'appuyer sur la poitrine.
	
	\item \D effectue une projection au sol.
\end{enumerate}

Il est important de bien verrouiller le bras armé de \A.

% Romain
\end{technique}


\section{Mains nues contre une arme}


\begin{technique}

\A est armé d'une matraque.

\begin{enumerate}
	\item \A attaque \D à la tête.
	
	\item \D esquive sur le côté droit et bloque le bras avec sa main gauche, au niveau du poignet.
	
	\item \D attrape le coude de \A avec sa main droite (par en-dessous, à l'extérieur), en se trouvant fléchi.
	
	\item \D se redresse et passe derrière \A en soulevant son coude pour pour le tordre.
	
	\item \D fléchit pour entraîner \A à terre.
	
	\item \D s'assoit par terre et passe ses jambes autour du bras de \A, puis il tire le bras vers lui pour le déboîter~\footnotemark.
	\footnotetext{À l'entrainement il faut être prudent car cette prise est très puissante.}
\end{enumerate}

Au temps (5) \D doit être faire attention à ne pas rester debout au-dessus de \A qui peut le frapper avec ses pieds.

Au dernier temps les jambes ne servent pas à pousser dans la direction opposée, mais à maintenir \A au sol.

% Source : Raphaël.
\end{technique}


\section{Divers}


\begin{technique}[Défense contre le tirage de cheveux]

\begin{enumerate}
	\item \A tire les cheveux de \D à l'arrière.
	
	\item \D attrape la main de \A avec ses deux mains, et tourne autour d'une de ses jambes pour passer sous le bras de \A tout en donnant un coup de pied en arrière de l'autre jambe.
\end{enumerate}

Idéalement \D devrait sortir du côté où se trouve le bras qui le tient, car il se trouve ainsi hors distance (tandis que l'autre côté est très exposé), mais il ne peut pas forcément le savoir.
S'il y parvient (en sentant la main quand il l'attrape) le coup de pied peut venir seulement après.

% Source : Raphaël.
\end{technique}



\section{Combat à la chaise}

La chaise se tient à deux mains ; une tient le socle et l'autre le dossier.

Pour empêcher l'adversaire d'attraper la chaise il faut la faire tourner régulièrement.

% Source : Romain
Trois attaques sont possibles.

\begin{coup}
\label{coup:close-combat:chaise:frappe-armée}

Lâcher une main et armer un coup loin en arrière.
\end{coup}

\begin{coup}
\label{coup:close-combat:chaise:coincer}

Coincer l'adversaire entre les pieds de la chaise et pousser.

Cette attaque s'utilise si l'autre essaie d'attraper la chaise et s'avance trop.
\end{coup}

\begin{coup}
\label{coup:close-combat:chaise:coup-pied}

Faire semblant de frapper au visage et en tournant la chaise frapper au ventre.
\end{coup}


\begin{technique}

\begin{enumerate}
	\item \A vise les jambes de \D avec un coup de chaise armée.
	
	\item Quand \D recule pour éviter le coup, \A vient le coincer entre les pieds.
	
	\item Si \D commence à se dégager et à frapper haut, frapper les côtes avec les pieds.
\end{enumerate}

Le premier coup vise les jambes car \D peut plus facilement intercepter une attaque haute, quitte à avoir un bleu.
Ne pas laisser \D reculer trop car sinon il aura le temps d'attraper la chaise.

% Source : Romain.
\end{technique}


\begin{technique}
Quand \D est coincé par la chaise, s'il cherche à dégager un bras (e.g. pour frapper avec un couteau), effectuer une torsion pour venir bloquer l'épaule avec un des pieds.
Ne pas hésiter à changer la prise, par exemple pour bloquer le bras.

% Source : Romain.
\end{technique}


\section{Attaque de sentinelle}


\D est une sentinelle, \A a un couteau et vient attaquer \D par l'arrière.


\begin{technique}

\begin{enumerate}
	\item \A commence par tirer l'épaule gauche de \D pour déséquilibrer, et attirer l'attention de l'autre côté.
	
	\item \A frappe \D à la gorge puis l'étrangle de la main gauche, et avec la jambe droite frappe la jambe droite de \D et appuie dessus.
	
	\item \A poignarde \D à la gorge avec le couteau.
\end{enumerate}

Il y a deux points de pivot pour déséquilibrer \D.
Il est important d'amener \D à terre (moins visible des autres sentinelles).

% Source : Romain.
\end{technique}


\begin{technique}

\D est armée d'une arme à feu, posée à terre et qui encombre la vue.

\begin{enumerate}
	\item \A commence par tirer l'épaule gauche de \D pour déséquilibrer, et attirer l'attention de l'autre côté.
	
	\item \A frappe \D à la gorge puis l'étrangle de la main gauche, et avec la jambe droite donne un coup de pied dans l'arme (tout en frappant la jambe de \D avec le genou si possible), avec un mouvement de balayette.
	
	\item \A poignarde \D à la gorge avec le couteau.
\end{enumerate}

Le coup dans l'arme est nécessaire pour empêcher D d'appuyer sur la gâchette ou d'éviter la baïonnette.
De plus si \D s'appuie sur l'arme il va être déséquilibrer.

% Source : Romain.
\end{technique}


\begin{technique}

\begin{enumerate}
	\item \A commence par tirer l'épaule gauche de \D pour déséquilibrer, et attirer l'attention de l'autre côté.
	
	\item \A attrape \D par le cou avec sa main gauche et le tire en arrière, et de la main droite il plante dans le creux entre le cou et l'épaule.
\end{enumerate}

Cette technique s'utilise si \D est trop grand.

% Source : Romain.
\end{technique}
