\chapter{Épée longue}

\section{Général}

\begin{technique}

\begin{enumerate}
	\item \A attaque à l'épaule droite.
	
	\item \D se déplace sur le côté gauche et absorbé en tierce.
	
	\item \D passe sa jambe droite vers la gauche et frappe À avec un coup montant.
	
	\item \A monte en bœuf pour se défendre.
\end{enumerate}

Cette technique travaille le contre-pied, et les défenses rapides pour \A.

Source : Romain.

\end{technique}


\begin{technique}
\begin{enumerate}
	\item \A et \D sont en garde, pied gauche en avant.
	
	\item \A attaque l'épaule gauche de \D (sur une diagonale) en avançant le pied droit.
	
	\item \D s'esquive et utilise son épée pour écarter l'épée de \A (dans la direction où elle allait).
	
	\item \A utilise l'élan fournit par le chassé de \D afin de reculer le pied droit et de passer dans la garde du bœuf (à droite).
	
	\item \D essaie de se de déplacer pour estoquer.
	
	\A doit viser au niveau de l'épaule droite de \D, s'il est plus haut, \D peut facilement passer en-dessous. De plus \A doit être vraiment fléchi et ferme, en mettant son épée bien devant lui pour occuper le centre.
	
	\item \A ramène son pied gauche derrière le droit pour amener la pointe de son épée au niveau du ventre de \D.
	
	Ce dernier mouvement ne peut être bien accompli que si \A est bien fléchi, ce qui lui permet de changer de direction facilement et de réagir vite.
\end{enumerate}

Cette technique est symétrique.

Source : Romain (CdA).
\end{technique}


\section{Liechtenauer (allemande)}
\label{sec:épée-longue:liechtenauer}


La référence pour l'escrime de Liechtenauer~\cite{wiktenauer:liechtenauer} (voir l'annexe~\ref{app:maitres:liechtenauer}) est le tétraptyque des quatre glossateurs, traduit par l'\textsc{Ardamhe}~\cite{ardamhe:tetraptyque}.


\subsection{Concepts généraux}


En épée longue allemande, le corps est divisé en quatre quadrants, et à chacun correspond une cible qui peut être attaquée avec un coup ascendant ou descendant.

Après chaque attaque il est important de finir dans une garde :
\begin{itemize}
	\item une attaque descendante termine dans la garde de la charrue ;
	\item une attaque ascendante termine dans la garde du bœuf.
\end{itemize}
Les attaques de base se font sur les diagonales, donc si l'attaque démarre à gauche, la garde sera du côté droit, et inversement.

% TODO: traduction, etc. ; mettre sous forme de définition
En escrime allemande il existe quatre distances différentes~\cite{kronenburg:dijon:going_distance:2015} :
\begin{enumerate}
	\item zufechten : hors distance de frappe, même en faisant un pas (?) ;
	\item fechten : contact des épées, hors distance pour une touche directe ;
	\item kriegen : contact des épées, à distance pour une touche directe ;
	\item ringen : contact des épées, à distance pour toucher l'autre avec la main (soit juste les bras - armringen, soit le corps même - leibringen).
\end{enumerate}

% TODO: winden, schnappen, duplieren, zucken
% coup de maîtres
% http://www.amheonweb.net/forum/viewtopic.php?f=7&t=535


\begin{coup}[Oberhau (coup supérieur)]
\label{épée-longue:coup:oberhau}
\index{coup!allemand!oberhau}
\index{oberhau}

Un oberhau est un coup descendant à partir d'une garde haute.

\end{coup}

\begin{coup}[Unterhau (coup inférieur)]
\label{épée-longue:coup:unterhau}
\index{coup!allemand!unterhau}
\index{unterhau}

Un unterhau est un coup ascendant à partir d'une garde basse.

\end{coup}

L'oberhau et l'unterhau sont des termes génériques.
En général si aucune indication plus précise n'est donnée il s'agit d'une frappe diagonale : par exemple si l'on est en garde haute avec l'épée au niveau de l'épaule droite, alors un oberhau consistera en une frappe sur la diagonale droite.

\begin{garde}[Versetzen]
\index{versetzen}

Le versetzen est un simple mouvement de défense en opposant sa lame au coup.

\end{garde}


\subsection{Exercices généraux}


Les deux techniques suivants sont utiles pour gagner en fluidité au niveau des poignets, sur les attaques descendantes et ascendantes. Ils peuvent se faire dans le vide, ou avec un partenaire légèrement hors distance.

\begin{exercice}

\begin{enumerate}
	\item \A commence pied gauche en avant.
	\item \A frappe en diagonale droite.
	\item Quand son arme a dépassé le flanc de \D, \A ramène sa main droite en arrière et tourne le poignet gauche, de manière à croiser les poignets avec la lame vers l'arrière.
	\item \A lève les bras et attaque sur la diagonale inverse.
	\item Quand l'arme a dépassé le flanc de \D, \A tourne les deux poignets pour amener l'épée sur le côté.
\end{enumerate}

Au dernier point \A n'a plus qu'à lever les mains pour se retrouver dans la position de départ. L'intérêt de l'technique est d'enchaîner rapidement plusieurs séries.

Source : CdA.
\end{exercice}


\begin{exercice}
Cet exercice est exactement comme le précédent mais avec des attaques ascendantes sur les diagonales. Cette fois-ci les poignets ne sont pas croisés à gauche, mais croisés à droite.

Source : CdA.
\end{exercice}


\begin{technique}

\begin{enumerate}
	\item Départ pieds gauches en avant, \A lance un coup furieux qui est tout juste à la bonne distance.
	\item \D porte son poids sur la jambe arrière pour laisser passer court.
	\item \D passe le pied droit devant et attaque droit devant.
\end{enumerate}

Si le furieux était fait plus proche, alors il faudrait parer avec un furieux en allant sur le côté, mais ici comme le coup peut passer court il est plus économique de procéder ainsi.

Pour que le coup passe il faut que le coup de \D soit franc et termine la pointe loin sur le côté (pas menaçant d'estoc).

Finalement une variante est la suivante : \A avance le pied droit mais sans attaquer. Dans ce cas \D doit attaquer directement. Une manière de savoir si \A attaque ou non est de surveiller les hanches (plutôt que de regarder à la fois les jambes et les bras) : si \A attaque elles seront bien positionnées, sinon elles seront vrillées.

Cette technique peut se faire sur celui du miroir (ex.~\ref{ex:general:miroir}).
\end{technique}


% TODO: techniques séparés
\begin{technique}
Dans cette technique nous allons étudier les coups de maître comme des brisures de garde :
\begin{itemize}
	\item \D en garde du fou : coup crânien~\footnote{Sans masque le faire légèrement hors distance et menacer la poitrine.} ;
	\item \D en garde de la charrue ou en longue pointe : coup bigle ;
	\item \D en garde du bœuf : coup tordu~\footnote{Sans gants le faire sur le fort de la lame, en principe on vise les doigts.} ;
	\item \D en garde du jour : coup de travers.
\end{itemize}
La longue pointe peut aussi briser toute les gardes.

Note sur le coup tordu : il est important de le faire en croisant les poignets (pour \D en bœuf du côté gauche), et non pas en faisant un coup furieux, car dans ce dernier cas nous n'occupons plus le centre et l'autre peut facilement changer sa lame de côté. Avec les poignets croisés, on est vraiment face à l'adversaire, et on est plus offensif même si l'épée est sur le côté (il est possible de la ramener rapidement vers le centre).

De même on doit faire le coup tordu du même côté que la garde de \D (donc sortir à droite si \D est en garde à sa gauche), car sinon on n'occupe pas le centre et \D peut estoquer à la cuisse.

% TODO: principe valable plus généralement
\begin{figure}[ht]
	\centering
	\includegraphics[scale=1]{epee_longue/coup_cranien}
	\caption{Schéma de déplacement pour le coup crânien. Quand \A est sur le cercle, il se trouve hors distance par rapport à \D situé au centre. En sautant sur une ligne droite entre deux points du cercle, \A se retrouve, au milieu de son segment, à un endroit où il peut atteindre \D, et c'est à cet endroit qu'il porte le coup. Diagramme dû à Thomas.}
\end{figure}

Source : Raphaël (CdA).
\end{technique}



\subsection{Le coup furieux (Zornhau) et ses pièces}


% TODO: différence zornhau oberhau

\begin{coup}[Coup furieux – \emph{Zornhau}]
\index{coup!de maître (allemand)!zornhau}
\index{zornhau}

Le coup furieux (all. \emph{Zornhau}, ang. \emph{wrath strike}) est le coup de maître le plus simple.
Il consiste à avancer le pied arrière en portant un coup diagonal au niveau de l'épaule~\cite[fol.~19r-20v, p.~16]{farrell:ringeck}.
Si le coup n'a pas touché la cible, la pointe est menaçante, à hauteur de la poitrine ou du visage.

\end{coup}


Noter qu'à la fin l'attaquant ne se trouve pas en fente.
De plus les mains ne doivent pas être trop hautes (à peu près à la hauteur du nombril – pour le coup "normal").


\begin{technique}

\begin{enumerate}
	\item \A porte un coup furieux en restant sur la même ligne.
	
	\item En réaction \D porte aussi un coup furieux mais en se décalant sur le côté.
	La pointe de l'épée est dirigée vers le visage de \A.
\end{enumerate}

À la fin du temps (2), l'épée de \D se trouve alignée avec la ligne qui joignait originellement \A et \D.
Pour cette raison \D se trouve dans une position bien plus forte.
Cela montre l'intérêt de se décaler lors de l'attaque, et l'technique suppose que \A est naïf.

Source : Raphaël (CdA), d'après~\cite[fol.~19r-20v, §1, p.~16]{farrell:ringeck}.
\end{technique}


\begin{technique}[Abnemmen]
\label{épée-longue:tech:abnemmen}

\begin{enumerate}
	\item \A porte un coup furieux en se décalant sur le côté.
	
	\item En réaction \D porte un coup furieux en se décalant sur le côté.
	
	\item Si \A résiste, \D se baisse pour passer sous l'épée de \A et fait une fente sur la gauche.
	En même temps il fait glisser son épée le long de l'épée de \A (vers soi), la fait passer de l'autre côté et porte une attaque en finissant 
\end{enumerate}

L'attaque au point (3) peut se faire selon la même diagonale que la première attaque (temps (2)), ou bien sur la diagonale opposée.
Il est important de toujours garder le contact avec l'épée de \A, et de ne jamais mettre son épée en arrière.
En principe \D n'a pas le temps de décaler la jambe droite pour revenir d'une vraie garde, mais il doit le faire dès que possible pour redevenir stable.

Contrairement à la technique précédente, la position au temps (2) est symétrique car \A et \D se sont tous les deux décalés.
Pour cette raison les rôles peuvent être inversés au point (3) (\A peut attaquer s'il sent une résistance de la part de \D).

Cette attaque est à utiliser dès que l'on sent une forte résistance de la part de l'opposant.
Celle-ci donne alors le point de pivot nécessaire.

Source : Raphaël (CdA), d'après~\cite[fol.~19r-20v, §2, p.~16]{farrell:ringeck}.
\end{technique}


\begin{technique}

\begin{enumerate}
	\item \A porte un coup furieux en se décalant sur le côté.
	
	\item En réaction \D porte un coup furieux en se décalant sur le côté.
	
	\item \D lève les mains (soit dos de la main droite vers le haut – en bœuf – soit paume en haut) pour menacer la poitrine de \A.
	
	\item \A lève son épée verticalement dans un mouvement réflexe pour se protéger.
	
	\item \D contourne la garde et les bras de \A avec la pointe de son épée pour venir estoquer la poitrine, entre les bras.
\end{enumerate}

Cette technique est à utiliser quand aucun des deux opposants n'exerce de pression.
La réaction de \A au point (4) est mauvaise, la technique suivante montre la réponse correcte.

L'avantage de monter en bœuf au point (3) est d'offrir plus de possibilités si \A réagit (par exemple en enchaînant avec un coup de travers).
L'autre position est plus rapide à exécuter, mais il vaut mieux l'exécuter si \A ne pourra pas réagir.
Une fois les mains levées il s'agit d'une vraie garde (quillons environ horizontaux, etc.).
A priori le faible de \A se trouve accroché dans la garde.

Encore une fois cette technique est symétrique à partir du point (2).

Source : Raphaël (CdA), d'après~\cite[fol.~19r-20v, §3–4, pp.~16–17]{farrell:ringeck}.
\end{technique}


\begin{technique}[Mutation]

\begin{enumerate}
	\item \A porte un coup furieux en se décalant sur le côté.
	
	\item En réaction \D porte un coup furieux en se décalant sur le côté.
	
	\item \D lève les mains (soit dos de la main droite vers le haut – en bœuf – soit paume en haut) pour menacer la poitrine de \A.
	
	\item \A lève son épée verticalement pour placer son fort contre le faible de \D puis il ramène son épée vers le sol en gardant la pointe de \D dans sa garde. \A change de garde car \D pourrait estoquer dans la jambe à l'avant.
	
	\item \A peut estoquer la jambe de \D.
\end{enumerate}

Une mutation consiste à prendre le faible adverse dans son propre fort, et à amener sa pointe vers une cible (haute si on était bas avant, et inversement).

Source : Raphaël (CdA), d'après~\cite[fol.~23v-24v, §3, p.~23]{farrell:ringeck}.
\end{technique}


Les quatre techniques qui suivent s'enchaînent naturellement (voir l'atelier~\ref{app:ateliers:épée-longue-variations-distance}).

\begin{technique}[Zufechten et abnemmen]
\label{épée-longue:tech:dg-zufechten-abnemmen}

\A commence pied gauche en avant.

\begin{enumerate}
	\item \A fait un oberhau à droite en avançant.
	\item \D recule d'un pas et se protège (versetzen).
	\item \A fait un pas sur la gauche en ramenant son épée en arrière, juste assez pour passer l'épée de l'autre côté, et frappe sur l'anti-diagonale (abnehmen).
\end{enumerate}

Il s'agit d'une autre manière de voir l'abnemmen décrit dans la technique~\ref{épée-longue:tech:abnemmen}.

Source : d'après~\cite{kronenburg:dijon:going_distance:2015}.

\end{technique}


\begin{technique}[Fechten et duplieren]
\label{épée-longue:tech:dg-fechten-duplieren}

\A commence pied gauche en avant.

\begin{enumerate}
	\item \A fait un oberhau à droite en avançant.
	\item \D ne bouge pas et se protège (versetzen).
	\item \A tourne sa lame et frappe à la tête sur l'anti-diagonale (duplieren).
\end{enumerate}

Plusieurs interprétations sont possibles.
Soit on peut rester sur place (ou avancer un peu en ligne droite), puis muter pour reprendre le liage.
Sinon on peut partir sur la droite avec le pied droit.
Enfin on peut partir sur la gauche avec le pied gauche à condition de se baisser et de finir en bœuf à droite, afin d'être couvert (ou bien si \D est faible on peut partir à gauche en appuyant sur sa lame).

Source : d'après~\cite{kronenburg:dijon:going_distance:2015}.

\end{technique}


\begin{technique}[Kriegen et absetzen]
\label{épée-longue:tech:dg-kriegen-absetzen}

\A commence pied gauche en avant.

\begin{enumerate}
	\item \A fait un oberhau à droite en avançant.
	\item \D fait un pas en avant et se protège (versetzen).
	\item \A tourne ses mains derrière l'épée de \D, accroche et écarte l'épée de \D puis frappe à la tête (anti-diagonale).
\end{enumerate}

L'écartement de l'épée peut se faire de plusieurs manières, selon la distance : en lâchant la main gauche et en donnant un coup de pommeau dans la lame (le dos de la main droite est en contact avec le plat de l'épée), en écartant les doigts de la main gauche pour pousser sur les quillons, en lâchant la main gauche et en crochetant le poignet de \D, ou bien encore en poussant les quillons avec le poignet gauche.

C'est en pivotant les hanches que l'on va à la fois bien placer les mains pour écarter l'épée et amener la lame au bon endroit.

Source : d'après~\cite{kronenburg:dijon:going_distance:2015}.

\end{technique}


\begin{technique}[Armringen et einlauffen]
\label{épée-longue:tech:dg-armringen-einlauffen}

\A commence pied gauche en avant.

\begin{enumerate}
	\item \A fait un oberhau à droite en avançant.
	\item \D fait un pas en avant et se protège (versetzen).
	\item \D repousse l'épée de A vers le haut en avançant le pied gauche (einlauffen).
	\item \D lâche sa main gauche, passe le bras derrière le bras droit de \A et attrape son propre coude (à la Fiore).
	\item Pour se défendre, \A lâche sa main gauche, crochette le poignet de \D avec le pommeau et du bras gauche pousse le coude de \D pour l'amener à terre.
\end{enumerate}

Au temps 4) il est aussi possible de passer sous le bras de \A : dans ce cas le mouvement se fait plus dans un plan vertical (par exemple en reculant ensuite pour laisser tomber l'épée sur \A), tandis que l'autre se fait sur un plan horizontal (en poussant \A vers la gauche ou vers la droite).

Source : d'après~\cite{kronenburg:dijon:going_distance:2015}.

\end{technique}


\begin{technique}[Leibringen et einlauffen]
\label{épée-longue:tech:dg-leibringen-einlauffen}

Idem que la technique précédente excepté :
\begin{enumerate}
	\item[5.] \A lâche la main gauche, fait un pas pour placer sa jambe devant \D, la hanche contre le centre de \D, et en passant le bras dans le dos de \D, \A le projette.
\end{enumerate}

Source : d'après~\cite{kronenburg:dijon:going_distance:2015}.

\end{technique}


% Krumphau
% deux versions : en rapproché, coup doux pour venir prendre le liage (et si la distance est un peu plus grande on peut venir couper sur la lame) ; de loin (par exemple ouverture d'un combat) : coup puissant (avec un grand saut) car on ne sait pas ce que l'autre va faire et on veut juste virer son épée dans tous les cas


\section{Italienne}


% poids typique : 1.7 kg

% Romain
Les parades se font toujours en se déplaçant sur le côté afin d'adoucir le choc : les épées étant lourdes il n'est pas possible de prendre une parade franche sans se déplacer.

% Romain
Une attaque peut être précédée d'un batté.
Ce batté peut envoyer l'épée dans la même direction que celle du déplacement, mais la position de l'attaque est telle que le défenseur ne peut pas revenir.

% Romain
L'escrime italienne à l'épée longue se fait typiquement en armure de plates.
Sur une amure trois zones absorbent très bien les chocs du fait de la largeur de la plaque de métal : les avant-bras et bras, les cuisses et le ventre (côtés compris).
Il est donc possible de rabattre volontairement l'arme adverse sur ces zones, cf les trois premiers techniques.
Cela provoque un effet de surprise, et de plus l'opposant perd l'amplitude qui donne de la force à sa frappe, ce qui augmente l'intérêt d'avancer vraiment.

% Romain
Quand la distance est faible et que l'on tient l'épée à une main, il faut avoir le dos de la main sur le dessus (pronation).
En effet si la lame est chassée dans cette position il sera beaucoup plus facile de revenir que si la main est tournée dans l'autre sens.

% Romain
Si \A et \D ont le fer en contact et que \A ne menace pas \D avec sa pointe en la gardant bien entre les deux, \D peut tourner ses poignets d'un côté en se déplaçant dans la même direction, ce qui permet d'estoquer.
Par exemple si son l'épée est placée pour attaquer l'épaule gauche de \A, \D tourne ses mains dans le sens anti-horaire – avec la main droite passant de supination à pronation – et va vers la droite.
De même si \A essaie de faire une entrée en lutte alors qu'il se trouve trop loin il est possible d'utiliser cette technique.


\begin{coup}[Estramaçon]
\index{coup!estramaçon}
\index{estramaçon}

L'estramaçon est un grand coup donné verticalement avec le tranchant.
\end{coup}


\begin{technique}

\begin{enumerate}
	\item \A attaque depuis sa droite l'épaule gauche de \D.
	\item \D pare en quarte en se déplaçant sur le côté droit.
	\item \D lâche sa main gauche et vient heurter la lame de \A avec son avant-bras (par-dessus) pour la rabattre sur son bras, en changeant de jambe.
	\item \D tourne sa main en pronation et vient placer sa main gauche sous la lame.
\end{enumerate}

Au point (3) il est important que \D s'accroupisse très bas et entre dans la distance.
De plus \D doit prendre garde à toujours garder sa pointe menaçante, bien au centre.

Source : Romain, d'après Fiore.

\end{technique}


\begin{technique}

\begin{enumerate}
	\item \A attaque depuis sa droite l'épaule gauche de \D.
	\item \D pare en quarte en se déplaçant sur le côté droit.
	\item \D lâche sa main gauche et vient passer son bras entre les deux lames, pour ensuite enrouler son bras autour de la lame, en changeant de jambe. \D attrape le quillon (du bas) de \A.
\end{enumerate}

Cette technique est moins efficace que la technique précédente.
Il est important de tenir le quillon car sinon \A peut dégager sa lame en tirant dessus, et en profiter pour couper sous l'aisselle (sauf si \D est suffisamment près et menaçant avec sa pointe).

Source : Romain, d'après Fiore.

\end{technique}


\begin{technique}

\begin{enumerate}
	\item \A attaque depuis sa droite l'épaule gauche de \D.
	\item \D pare en quarte en se déplaçant sur le côté droit.
	\item \D pose sa paume gauche sur le plat de sa lame et s'en sert pour venir abattre la lame de \A contre sa cuisse (à \D).
\end{enumerate}

L'intérêt de poser sa main sur sa propre lame est de ne pas prendre le risque d'attraper le tranchant de \A.

Source : Romain, d'après Fiore.

\end{technique}


\begin{technique}

\begin{enumerate}
	\item \A attaque depuis la droite.
	\item \D pare en quarte puis monte en bœuf (mutation) pour estoquer.
	\item \A lève les mains, fait faire un tour avec son épée derrère lui et vient frapper derrière le genou droit de \D en faisant un grand saut sur le côté.
\end{enumerate}

L'idée pour \A est de finir juste à porter du genou, qui est situé un peu devant l'épaule, et ainsi \D est trop loin pour frapper.
Le coup doit arrivée derrière le genou car le côté est protéger par des disques de métal.

Le coup en (3) est l'équivalent italien d'un Zwerchau, qui est beaucoup plus ample.

Source : Romain, d'après Marozzo.

\end{technique}


\begin{technique}

\begin{enumerate}
	\item \A attaque depuis sa droite.
	\item \D pare en quarte.
	\item \D vient en seconde.
	\item \D peut venir piquer la jambe ou, mieux, venir frapper au visage avec le faux tranchant.
\end{enumerate}

\end{technique}


\begin{technique}

\begin{enumerate}
	\item \A attaque avec un estoc.
	\item \D pare en quarte.
	\item \D chasse l'épée de \A vers la droite avec le faux tranchant et il finit en octave.
	\item \D laisse son épée aller en arrière et profite de l'élan pour venir frapper.
\end{enumerate}

Un bon estoc se fait en laissant d'abord tomber l'épée horizontalement, et ensuite en poussant la pointe.
Le pied se pose juste après l'impact, ce qui permet de pouvoir continuer d'avancer si l'autre recule.
Cette manière de faire est aussi plus précise.
% ça diminue le chance de coup double car l'autre voit que \A a décidé de prendre l'initiative

Au point (3) \D peut attendre que \A décide d'avancer/reculer, ou bien encore qu'il relâche un peu sa menace, pour lancer la suite.

Source : Romain.

\end{technique}


Les trois techniques qui suivent sont efficaces contre un adversaire qui cherche à avoir la tête, et en particulier en combat libre où il y a de nombreux coups doubles.
Au dernier temps la position de la lame empêche \A de pouvoir toucher \D.
Dans tous les cas, au moment de la parade, \D doit maintenir sa pointe en direction du visage de \A pour garder le contrôle du centre.
Ainsi si \A recule \D peut le suivre facilement, et \A ne peut pas non plus attaquer.


\begin{technique}

\begin{enumerate}
	\item \A attaque \D sur son épaule droite.
	
	\item \D pare avec le faux tranchant (en sixte).
	
	\item \D avance en diagonale vers la gauche et vient frapper \A avec le vrai tranchant.
\end{enumerate}

Source : Romain, d'après Marozzo.

\end{technique}


\begin{technique}

\begin{enumerate}
	\item \A attaque \D sur son épaule droite ou sur la tête.
	
	\item \D pare avec le faux tranchant (en quinte, pointe vers la gauche).
	
	\item \D avance en diagonale vers la gauche et vient frapper \A à la tête avec le vrai tranchant, l'épée totalement dans l'axe.
\end{enumerate}

Source : Romain, d'après Marozzo.

\end{technique}


\begin{technique}

\begin{enumerate}
	\item \A attaque \D sur son épaule gauche.
	
	\item \D pare avec le vrai tranchant (en quarte).
	
	\item \D avance en diagonale vers la droite et vient frapper \A avec le faux tranchant.
\end{enumerate}

Source : Romain, d'après Marozzo.

\end{technique}


\begin{technique}[Coup de travers italien]

\begin{enumerate}
	\item \A menace \D dans le cadrant supérieur droit en passant en bœuf.
	
	\item \A déplace son pied gauche loin sur le côté mais en gardant son buste et son épée aux mêmes endroits.
	
	\item \A décroche son épée et exécute un coup de travers en ramenant son pied arrière, en ciblant l'arrière de la jambe de \D.
\end{enumerate}

Le fait de déplacer le pied gauche sans le reste du corps induit une torsion qui va donner la force au coup en rotation.

La première menace permet de fixer l'épée adverse, et il faut donc l'encourager à rester de ce côté en menaçant vraiment, sinon dès qu'il sentira que l'épée décroche il pourra revenir prendre le centre facilement.

Source : Romain, d'après Marozzo.
\end{technique}


\begin{technique}

\begin{enumerate}
	\item \A et \D démarre en longue pointe.
	
	\item \A donne un coup vertical en visant la tête.
	
	\item \D se décale sur le côté et frappe en diagonale pour intercepter la lame de \A tout en le frappant à la tête.
\end{enumerate}

Comme \D se décale en second il a un léger avantage au niveau de l'axe.
La frappe de \D doit être diagonale pour briser la symétrie : si elle était aussi verticale alors il pourrait y avoir un coup double.

Source : Romain.
% travail avec un boken pour mieux sentir
\end{technique}

