\chapter{Épée longue}

\section{Général}

\subsection{Techniques}

\begin{exercice}
\begin{enumerate}
	\item \A et \D sont en garde, pied gauche en avant.
	
	\item \A attaque l'épaule gauche de \D (sur une diagonale) en avançant le pied droit.
	
	\item \D s'esquive et utilise son épée pour écarter l'épée de \A (dans la direction où elle allait).
	
	\item \A utilise l'élan fournit par le chassé de \D afin de reculer le pied droit et de passer dans la garde du bœuf (à droite).
	
	\item \D essaie de se de déplacer pour estoquer.
	
	\A doit viser au niveau de l'épaule droite de \D, s'il est plus haut, \D peut facilement passer en-dessous. De plus \A doit être vraiment fléchi et ferme, en mettant son épée bien devant lui pour occuper le centre.
	
	\item \A ramène son pied gauche derrière le droit pour amener la pointe de son épée au niveau du ventre de \D.
	
	Ce dernier mouvement ne peut être bien accompli que si \A est bien fléchi, ce qui lui permet de changer de direction facilement et de réagir vite.
\end{enumerate}

Cet exercice est symétrique.

Source : Romain (CdA).
\end{exercice}

\section{Liechtenauer (allemande)}

La référence pour l'escrime de Liechtenauer est le tétraptyque des quatre glossateurs, traduit par l'\textsc{Ardamhe}~\cite{ardamhe:tetraptyque}.

En épée longue allemande, le corps est divisé en quatre quadrants, et à chacun correspond une cible qui peut être attaquée avec un coup ascendant ou descendant.

Après chaque attaque il est important de finir dans une garde :
\begin{itemize}
	\item une attaque descendante termine dans la garde de la charrue ;
	\item une attaque ascendante termine dans la garde du bœuf.
\end{itemize}
Les attaques de base se font sur les diagonales, donc si l'attaque démarre à gauche, la garde sera du côté droit, et inversement.

\subsection{Exercices}

Les deux exercices suivants sont utiles pour gagner en fluidité au niveau des poignets, sur les attaques descendantes et ascendantes. Ils peuvent se faire dans le vide, ou avec un partenaire légèrement hors distance.

\begin{exercice}
\begin{enumerate}
	\item \A commence pied gauche en avant.
	\item \A frappe en diagonale droite.
	\item Quand son arme a dépassé le flanc de \D, \A ramène sa main droite en arrière et tourne le poignet gauche, de manière à croiser les poignets avec la lame vers l'arrière.
	\item \A lève les bras et attaque sur la diagonale inverse.
	\item Quand l'arme a dépassé le flanc de \D, \A tourne les deux poignets pour amener l'épée sur le côté.
\end{enumerate}

Au dernier point \A n'a plus qu'à lever les mains pour se retrouver dans la position de départ. L'intérêt de l'exercice est d'enchaîner rapidement plusieurs séries.

Source : CdA.
\end{exercice}

\begin{exercice}
Cet exercice est exactement comme le précédent mais avec des attaques ascendantes sur les diagonales. Cette fois-ci les poignets ne sont pas croisés à gauche, mais croisés à droite.

Source : CdA.
\end{exercice}

\begin{exercice}
\begin{enumerate}
	\item Départ pieds gauches en avant, \A lance un coup furieux qui est tout juste à la bonne distance.
	\item \D porte son poids sur la jambe arrière pour laisser passer court.
	\item \D passe le pied droit devant et attaque droit devant.
\end{enumerate}

Si le furieux était fait plus proche, alors il faudrait parer avec un furieux en allant sur le côté, mais ici comme le coup peut passer court il est plus économique de procéder ainsi.

Pour que le coup passe il faut que le coup de \D soit franc et termine la pointe loin sur le côté (pas menaçant d'estoc).

Finalement une variante est la suivante : \A avance le pied droit mais sans attaquer. Dans ce cas \D doit attaquer directement. Une manière de savoir si \A attaque ou non est de surveiller les hanches (plutôt que de regarder à la fois les jambes et les bras) : si \A attaque elles seront bien positionnées, sinon elles seront vrillées.

Cet exercice peut se faire sur celui du miroir (ex.~\ref{ex:general:miroir}).
\end{exercice}

% TODO: exercices séparés
\begin{exercice}
Dans cet exercice nous allons étudier les coups de maître comme des brisures de garde :
\begin{itemize}
	\item \D en garde du fou : coup crânien~\footnote{Sans masque le faire légèrement hors distance et menacer la poitrine.} ;
	\item \D en garde de la charrue ou en longue pointe : coup bigle ;
	\item \D en garde du bœuf : coup tordu~\footnote{Sans gants le faire sur le fort de la lame, en principe on vise les doigts.} ;
	\item \D en garde du jour : coup de travers.
\end{itemize}
La longue pointe peut aussi briser toute les gardes.

Note sur le coup tordu : il est important de le faire en croisant les poignets (pour \D en bœuf du côté gauche), et non pas en faisant un coup furieux, car dans ce dernier cas nous n'occupons plus le centre et l'autre peut facilement changer sa lame de côté. Avec les poignets croisés, on est vraiment face à l'adversaire, et on est plus offensif même si l'épée est sur le côté (il est possible de la ramener rapidement vers le centre).

De même on doit faire le coup tordu du même côté que la garde de \D (donc sortir à droite si \D est en garde à sa gauche), car sinon on n'occupe pas le centre et \D peut estoquer à la cuisse.

\begin{figure}[ht]
	\centering
	\includegraphics[scale=1]{epee_longue/coup_cranien}
	\caption{Schéma de déplacement pour le coup crânien. Quand \A est sur le cercle, il se trouve hors distance par rapport à \D situé au centre. En sautant sur une ligne droite entre deux points du cercle, \A se retrouve, au milieu de son segment, à un endroit où il peut atteindre \D, et c'est à cet endroit qu'il porte le coup. Diagramme dû à Thomas.}
\end{figure}


Source : Raph (CdA).
\end{exercice}


