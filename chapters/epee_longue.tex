\chapter{Épée longue}

\section{Général}

\subsection{Techniques}

\begin{exercice}
\begin{enumerate}
	\item \A et \D sont en garde, pied gauche en avant.
	
	\item \A attaque l'épaule gauche de \D (sur une diagonale) en avançant le pied droit.
	
	\item \D s'esquive et utilise son épée pour écarter l'épée de \A (dans la direction où elle allait).
	
	\item \A utilise l'élan fournit par le chassé de \D afin de reculer le pied droit et de passer dans la garde du bœuf (à droite).
	
	\item \D essaie de se de déplacer pour estoquer.
	
	\A doit viser au niveau de l'épaule droite de \D, s'il est plus haut, \D peut facilement passer en-dessous. De plus \A doit être vraiment fléchi et ferme, en mettant son épée bien devant lui pour occuper le centre.
	
	\item \A ramène son pied gauche derrière le droit pour amener la pointe de son épée au niveau du ventre de \D.
	
	Ce dernier mouvement ne peut être bien accompli que si \A est bien fléchi, ce qui lui permet de changer de direction facilement et de réagir vite.
\end{enumerate}

Cet exercice est symétrique.

Source : Romain (CdA).
\end{exercice}

\section{Allemande}

En épée longue allemande, le corps est divisé en quatre quadrants, et à chacun correspond une cible qui peut être attaquée avec un coup ascendant ou descendant.

Après chaque attaque il est important de finir dans une garde :
\begin{itemize}
	\item une attaque descendante termine dans la garde de la charrue ;
	\item une attaque ascendante termine dans la garde du bœuf.
\end{itemize}
Les attaques de base se font sur les diagonales, donc si l'attaque démarre à gauche, la garde sera du côté droit, et inversement.


\subsection{Exercices}

Les deux exercices suivants sont utiles pour gagner en fluidité au niveau des poignets, sur les attaques descendantes et ascendantes. Ils peuvent se faire dans le vide, ou avec un partenaire légèrement hors distance.

\begin{exercice}
\begin{enumerate}
	\item \A commence pied gauche en avant.
	\item \A frappe en diagonale droite.
	\item Quand son arme a dépassé le flanc de \D, \A ramène sa main droite en arrière et tourne le poignet gauche, de manière à croiser les poignets avec la lame vers l'arrière.
	\item \A lève les bras et attaque sur la diagonale inverse.
	\item Quand l'arme a dépassé le flanc de \D, \A tourne les deux poignets pour amener l'épée sur le côté.
\end{enumerate}

Au dernier point \A n'a plus qu'à lever les mains pour se retrouver dans la position de départ. L'intérêt de l'exercice est d'enchaîner rapidement plusieurs séries.

Source : CdA.
\end{exercice}

\begin{exercice}
Cet exercice est exactement comme le précédent mais avec des attaques ascendantes sur les diagonales. Cette fois-ci les poignets ne sont pas croisés à gauche, mais croisés à droite.

Source : CdA.
\end{exercice}

