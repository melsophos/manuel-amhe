\chapter{Épée longue : généralités}


\begin{technique}

\begin{enumerate}
	\item \A attaque à l'épaule droite.
	
	\item \D se déplace sur le côté gauche et absorbé en tierce.
	
	\item \D passe sa jambe droite vers la gauche et frappe À avec un coup montant.
	
	\item \A monte en bœuf pour se défendre.
\end{enumerate}

Cette technique travaille le contre-pied, et les défenses rapides pour \A.

Source : Romain.

\end{technique}


\begin{technique}
\begin{enumerate}
	\item \A et \D sont en garde, pied gauche en avant.
	
	\item \A attaque l'épaule gauche de \D (sur une diagonale) en avançant le pied droit.
	
	\item \D s'esquive et utilise son épée pour écarter l'épée de \A (dans la direction où elle allait).
	
	\item \A utilise l'élan fournit par le chassé de \D afin de reculer le pied droit et de passer dans la garde du bœuf (à droite).
	
	\item \D essaie de se de déplacer pour estoquer.
	
	\A doit viser au niveau de l'épaule droite de \D, s'il est plus haut, \D peut facilement passer en-dessous. De plus \A doit être vraiment fléchi et ferme, en mettant son épée bien devant lui pour occuper le centre.
	
	\item \A ramène son pied gauche derrière le droit pour amener la pointe de son épée au niveau du ventre de \D.
	
	Ce dernier mouvement ne peut être bien accompli que si \A est bien fléchi, ce qui lui permet de changer de direction facilement et de réagir vite.
\end{enumerate}

Cette technique est symétrique.

Source : Romain (CdA).
\end{technique}

