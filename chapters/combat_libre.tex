\chapter{Combat libre}

% TODO: parler de Boorman
% \cite{boorman:dijon:applied_combatives:2014}

\section{Préparation}


\begin{exercice}

On forme une ligne d'attaquants face à un défenseur \D (plus ou moins) protégé.
Au signal, le premier de la ligne \A s'avance et dispose d'un temps limité pour porter une frappe sur \D (uniquement sur une zone protégée).
Ce dernier n'a que le droit de se défendre.
Si \A parvient à toucher \D, il s'arrête et se remet dans la file.

Le temps peut être plus ou moins court, une dizaine de seconde est correct, et il est possible de faire plusieurs files pour faire tourner plus vite.
Cet exercice prépare à la fois les attaquants à être rapide dans leur décision et dans leur attaque, mais aussi elle renforce les défenseurs en les forçant à être toujours sur la défensive et à être confronté à de nombreux styles.
Les attaquants peuvent être protégés ou non, s'il n'y a pas assez de masques, mais cela reste préférable.

\end{exercice}



\begin{exercice}
Même exercice que le précédent, mais cette fois-ci \D a le droit de riposter (là où est protégé \D).
Si \A est touché, il arrête le combat et se remet dans la file.

Pour augmenter la difficulté les zones autorisées peuvent être différentes (en particulier si les attaquants sont moins protégés).

Variante : au moment où le signal est donné, \D annonce la seule cible que \A a le droit de toucher.

\end{exercice}


\section{Entrée en lutte}

L'idée de vouloir entrer en lutte peut venir du fait que l'adversaire a tendance à avoir les bras très tendus, ou encore à se pencher en avant.
Toutefois il ne faut pas essayer à tout prix de passer une technique d'entrée en lutte, car cela pourrait impliquer de passer à côté d'autres occasions pour frapper à l'épée, et aussi de prendre plus de risques.
% Ainsi sur le plan cognitif il faut rester prêt aux éventualités possibles.
