\chapter{Bâton long}


Le bâton long est un simple bâton dont la taille est approximativement de deux mètres.
Il faut se rappeler lors de la pratique amicale que le bâton est une arme en tant que telle : nous ne travaillons pas avec un simulateur.

Une main tient l'extrémité et l'autre est placée quelques paumes plus loin : l'écartement entre les mains est approximativement égal à la largeur des épaules ou un peu plus grand.
Plus l'écartement est grand plus la main avant sera exposée aux frappes de l'adversaire, mais il s'agit aussi d'une prise plus facile au début.
% Katori : largeur des épaules
Il existe deux manières de tenir le bâton :
\begin{itemize}
	\item prise lance : la main avant est en supination, les deux pouces sont tournés vers l'avant ;
	\item prise bâton : la main avant est en pronation, les deux pouces sont tournés l'un vers l'autre.
\end{itemize}
Nous nous occuperons principalement de la seconde prise.
Par main avant nous désignerons toujours la main qui se trouve le plus en avant sur le bâton, pas nécessairement la main qui se trouve la plus proche de l'adversaire.

Les attaques au bâton sont principalement des coups de taille armés sur le côté, des coups de taille circulaire et des estocs.


\begin{coup}[Frappe diagonale descendante au bâton]
\index{coup!bâton long}

Le coup est armé en plaçant la main avant au niveau de l'épaule et la main arrière au niveau du plexus, le bâton est dirigé vers le haut.
Le pied en arrière est le même que la main à l'avant du bâton.
La frappe se fait en abattant le bâton selon une direction diagonale grâce à un mouvement de hanche.
\end{coup}


\begin{coup}[Frappe diagonale ascendante au bâton]

Le coup est armé en plaçant la main avant au niveau de la hanche droite et la main arrière au niveau de la hanche gauche, le bâton est horizontal et dirigé vers l'arrière.
Le pied en arrière est le même que la main à l'avant du bâton.
La frappe se fait remontant les deux mains au niveau de la tête.
\end{coup}


% TODO: améliorer
\begin{coup}[Frappe droite descendante au bâton]

Le coup est armé en plaçant la main avant au niveau de l'épaule et la main arrière au niveau du plexus, le bâton est dirigé vers le haut.
Le pied en arrière est le même que la main à l'avant du bâton.
La frappe se fait en abattant le bâton droit devant soi (les mains se retrouvent alignées au centre de la personne).
\end{coup}


La partie qui heurte l'adversaire doit être l'extrémité du bâton car il s'agit de la partie qui possède le plus d'énergie cinétique.
Les cibles au bâton long sont les suivantes : les tempes, la nuque, les omoplates, les coudes, les poignets et les genoux (en résumé la plupart des articulations).
\index{cible!bâton long}
% TODO: schéma

L'idée du bâton long est d'armer très rapidement après avoir frappé : ainsi l'adversaire se trouve sous la menace d'une nouvelle frappe et ne peut pas prendre l'initiative (surtout s'il a une arme plus courte).
La manière d'armer demande de l'entraînement pour être réalisé rapidement et efficacement, mais elle est assez naturelle : il s'agit de tirer le bâton avec la main arrière et de le laisser glisser entre les doigts légèrement desserrés.
Ce mouvement est très rapide et permet de soustraire notre bâton à l'adversaire.
En effet plus notre bâton reste longtemps près de l'adversaire plus celui-ci peut en prendre le contrôle (et ce même s'il est bien verrouillé -- voire plus bas).
Pour cette raison après une attaque il est plus prudent soit d'enchaîner directement avec une autre attaque, soit de réarmer en arrière.
Pendant le bref instant où le bâton se trouve en avant on doit se retrouver dans une des gardes décrites plus bas.

% Les estocs sont très naturels au bâton et visent principalement le ventre, le torse et la tête.


\index{verrouillage}
Un concept important au bâton (et aux armes d'hast en générale) est celui de verrouillage : la main (ou l'avant-bras) arrière doit toujours être bloquée contre une partie du corps.
Cela permet de s'assurer que l'adversaire ne peut pas éjecter facilement le bâton (par exemple avec un enroulé).
Pour une garde basse le verrouillage se fait contre la hanche arrière tandis que pour une garde haute il se fait à la tête.


\begin{garde}[Garde basse]
\index{garde!bâton long}

La main arrière est calée contre la hanche, et le coude avant près du corps.
Le bâton est étendu devant soi.
\end{garde}


\begin{garde}[Garde haute]

La main arrière est au-dessus de la tête, l'autre devant, avec les doigts dépliés derrière le bâton (pour éviter les coups dessus).
\end{garde}


En garde haute les attaques hautes sont bloquées avec la partie comprise entre les deux mains.

En position de garde il est important que le bâton de l'adversaire ne soit pas derrière le nôtre (c'est-à-dire du côté où se trouvent nos bras) car il lui est alors facile de rentrer en lutte ou d'attaquer les bras (et la partie arrière du bâton n'est plus utilisable).
Ainsi selon les gardes respectives, soit aucun n'est désavantagé, soit un l'est, soit les deux le sont.
On définira la garde "normale" comme celle où aucun n'est avantagé (par exemple les deux ont le pied gauche en avant et chacun est protégé par son bâton).


\begin{exercice}

\obj{S'entrainer à enchainer des coups.}

\begin{enumerate}
	\item À partir de la garde tirer sur le bâton avec la main arrière pour le ramener derrière.
	
	\item Abaisser les bras (en "fouettant") pour porter l'attaque.
	
	\item Se retrouver dans la garde opposer.
\end{enumerate}

De même en faisant une attaque montante on se retrouve en garde haute.
\end{exercice}


\begin{exercice}

Vérifier qu'en garde haute il suffit de peu d'efforts pour bloquer les attaques simples.
\end{exercice}
