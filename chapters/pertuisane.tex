\chapter{Pertuisane}


\section{Saviolo}


Dans ce style de combat l'idée est de se déplacer sur un cercle~\cite{livermore:cornucopia:partizan:2014}.
Au début d'une série, le pied ne se pose pas complètement (i.e. seul le talon est posé) afin de permettre de changer facilement de direction ou de cible.

La main gauche est très mobile (donc pas nécessairement verrouillée à la hanche ou à la tête) et permet de changer la position de la pointe. Ainsi avec la main au niveau de la hanche la tête de l'arme est vers le haut, et inversement avec la main près de l'épaule la pointe est vers le bas.

Au moment où l'un des deux adversaires a sa hampe en contact avec celle de l'autre il peut en profiter pour faire levier et l'écarter ou la coincer au sol. Dans ce dernier cas il ne reste plus qu'à ramener l'arme pour trancher au visage.


\begin{technique}

\begin{enumerate}
	\item \A et \D se font face, pieds droits en avant, l'arme à l'intérieur de l'autre.
	
	\item \A fait un pas sur la droite (le pied gauche est immobile) ; seul le talon est posé. En remontant la main gauche, il fait passer la pointe sous l'arme de \D et vient battre celle-ci pour l'écarter.
	
	\item \A avance encore son pied et estoque au niveau du ventre (main gauche au niveau de la tête).
	
	\item \D lève sa main gauche et ramène sa pertuisane pour se protéger du coup.
	
	\item \alt{Si \A ne réagit pas, \D écarte l'arme \A et l'amène au sol avant de le trancher avec son arme.}
\end{enumerate}

Noter que le pied arrière de \A ne bouge à aucun moment.

% \source{Paul et Raphaël (ENS), d'après~\cite{livermore:cornucopia:partizan:2014}}
\end{technique}


\begin{technique}

Cet exercice commence comme le précédent, mais l'attaque de \A au point 3) est une feinte.
\begin{enumerate}
	\item \A et \D se font face, pieds droits en avant, l'arme à l'intérieur de l'autre.
	
	\item \A fait un pas sur la droite, passe sa pertuisane sous celle de \D et l'écarte d'un batté.
	
	\item \A fait comme s'il allait porter un estoc (descendant) mais n'avance pas le pied.
	
	\item \D lève la main gauche pour parer l'estoc avec la hampe.
	
	\item Avant le contact avec la hampe de \D, \A ramène sa main gauche au niveau de la hanche pour faire passer son arme au-dessus de celle de \D afin de l'abattre sur le bras de \D.
	
	\item \D se protège en tournant les hanches vers la droite.
\end{enumerate}

Le déplacement du pied à l'avant-dernier temps (5) n'est pas clair. On peut déplacer le pied gauche sur le côté. On peut aussi ne pas bouger, ou bien avancer le pied droit (ce dernier mouvement peut être dangereux car la tête de la pertuisane de \D peut être très proche). Finalement Raphaël faisait un pas sur le côté en croisant les bras (à l'allemande).

Pour réussir à se protéger au temps (6), il faut pour cela que \D ne se soit pas précipité sur son premier contre et qu'il ait bien avancé le pied droit afin de gagner le temps nécessaire.

% \source{Paul et Raphaël (ENS), d'après~\cite{livermore:cornucopia:partizan:2014}}
\end{technique}
