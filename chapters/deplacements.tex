\chapter{Déplacements}


%%%%%%%%%%%%%%%%%%%%%%
\section{Déplacements}
%%%%%%%%%%%%%%%%%%%%%%



\begin{exercice}
\label{ex:general:miroir}

\A et \D se font face et se déplacent en miroir.
\end{exercice}


%%%%%%%%%%%%%%%%
\section{Chutes}
%%%%%%%%%%%%%%%%

Savoir bien chuter est utile dans le cadre de la lutte, mais aussi lors de spectacles.
% enlever la peur de tomber

% Romain
Quelques principes :
\begin{itemize}
	\item chute en avant : mettre les avants bras, ne pas tomber sur les coudes ou les genoux ;
	\item chute en arrière : remonter les épaules pour protéger la tête, se tourner un peu pour atterrir sur l'épaule et pas sur les coudes.
\end{itemize}
Une chute sur une articulation peut faire très mal si l'on porte une armure en plus.


\begin{exercice}[Chute]

Se laisser tomber sur un tapis.

\end{exercice}


\begin{exercice}[Chute en hauteur]

Se laisser tomber d'un banc ou d'une table sur un tapis.

Source : Romain.

\end{exercice}


\begin{exercice}
\A et \D ont leur hanche droite en contact et regardent dans des directions opposées.
\A appuie et fait tomber \D en arrière.

Source :~\cite{petit:dijon:close_longword:2015}.
\end{exercice}


\begin{exercice}
\A et \D regardent dans la même direction.
\A a sa hanche gauche contre la hanche droite de \D.
\A appuie sur le torse de \D pour le faire tomber en arrière.

Source :~\cite{petit:dijon:close_longword:2015}.
\end{exercice}


\begin{exercice}
Idem que le premier exercice, mais \A mais sa jambe bien plus loin.
En général \D va tomber en avant.

Source :~\cite{petit:dijon:close_longword:2015}.
\end{exercice}


\begin{exercice}[Roulade par dessus un obstacle]
Mettre un tapis derrière un banc et faire une roulade sur le tapis sans toucher le banc.
\end{exercice}

