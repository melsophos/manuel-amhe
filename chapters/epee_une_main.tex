\chapter{Épée à une main}


Les gardes de prime et de second sont efficaces contre les estocs et les tailles.


\begin{technique}

\begin{enumerate}
	\item \A attaque épaule ou jambe.
	
	\item \D esquive dans l'intérieur et frappe au poignet.
\end{enumerate}

% Romain
\end{technique}


\begin{technique}

\begin{enumerate}
	\item \A attaque l'épaule droite en revers.
	
	\item D pare en sixte en se déplaçant sur l'extérieur.
	
	\item \D appuie avec sa main gauche sur les mains de \A et vient frapper gorge.
\end{enumerate}

Quand il appuie \D ne doit pas retirer son épée.

% Romain
\end{technique}


\begin{technique}

\begin{enumerate}
	\item \A attaque avec un estoc.
	
	\item \D pare et ré-attaque avec un estoc.
	
	\item \A se baisse et passe en prime. 
	
	\item \A vient saisir le bras de \D et en avançant la jambe droite frappe de taille.
\end{enumerate}

Sur le dernier temps \A doit rester très près pour éviter le coup double.
Cette technique peut s'utiliser avec un bocle.

% Romain
\end{technique}

