\chapter{Épée à une main}


Les gardes de prime et de second sont efficaces contre les estocs et les tailles.


\begin{exercice}

\begin{enumerate}
	\item \A porte un coup horizontal sur l'épaule droite de \D.
	
	\item \D pare en tierce.
	
	\item \D étend son bras pour frapper horizontalement à l'épaule gauche de \A
	
	\item \A ramène son épée (en gardant le contact du fer) au dessus de sa tête.
	
	\item \A attaque de l'autre côté (plutôt au ventre).
\end{enumerate}

Arrivé au dernier temps l'enchaînement recommence en inversant les rôles (\D peut garder le contact du fer lors de la transition).

Pour se protéger efficacement, \A doit être fléchi sur ses jambes et garder son épée en diagonale, de sorte à ce que si l'épée de \D glisse le long de la lame, alors elle "tombe" à côté.
\end{exercice}


\begin{exercice}

\begin{enumerate}
	\item \A attaque aux épaules en diagonale ou verticalement.
	
	\item \D esquive en levant son épée (les mains sont du même côté que l'esquive) pour se protéger au cas où.
	
	\item \D porte un coup (vérifier que la distance est tout juste bonne).
\end{enumerate}

Une autre possibilité est de placer directement l'épée pendant l'esquive.
\end{exercice}


\begin{exercice}

Enchaînement de parades et de ripostes.
Combinaisons possibles :
\begin{itemize}
	\item quarte puis seconde pour écarter la lame, suivie d'une attaque sur la seconde, et poursuivre (quand le rythme s'accélère il suffit de lever le pied sans reculer) ;
	\item parade quarte puis attaque quinte, parade quinte attaque quarte, parade quinte (avec l'épée inversée).
\end{itemize}
\end{exercice}


\begin{exercice}

L'attaquant porte un coup sur la quinte et le défenseur a le choix entre plusieurs ripostes :
\begin{itemize}
	\item démarrant pied gauche en avant, il se décale sur le côté en avançant le pied droit en premier, puis le gauche, en portant un coup direct sur la nuque de son opposant. Il finit en garde ;
	\item il se baisse en plaçant son épée au-dessus de lui, la pointe du côté gauche, puis il remonte en avançant pour trancher les bras de l'adversaire ;
	\item il bloque en quarte en pointant son épée vers l'ennemi (pour en principe finir sur un estoc) ;
	\item il esquive en se baissant sur le côté gauche et en se protégeant la tête avec son épée, puis il avance le pied droit et attaque ;
	\item il donne un coup sur le côté gauche de la lame adverse pour la dégager vers la droite et finit en position pour un estoc.
\end{itemize}
\end{exercice}


\begin{technique}

\begin{enumerate}
	\item \A attaque épaule ou jambe.
	
	\item \D esquive dans l'intérieur et frappe au poignet.
\end{enumerate}

% Romain
\end{technique}


\begin{technique}

\begin{enumerate}
	\item \A attaque l'épaule droite en revers.
	
	\item D pare en sixte en se déplaçant sur l'extérieur.
	
	\item \D appuie avec sa main gauche sur les mains de \A et vient frapper gorge.
\end{enumerate}

Quand il appuie \D ne doit pas retirer son épée.

% Romain
\end{technique}


\begin{technique}

\begin{enumerate}
	\item \A attaque avec un estoc.
	
	\item \D pare et ré-attaque avec un estoc.
	
	\item \A se baisse et passe en prime. 
	
	\item \A vient saisir le bras de \D et en avançant la jambe droite frappe de taille.
\end{enumerate}

Sur le dernier temps \A doit rester très près pour éviter le coup double.
Cette technique peut s'utiliser avec un bocle.

% Romain
\end{technique}


\begin{technique}

\begin{enumerate}
	\item \A attaque à l'épaule gauche.
	
	\item \D pare en prime haute (pied droit en avant).
	
	\item \D déplace son pied gauche en pivotant autour de son côté droit et en passant sous son bras.
	
	\item \D percute le poignet de l'adversaire de la main gauche, déplace son pied droit latéralement.
	
	\item \D ramène son pied gauche à l'arrière et accompagne le mouvement d'une coupe horizontale au niveau du ventre ou de la cuisse (en faisant passer d'abord l'épée dans le dos).
\end{enumerate}

Il faut veiller à être tout juste à la bonne distance pour couper de la pointe.
Au temps 5) le mouvement est obtenu grâce à une rotation des hanches, qui lance aussi le coup naturellement.
\end{technique}


\begin{technique}

\begin{enumerate}
	\item \A attaque à l'épaule droite.
	
	\item \D bloque en tierce, le bras sur le côté, en pivotant légèrement.
	
	\item Sans bouger les pieds \D pivote ensuite les hanches vers la gauche ce qui permet de ramener le tranchant vers \A.
	
	\item \D avance pour frapper l'adversaire.
\end{enumerate}

En réalité cette technique est peu puissante car le bras est peu armé.
Son intérêt peut être d'entrer en lutte.
Par contre il est possible de remplacer l'attaque par un estoc, beaucoup plus efficace.

Cette technique peut se faire aussi de l'autre côté.
\end{technique}
