\chapter{Bâton court}


\begin{technique}

\A est armé d'une épée une main.

\begin{enumerate}
	\item \A attaque l'épaule droite.
	
	\item \D contrôle la lame en tierce.
	
	\item \D attrape l'autre extrémité (s'il ne la tenait pas déjà) et avance pour placer le bâton entre le bras et le torse, en plaquant la lame contre le corps.
\end{enumerate}

Cette technique peut être suivie d'un désarmement.
\end{technique}


\begin{technique}

\A est armé d'une épée longue.

\begin{enumerate}
	\item \A attaque verticalement.
	
	\item \D se jette en avant et bloque en quinte en tenant le bâton avec une main à chaque extrémité, afin de bloquer la lame ou, mieux, les poignets.
\end{enumerate}

\D peut enchainer de différentes manières, par exemple en poussant pour ramener les bras de l'adversaire dans son dos.
\end{technique}

