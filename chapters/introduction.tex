\chapter{Introduction}


%%%%%%%%%%%%%%%%%%
\section{Objectif}
%%%%%%%%%%%%%%%%%%


\index{\textsc{Amhe}}
\index{arts martiaux traditionnels}

Ce manuel est une compilation d'outils, de techniques et d'exercices pour pratiquer les \emph{arts martiaux historiques}, principalement \emph{européens} (\textsc{Amhe}) et incluant l'escrime médiévale~\footnotemark{}.
\footnotetext{Un terme plus adapté pourrait être simplement \emph{arts martiaux traditionnels}.
En effet il est fréquent de chercher l'inspiration dans d'autres styles, modernes ou asiatiques, lorsque les sources font défaut ; de plus l'étude d'autres arts martiaux pourrait trouver sa place dans cette approche (comme les arts du combat en Amérique pré-colombienne).}
Le but n'est pas de fournir une étude détaillée d'une arme ou d'un style particulier mais plutôt d'offrir un panel dans lequel on pourra piocher pour découvrir ses goûts et acquérir une culture générale.
Diverses annexes complètent le texte, par exemple sur les principaux traités et sur la fabrication de matériel.
En particulier les convention adoptées sont décrites dans l'annexe~\ref{app:conventions}.


%%%%%%%%%%%%%%%%%%%
\section{Ce manuel}
%%%%%%%%%%%%%%%%%%%


L'interprétation et l'enseignement des \textsc{Amhe} comportent une certaine part de sensibilité personnelle.
En particulier ce manuel reflète notre manière d'aborder l'escrime et a été influencée par les échanges au cours de notre pratique, mais rien ne garantit que notre interprétation soit juste ni que notre approche convienne à tous.
Le lecteur est encouragé de réfléchir et d'étudier les sources par lui-même avant de se convaincre que la description est correcte.
Finalement nous rappelons aussi que l'interprétation évolue au cours du temps~\footnotemark{}.
\footnotetext{Cela peut se voir en comparant les vidéos sur Youtube au sujet d'un même manuscrit à une dizaine d'années d'intervalle.}
% Finalement il est toujours utile d'étudier des interprétations qui diffèrent des nôtres et de les garder sous la main afin de fournir une base de travail.
Une grande partie du matériel présenté est issu des cours donnés par le Chapitre des armes et le club d'escrime ancienne de l'\textsc{Ens}.

Ce manuel est un \textbf{brouillon} et la qualité des différents chapitres est inégale, toutefois nous pensons que ce texte peut déjà servir en l'état actuel.
À terme l'objectif est de compléter chaque chapitre avec une introduction générale, des références aux sources, des exercices et des techniques, des schémas et des figures ainsi qu'une analyse du style.
Les remarques, critiques et suggestions sont extrêmement bienvenues et peuvent être soumises sur Github~\footnotemark{} (méthode privilégiée) ou envoyées par mail.
\footnotetext{\url{https://github.com/melsophos/manuel_amhe/issues}}

L'intégralité des sources Latex peut être téléchargées sur Github~\footnotemark{}.
\footnotetext{\url{https://github.com/melsophos/manuel_amhe}}
De cette manière n'importe quelle partie du texte peut être réutilisée, par exemple pour un document plus spécialisé ou pour préparer un atelier (à condition de respecter la licence Art libre).
Les modifications entre deux versions peuvent être trouvées dans le fichier \texttt{CHANGES} qui se trouve sur Github.


%%%%%%%%%%%%%%%%%%%%%%%%%%%%%%%%%%%%%
% \section{Arts martiaux traditionnels}
%%%%%%%%%%%%%%%%%%%%%%%%%%%%%%%%%%%%%



%%%%%%%%%%%%%%%%%%
\section{Pratique}
%%%%%%%%%%%%%%%%%%


\noindent
La pratique des \textsc{Amhe} contient plusieurs aspects ainsi que des activités connexes :
\begin{itemize}
	\item la pratique en groupe (sous la tutelle d'un instructeur) ;
	\item l'étude des sources historiques et l'essai en groupe réduit ;
	\item le combat libre (mise en pratique en situation "réelle") ;
	\item les compétitions sportives (tournois) ;
	\item les démonstrations et spectacles (par exemple en fête médiévale).
\end{itemize}
La pratique inclue l'étude de techniques accompagné d'exercices préparatoires~\footnotemark{}.%
\footnotetext{À notre avis l'escrime médiévale manque de katas (séquence de techniques) : le fait de connaître la série permet de se concentrer sur l'amélioration des techniques se trouvant à l'intérieur, et la répétition fréquente permet de fournir de nombreuses occasions.
	De plus les katas permettent de travailler dans certaines conditions qui ne seraient pas sûres autrement : dans le noir, sans protection mais à grande vitesse, etc.}
De plus les entraînements comprennent souvent une part de musculation afin d'être en meilleure forme physique et d'éviter les blessures (à court ou long terme).
Il n'est pas forcément nécessaire d'étudier soi-même les sources pour faire des progrès pendant un certain temps.
De même, bien que le combat libre est très utile afin de progresser, rien n'oblige à en faire, et encore moins à participer à des tournois.
Finalement il peut être intéressant d'étudier l'escrime de spectacle afin de faire de belles démonstrations et de mieux partager notre intérêt, mais encore une fois ce n'est pas une obligation.
Ainsi il n'y a pas de programme fixée une fois pour toute et chacun peut adapter son étude par rapport à ses centres d'intérêts.


%%%%%%%%%%%%%%%%%
\section{Sources}
%%%%%%%%%%%%%%%%%


Ce manuel a commencé à croître à partir des cours donnés au club d'escrime ancienne de l'\textsc{Ens} (Paris) par Romain Wenz, Thomas Mainguy et Samuel Baumard.
À cette base se sont ajoutées des idées provenant de mes recherches personnelles, de discussions, de stages et de lectures (blog, livres…).
Pour cette raison il est très difficile d'attribuer précisément l'origine de chaque idée et de donner des références exhaustives, en particulier à tous les articles de blog.
Un autre problème se pose lorsqu'une même technique est présentée par différents auteurs avec des variations : dans ce cas j'ai adopté l'interprétation qui me paraît le plus juste, parfois en faisant une synthèse des différentes approches.
Même lorsqu'un exercice ou une technique ne provient d'un autre auteur j'ai tâché de l'adapter en fonction de mon approche et du contenu de ce manuel : ainsi il ne s'agit plus de la version proposée originellement, si bien que l'auteur pourrait éventuellement ne pas être d'accord avec mon interprétation.
Finalement un dernier argument est de ne pas alourdir le texte.
Toutefois je me suis efforcé d'indiquer quand un instructeur m'a particulièrement inspiré pour le contenu d'une section.
% Malgré tout je me suis efforcé d'apporter des références précises aux divers auteurs dont je me suis inspiré, en particulier dans les cas suivants : description d'un exercice spécial ou d'un atelier, étude (technique, historique…) détaillée.
De plus la source historique est indiquée pour toute technique où j'ai pu l'identifier.


\index{Wiktenauer}
Le wiki Wiktenauer~\cite{wiktenauer} représente une importante source d'informations : on peut y trouver une biographie des grands maîtres, la transcription (et souvent la traduction) accompagnée des éventuelles illustrations d'un grand nombre de manuscrit ainsi que des liens vers les travaux correspondants (interprétations, livres et traductions dans d'autres langues).
Il s'agit de la première ressource à consulter lorsque l'on cherche une information.

Comme tout domaine spécialisé, les \textsc{Amhe} ont développé un vocabulaire technique assez important, formé en partie des termes employés dans les manuscrits dans la langue d'origine et de leurs traductions.
Pour cette raison de nombreux documents incluent un glossaire, et au besoin on pourra se reporter à des glossaires plus généraux~\cite{Forgeng:2005:FechtkunstGlossary, FIE:2014:BrefsGlossairesLescrime, SalleArmes:2013:GlossaireEscrime}.

Finalement un certain nombre de blogs sur les \textsc{Amhe} existent et contiennent des billets sur des sujets variés (pédagogie, interprétation des sources, contexte historique, etc.).
Parmi les plus actifs et intéressants se trouvent : Hroarr~\cite{Blog:Hroarr} (Norling et al.), l'OGN~\cite{Blog:OGN}, Devon Boorman~\cite{Blog:Boorman}, Hans Talhoffer~\cite{Blog:HansTalhoffer} (Kleinau).


%%%%%%%%%%%%%%
\section{Plan}
%%%%%%%%%%%%%%


Tout d'abord le livre n'est pas organisé de manière tout à fait linéaire : les chapitres (surtout au début) sont censés être progressifs, mais parfois nous utilisons des éléments introduits dans des chapitres ultérieurs.
Cela est d'autant plus vrai pour certains exercices qui peuvent utiliser des concepts vus plus tard. 
% Par exemple le chapitre sur les déplacements précède celui sur les attaques et les défenses, mais certains exercices requièrent l'utilisation d'attaques.
Le lecteur est donc encouragé à mettre de côté les parties en question et à y revenir plus tard, ou bien à utiliser l'index pour trouver les informations nécessaires.

La partie~\ref{part:notions-générales} décrit les notions générales telles que la structure du corps, les déplacements, l'attaque et la défense.
Nous conseillons de commencer la lecture par ce chapitre, surtout pour les débutants, et de pratiquer régulièrement les exercices qui y sont donnés.
Ces chapitres requièrent l'utilisation d'une arme mais il n'est pas nécessaire d'avoir étudier les autres parties pour cela car les exercices sont généraux et adaptables à la plupart des armes.

Les parties suivantes sont consacrées à l'étude des armes, qui sont regroupées par grandes familles.
Ces parties et leurs chapitres sont globalement indépendants les uns des autres.

La partie~\ref{part:applications} contient différentes applications et compléments.
Ces chapitres sont relativement indépendants des autres parties.

Finalement différentes annexes apportent des informations supplémentaires sur des sujets connexes.
En particulier les conventions sont décrites dans l'annexe~\ref{app:conventions}.
Un index à la fin du manuel permet de retrouver rapidement une notion recherchée.
Une liste des définitions, des coups et des gardes est donnée dans l'annexe~\ref{app:listes}.

% Nous conseillons au lecteur débutant de jeter un œil aux introductions de chaque chapitre


%%%%%%%%%%%%%%%%%%%%%%%
\section{Remerciements}
%%%%%%%%%%%%%%%%%%%%%%%


Ce manuel doit beaucoup aux membres du Chapitre des armes, du club de l'\textsc{Ens} et du club de Katori, et en particulier aux personnes suivantes : Samuel Baumard, Jean-Paul Blond, Arthur Boutillon, Lionel Cambos, Agnès Daipra, Thomas Mainguy, Paul Melotti, Jan Orkisz, Serge Rajevic, Raphaël Rigal, Léo Vallet, Romain Wenz.
