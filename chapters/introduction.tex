\chapter{Introduction}


\section{Pratique}

Les exercices et les techniques proposées permettent de travailler un élément précis.
Il existe certainement des contres plus judicieux (surtout si l'on sait ce que l'autre est censé faire) ou d'autres manières de procéder, mais alors on perd l'intérêt de faire cet exercice et le partenaire ne peut plus travailler dans de bonnes conditions.
Cela ne veut pas dire qu'il ne faut pas résister – au contraire il est intéressant de varier le degré de résistance pour progresser –, mais il ne faut pas changer totalement les lignes de la technique (sauf avec accord du partenaire quand l'on souhaite explorer d'autres possibilités).


\section{Conventions}

Dans les exercices chaque point correspond à un temps. Ainsi même si aucune conjonction n'est précisée entre deux phrases, il faut comprendre que les deux actions doivent avoir lieu en même temps. Exemple : « Faire un pas à droite. Porter un estoc. » signifie que l'estoc a lieu en même temps que le pas à droite.

Les crochets servent à indiquer comment arrêter l'exercice à un temps donné.

\A désigne l'attaquant (c'est-à-dire le premier à agir) et \D le défenseur.

On appellera~\footnote{Ces dénominations viennent des attaques avec une épée courte. Elles ne sont pas aussi visuelles pour les autres armes – hast, épée longue… – mais elles permettent d'avoir une dénomination unique.} diagonale droite la diagonale NO–SE, et diagonale inverse la diagonale NE–SO.

On utilisera le terme de croix pour une épée afin de désigner la croix formée par la poignée et la garde.
Celle-ci servira de cible pour certains exercices (l'épée en nylon étant tenue par la lame).

Dans certains cas les temps sont extrêmement décomposés, mais une exécution fluide de la technique conduira à "fusionner" certains temps.

\section{Remerciements}

Ce manuel doit beaucoup aux membres du Chapitre des armes et du club de l'ENS : Arthur, Jan, Paul, Raphaël, Romain, Samuel, Thomas…
