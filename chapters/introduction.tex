\chapter{Introduction}


%%%%%%%%%%%%%%%%%%
\section{Objectif}
%%%%%%%%%%%%%%%%%%


\index{\textsc{Amhe}}
\index{Arts martiaux traditionnels}
Ce manuel est une compilation d'outils, de techniques et d'exercices pour pratiquer les \emph{arts martiaux historiques}, principalement \emph{européens} (\textsc{Amhe}) et incluant l'escrime médiévale~\footnotemark{}.
\footnotetext{Un terme plus adapté pourrait être simplement \emph{arts martiaux traditionnels}.
En effet il est fréquent de chercher l'inspiration dans d'autres styles, modernes ou asiatiques, lorsque les sources font défaut ; de plus l'étude d'autres arts martiaux pourrait trouver sa place dans cette approche (comme les arts du combat en Amérique pré-colombienne).}
Le but n'est pas de fournir une étude détaillée d'une arme ou d'un style particulier mais plutôt d'offrir un panel dans lequel on pourra piocher pour découvrir ses goûts et acquérir une culture générale.
Diverses annexes compléteront le texte, par exemple sur les principaux traités et sur la fabrication de matériel.


%%%%%%%%%%%%%%%%%%%%%%%%%%%%%%%%%%%%%
\section{Arts martiaux traditionnels}
%%%%%%%%%%%%%%%%%%%%%%%%%%%%%%%%%%%%%


% TODO: historique

\index{\textsc{Amhe}}
\index{arts martiaux traditionnels}

\index{Wiktenauer}
Le wiki Wiktenauer~\cite{wiktenauer} représente une importante source d'informations : on peut y trouver une biographie des grands maîtres, la transcription (et souvent la traduction) accompagnée des éventuelles illustrations d'un grand nombre de manuscrit ainsi que des liens vers les travaux correspondants (interprétations, livres et traductions dans d'autres langues).

Les \textsc{Amhe} comportent une certaine part d'arbitraire et de sensibilité personnelle dans l'interprétation des sources.
Dans ce manuel nous donnons notre propre vision, influencée et dérivée des contacts avec les autres membres du Chapitre des armes et lors des stages, mais nous ne 
Le lecteur est encouragé de réfléchir et d'étudier les sources par lui-même avant de se convaincre que la description est correcte.
Finalement nous rappelons aussi que l'interprétation évolue au cours du temps~\footnotemark{}.
\footnotetext{Cela peut se voir en comparant les vidéos sur Youtube au sujet d'un même manuscrit à une dizaine d'années d'intervalle.}

Finalement il est toujours utile d'étudier des interprétations qui diffèrent des nôtres et de les garder sous la main afin de fournir une base de travail.


%%%%%%%%%%%%%%%%%%
\section{Pratique}
%%%%%%%%%%%%%%%%%%


Les exercices et les techniques proposées permettent de travailler un élément précis.
Il existe certainement des contres plus judicieux (surtout si l'on sait ce que l'autre est censé faire) ou d'autres manières de procéder, mais alors on perd l'intérêt de faire cet exercice et le partenaire ne peut plus travailler dans de bonnes conditions.
Cela ne veut pas dire qu'il ne faut pas résister – au contraire il est intéressant de varier le degré de résistance pour progresser –, mais il ne faut pas changer totalement les lignes de la technique (sauf avec accord du partenaire quand l'on souhaite explorer d'autres possibilités).

Il est possible de trouver des documents sur la manière d'organiser un cours~\cite{linnard:how_we_train}.


%%%%%%%%%%%%%%%%%%%%%
\section{Conventions}
%%%%%%%%%%%%%%%%%%%%%


Dans les exercices chaque point correspond à un temps. Ainsi même si aucune conjonction n'est précisée entre deux phrases, il faut comprendre que les deux actions doivent avoir lieu en même temps. Exemple : « Faire un pas à droite. Porter un estoc. » signifie que l'estoc a lieu en même temps que le pas à droite.

Les crochets servent à indiquer comment arrêter l'exercice à un temps donné.

\A désigne l'attaquant (c'est-à-dire le premier à agir) et \D le défenseur.

On appellera~\footnote{Ces dénominations viennent des attaques avec une épée courte. Elles ne sont pas aussi visuelles pour les autres armes – hast, épée longue… – mais elles permettent d'avoir une dénomination unique.} diagonale droite la diagonale NE–SO, et diagonale inverse la diagonale NE–SO.
\index{diagonale!droite}
\index{diagonale!inverse}

On utilisera le terme de croix pour une épée afin de désigner la croix formée par la poignée et la garde.
Celle-ci servira de cible pour certains exercices (l'épée en nylon étant tenue par la lame).

Dans certains cas les temps sont extrêmement décomposés, mais une exécution fluide de la technique conduira à "fusionner" certains temps.

Nous utiliserons parfois un vocabulaire tiré de styles proches afin de préciser le mouvement.
Nous utiliserons parfois le terme français, parfois le terme original selon ce qui nous semble le plus précis.


%%%%%%%%%%%%%%%%%%%%%%%
\section{Remerciements}
%%%%%%%%%%%%%%%%%%%%%%%


Ce manuel doit beaucoup aux membres du Chapitre des armes et du club de l'ENS : Arthur, Jan, Paul, Raphaël, Romain, Samuel, Thomas…
