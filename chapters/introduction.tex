\chapter{Introduction}


%%%%%%%%%%%%%%%%%%
\section{Objectif}
%%%%%%%%%%%%%%%%%%


\index{\textsc{Amhe}}
\index{arts martiaux traditionnels}

Ce manuel est une compilation d'outils, de techniques et d'exercices pour pratiquer les \emph{arts martiaux historiques}, principalement \emph{européens} (\textsc{Amhe}) et incluant l'escrime médiévale~\footnotemark{}.
\footnotetext{Un terme plus adapté pourrait être simplement \emph{arts martiaux traditionnels}.
En effet il est fréquent de chercher l'inspiration dans d'autres styles, modernes ou asiatiques, lorsque les sources font défaut ; de plus l'étude d'autres arts martiaux pourrait trouver sa place dans cette approche (comme les arts du combat en Amérique pré-colombienne).}
Le but n'est pas de fournir une étude détaillée d'une arme ou d'un style particulier mais plutôt d'offrir un panel dans lequel on pourra piocher pour découvrir ses goûts et acquérir une culture générale.
Diverses annexes complètent le texte, par exemple sur les principaux traités et sur la fabrication de matériel.
En particulier les convention adoptées sont décrites dans l'annexe~\ref{app:conventions}.

L'interprétation et l'enseignement des \textsc{Amhe} comportent une certaine part de sensibilité personnelle.
En particulier ce manuel reflète notre manière d'aborder l'escrime et a été influencée par les échanges au cours de notre pratique, mais rien ne garantit que notre interprétation soit juste ni que notre approche convienne à tous.
Le lecteur est encouragé de réfléchir et d'étudier les sources par lui-même avant de se convaincre que la description est correcte.
Finalement nous rappelons aussi que l'interprétation évolue au cours du temps~\footnotemark{}.
\footnotetext{Cela peut se voir en comparant les vidéos sur Youtube au sujet d'un même manuscrit à une dizaine d'années d'intervalle.}
% Finalement il est toujours utile d'étudier des interprétations qui diffèrent des nôtres et de les garder sous la main afin de fournir une base de travail.

Ce manuel contient notre propre vision,
Dans ce manuel nous donnons notre propre vision, influencée et dérivée des contacts avec d'autres escrimeurs, en particulier du Chapitre des armes et de l'\textsc{Ens}.

Ce manuel est un \textbf{brouillon}, ce qui signifie que toutes les sections ne sont pas également bien présentées ni complètes.
Malgré cela nous pensons que ce texte peut servir même en l'état actuel ; toute critique ou contribution visant à l'améliorer est appréciée.


%%%%%%%%%%%%%%%%%%%%%%%%%%%%%%%%%%%%%
% \section{Arts martiaux traditionnels}
%%%%%%%%%%%%%%%%%%%%%%%%%%%%%%%%%%%%%



%%%%%%%%%%%%%%%%%%
\section{Pratique}
%%%%%%%%%%%%%%%%%%


\noindent
La pratique des \textsc{Amhe} contient plusieurs aspects ainsi que des activités connexes :
\begin{itemize}
	\item la pratique en groupe (sous la tutelle d'un instructeur) ;
	\item l'étude des sources et l'essai en groupe réduit ;
	\item le combat libre (mise en pratique en situation "réelle") ;
	\item les compétitions sportives (tournois) ;
	\item les démonstrations et spectacles (par exemple en fête médiévale).
\end{itemize}
Finalement les entraînements comprennent souvent une part de musculation afin d'être en meilleure forme physique et d'éviter les blessures (à court ou long terme).
Il n'est pas forcément nécessaire d'étudier soi-même les sources pour faire des progrès pendant un certain temps.
De même, bien que le combat libre est très utile afin de progresser, rien n'oblige à en faire, et encore moins à participer à des tournois.
Finalement il peut être intéressant d'étudier l'escrime de spectacle afin de faire de belles démonstrations et de mieux partager notre intérêt, mais encore une fois ce n'est pas une obligation.
Ainsi il n'y a pas de programme fixée une fois pour toute et chacun peut adapter son étude par rapport à ses centres d'intérêts.


% TODO: déplacer dans un chapitre à part, avec la mentalité générale (objectif d'entraînement, etc.)
Avant de continuer nous souhaitons aborder un point important qui peut avoir un impact négatif sur les progrès que l'on peut faire.
Il est fréquent de rencontrer des étudiants qui vont dire que telle technique étudiée ne marche pas car il existe un contre efficace, éventuellement en accélérant beaucoup le geste.
Ceci n'est pas un bon état esprit pour aborder l'escrime car les exercices et les techniques proposés permettent de travailler un élément précis : ils sont créés de telles sortes à ce que l'accent soit mis sur un ou plusieurs points, avec pour objectif d'améliorer la compréhension de l'arme étudiée.
Il existe certainement des contres plus judicieux (surtout si l'on sait ce que l'autre est censé faire en avance ou que l'on va délibérément plus vite que lui) ou d'autres manières de procéder, mais alors on perd l'intérêt de faire cet exercice et le partenaire ne peut plus travailler dans de bonnes conditions.
Cela ne veut pas dire qu'il ne faut pas résister – au contraire il est intéressant de varier le degré de résistance pour progresser –, mais il ne faut pas changer totalement les lignes de la technique (sauf avec accord du partenaire quand l'on souhaite explorer d'autres possibilités).


%%%%%%%%%%%%%%%%%
\section{Sources}
%%%%%%%%%%%%%%%%%


Ce manuel a commencé à croître à partir des cours donnés au club d'escrime ancienne de l'\textsc{Ens} (Paris) par Romain, Thomas et Samuel.
À cette base se sont ajoutés des idées provenant de mes recherches personnelles, de discussion, de stages et de lectures (blog, livres…).
Pour cette raison il est très difficile d'attribuer précisément l'origine de chaque idée et de donner des références exhaustives, en particulier à tous les articles de blog.
Malgré tout je me suis efforcé d'apporter des références précises aux divers auteurs dont je me suis inspiré, en particulier dans les cas suivants : description d'un exercice spécial ou d'un atelier, étude (technique, historique…) détaillée.
De plus la source historique est indiquée pour toute technique où j'ai pu l'identifier.

\index{Wiktenauer}
Le wiki Wiktenauer~\cite{wiktenauer} représente une importante source d'informations : on peut y trouver une biographie des grands maîtres, la transcription (et souvent la traduction) accompagnée des éventuelles illustrations d'un grand nombre de manuscrit ainsi que des liens vers les travaux correspondants (interprétations, livres et traductions dans d'autres langues).
Il s'agit de la première ressource à consulter lorsque l'on cherche une information.

Finalement un certain nombre de blogs sur les \textsc{Amhe} existent et contiennent des billets sur des sujets variés (pédagogie, interprétation des sources, contexte historique, etc.).
Parmi les plus actifs et intéressants se trouvent : Hroarr~\cite{Blog:Hroarr} (Norling et al.), l'OGN~\cite{Blog:OGN}, Devon Boorman~\cite{Blog:Boorman}, Hans Talhoffer~\cite{Blog:HansTalhoffer} (Kleinau).


%%%%%%%%%%%%%%
\section{Plan}
%%%%%%%%%%%%%%


La partie~\ref{part:notions-générales} décrit les notions générales telles que la structure du corps, les déplacements, l'attaque et la défense.
Nous conseillons de commencer la lecture par ce chapitre, surtout pour les débutants, et de pratiquer régulièrement les exercices qui y sont donnés.
Ces chapitres requièrent l'utilisation d'une arme mais il n'est pas nécessaire d'avoir étudier les autres parties pour cela car les exercices sont généraux et adaptables à la plupart des armes.

Les parties suivantes sont consacrées à l'étude des armes, qui sont regroupées par grandes familles.
Ces parties et leurs chapitres sont globalement indépendants les uns des autres.

La partie~\ref{part:applications} contient différentes applications et compléments.
Ces chapitres sont relativement indépendants des autres parties.

Finalement différentes annexes apportent des informations supplémentaires sur des sujets connexes.
En particulier les conventions sont décrites dans l'annexe~\ref{app:conventions}.
Un index à la fin du manuel permet de retrouver rapidement une notion recherchée.
Une liste des définitions, des coups et des gardes est donnée dans l'annexe~\ref{app:listes}.


%%%%%%%%%%%%%%%%%%%%%%%
\section{Remerciements}
%%%%%%%%%%%%%%%%%%%%%%%


Ce manuel doit beaucoup aux membres du Chapitre des armes, du club de l'\textsc{Ens} et du club de Katori, et en particulier aux personnes suivantes : Agnès, Arthur, Jan, Jean-Paul, Léo, Lionel, Paul, Raphaël, Romain, Samuel, Thomas.
