\chapter{Introduction}

\section{Conventions}

Dans les exercices chaque point correspond à un temps. Ainsi même si aucune conjonction n'est précisée entre deux phrases, il faut comprendre que les deux actions doivent avoir lieu en même temps. Exemple : « faire un pas à droite. Porter un estoc. » signifie que l'estoc a lieu en même temps que le pas à droite.

Les crochets servent à indiquer comment arrêter l'exercice à un temps donné.

\A désigne l'attaquant (c'est-à-dire le premier à agir) et \D le défenseur.

On appellera~\footnote{Ces dénominations viennent des attaques avec une épée courte. Elles ne sont pas aussi visuelles pour les autres armes – hast, épée longue… – mais elles permettent d'avoir une dénomination unique.} diagonale droite la diagonale NO–SE, et diagonale inverse la diagonale NE–SO.

\section{Remerciements}

Ce manuel doit beaucoup aux membres du Chapitre des armes et du club de l'ENS : Arthur, Jan, Paul, Raphaël, Romain, Samuel, Thomas…
