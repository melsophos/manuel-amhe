\chapter{Épée et bouclier viking}


Le bouclier viking est circulaire et plat.
Plusieurs armes peuvent être utilisées en combinaison avec le bouclier : dans ce chapitre nous discuterons de l'épée.
Les cibles possibles sont : la jambe sous le bouclier (derrière le genou) et le côté opposé.
Il faut utiliser le manque de visibilité pour attaquer.
De plus il est important de toujours protéger le bras qui attaque.
Si l'adversaire ne protège pas son bras il suffit d'attaquer cette cible pour mettre fin au combat.

Il faut noter que le bouclier est une arme par lui-même : une attaque fréquente consiste à donner un coup du poing gauche, ce qui revient à donner un coup horizontal du bouclier.

\noindent
Quelques idées pour attaquer :
\begin{itemize}
	\item pied gauche en avant, attaque droit devant (avec ou sans changement de pied) ;
	\item à partir de l'attaque précédente ou en garde, tourner le poignet gauche vers la droite, ce qui inverse verticalement le bouclier, et passer le bouclier par dessus le bras droit.
	En même temps, le poignet droit pivote vers la gauche – le tranchant se tourne vers le haut – en déplaçant le bras vers la gauche, et il ne reste plus qu'à attaquer ;
	\item l'attaque précédente peut se faire sans tourner l'épée, ce qui mène à une attaque de bas en haut ;
	\item elle peut aussi se faire en sens inverse pour revenir dans la position de départ.
\end{itemize}

Si \A attaque \D à la tête (peu importe le côté) alors \D a juste à lever le bouclier pour être protégé.
L'idée générale pour l'attaquant est de faire semblant d'attaquer en haut et d'essayer ensuite de frapper un autre endroit en tirant parti du fait que le défenseur ne peut pas le voir derrière son bouclier.
Plusieurs options sont possibles :
\begin{itemize}
	\item dévier légèrement l'attaque pour la faire passer à côté du bouclier et trancher la jambe ;
	\item contourner le bouclier par en dessous et remonter vers le ventre (en passant sous la cotte de maille) ;
	\item l'attaque précédente peut se faire aussi dans le dos ;
	\item enfin en remontant plus haut on peut viser la gorge ;
	\item et enfin, un peu moins convaincant, dévier pour amener l'épée derrière le genou ennemi et trancher les tendons.
\end{itemize}
En exercice le défenseur peut dire s'il a réussi à voir l'attaque venir.


\begin{technique}

\begin{enumerate}
	\item \A attaque.
	
	\item \D pare avec son bouclier.
	
	\item En avançant le pied gauche, \A percute \D (entre le coude et l'épaule gauche) avec son bouclier pour briser/engourdir le bras.
	
	\item \A attaque la jambe gauche avec son épée.
\end{enumerate}
\end{technique}


\begin{technique}

\begin{enumerate}
	\item \A feinte en attaquant \D.
	
	\item Si \D pare assez haut avec son bouclier, alors \A frappe l'épaule de son ennemi avec son bouclier en réarmant.
	
	\item \A abat son épée sur le bras (ce qui permet au moins de trancher la guige).
\end{enumerate}
\end{technique}


Si \D garde son bouclier bien devant soi (par exemple après une des deux techniques précédentes) alors \A peut donner un coup de son bouclier sur le côté gauche (vu de face) afin de le faire pivoter, ce qui découvre le flanc de \D.
\A peut alors porter un estoc.

Une autre technique consiste à faire en sorte de forcer l'adversaire à ouvrir sa garde, en attaquant d'abord sur sa droite (ce qui lui fait ramener son épée et son bouclier de ce côté), puis rapidement sur sa gauche (lui faisant déplacer son bouclier), et pendant qu'il est ouvert on peut donner un coup de bouclier sur l'épaule puis couper un bras.

\begin{technique}

\A et \D commencent jambe gauche en avant.

\begin{enumerate}
	\item \A attaque la jambe sous le bouclier (gauche).
	
	\item \D avance le bras gauche pour couvrir (mais sans le descendre, car le fait d'avancer suffit pour protéger).
	
	\item \D frappe avec son épée le dos de l'avant-bras de \A par au-dessus.
\end{enumerate}
\end{technique}

