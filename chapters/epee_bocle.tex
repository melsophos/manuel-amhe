\chapter{Épée et bocle}


% TODO: tenue avec le pouce

% exercices généraux


%%%%%%%%%%%%%%
\section{I.33}
%%%%%%%%%%%%%%


% type d'épée : Cinato : XVI, clubs allemands : XIV

% Royal Armouries, Tower Fechtbuch
Le traité MS I.33 (appelé aussi \emph{Walpurgis Fechtbuch} ou \emph{Liber de arte dimicatoria}) est le plus vieux manuel qui nous soit parvenu – il a été rédigé dans la décennie de 1320.
L'auteur serait un prêtre nommé Liutger.

La référence pour l'étude du I.33 est la traduction et le commentaire de Cinato et Suprenant~\cite{cinato:I33:2009}.
Kenner a écrit un manuel sur le sujet~\cite{kenner:I33:2014}.
Les images en couleur peuvent être trouvées sur le site de Wiktenauer~\cite{wiktenauer:I33}.


%%%%%%%%%%%%%%%%%%%%
\section{Liegniczer}
%%%%%%%%%%%%%%%%%%%%


Le traité de Liegniczer sur l'épée-bocle est très court et contient seulement six enchaînements.
On peut en trouver diverses transcriptions/traductions~\cite{ardamhe:liegniczer, farrell:liegnieczer, lindholm:ringeck_others:2006} ainsi que des interprétations~\cite{farrell:pedagogy_liegnieczer:2014, youtube:sala_armi:liegniczer, youtube:memag:liegniczer, lindholm:ringeck_others:2006, Myers:LiegniczerBuckler, knight:epee_bocle}.
Notre interprétation suit fortement~\cite{youtube:sala_armi:liegniczer, youtube:memag:liegniczer, farrell:pedagogy_liegnieczer:2014}.

Le style développé par Liegniczer est très proche du système de Liechtenauer par le fait qu'il repose sur le concept de liage et de winden.
Il se rapproche aussi du I.33 par le fait que la bocle sert avant tout à couvrir la main d'arme, comme cela est indiqué dès le début de la première technique.
Keith Farrell a écrit un article sur la pédagogie de Liegniczer~\cite{farrell:pedagogy_liegnieczer:2014} afin de montrer que la structuration des six techniques présente bien un système cohérent et puissant.

Dans notre interprétation nous nous éloignons du livre de Lindholm, Svard and Clements~\cite{lindholm:ringeck_others:2006} qui ne nous semble pas correspondre tout à fait à l'esprit du système de Liegniczer : par exemple dans la technique 2, l'attaquant et les défenseurs séparent leurs deux mains.
% TODO: citer la page
De même l'interprétation de H.\ Knight~\cite[part I]{knight:epee_bocle} ne nous a pas convaincu, entre autres à cause des positions faibles qu'il adopte.

Dans chaque technique \A attaque tandis que \D est globalement passif : il semble que ce dernier ne soit pas initié à l'escrime et réagisse simplement avec des déflexions.
De même qu'en épée longue, \A va réagir d'une manière différente selon la réaction de \D, en apportant une réponse adaptée au mouvement de \D (ressenti).
% \emph{fühlen}
Ainsi qu'il a été expliqué plus haut, la bocle doit toujours protéger la main d'épée en se plaçant du même côté de l'épée de \D.
De même si \D ne prend pas soin de couvrir son bras alors \A doit frapper cette cible plus facile à atteindre.
Le seul moment où la bocle n'est pas utilisé pour couvrir la main est lorsque \A se trouve près de \D : dans ce cas la bocle est utilisée pour frapper et/ou contrôler les bras/mains de \D (entre autres lorsque l'on change d'axe ou que l'n vient chercher une cible basse).

L'idéal pour pratiquer ces techniques est de décomposer les techniques en séquences où \A doit réagir correctement selon ce que fait \D – d'abord en augmentant progressivement la longueur de l'enchaînement, puis en variant d'une fois à l'autre.


\begin{technique}[Liegniczer 1]

\begin{enumerate}
	\item \A et \D démarrent dans la custodia 2 (épaule droite).
	
	\item \A lance un oberhau sur l'épaule droite de \D.
	
	\item \D pare avec un oberhau et prend le liage.
	
	\item \alt{Si \D est faible, \A estoque.}
		Si \D est relativement fort au liage, \A exécute un winden en levant les mains en bœuf (en supination ou en pronation).
	
	\item \alt{Si \D ne réagit pas, \A estoque.}
		Si \D pousse la lame vers l'extérieur alors \A quitte le liage et vient frapper \D de l'autre côté en passant sous la lame (\emph{schnappen}),
		tout en frappant les mains de \D avec la bocle pour éloigner la menace.
\end{enumerate}

Au temps 4) le plus simple est de lever directement la main en supination ; cette position est relativement faible si l'on n'utilise pas le pouce en appui comme expliqué dans l'introduction de ce chapitre.
Le \emph{schnappen} au temps 5) sera d'autant plus efficace que \D écarte fort la lame – à l'extrême si \D écarte fort dès le temps 3) alors il n'y a pas de winden.

Pour illustrer nos explications générales, \A n'a pas besoin d'exécuter l'ensemble de la technique si \D fait une erreur.
Par exemple si \D est faible au temps 3) alors \A peut estoquer directement.

Au temps 3) les rôles sont symétriques et \D peut prendre l'initiative, auquel cas les rôles s'inversent.
De ce point de vue une autre manière d'interpréter l'enchaînement est que \D attaque \A qui prend le liage avec un oberhau, et ensuite enchaîne.

\begin{figure}[htp]
	\centering
	\includegraphics{diagrammes/epee_bocle/liegniczer_1}
	\caption{Diagramme pour la technique 1 de Liegniczer.}
	\label{épée-bocle:fig:diagramme-liegniczer-1}
\end{figure}


\end{technique}


% talhoffer : Myers:LiegniczerBuckler, knight:epee_bocle
