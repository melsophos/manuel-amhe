\chapter{Épée et bocle}


% TODO: tenue avec le pouce

% exercices généraux


%%%%%%%%%%%%%%
\section{I.33}
%%%%%%%%%%%%%%


% type d'épée : Cinato : XVI, clubs allemands : XIV

% Royal Armouries, Tower Fechtbuch
Le traité MS I.33 (appelé aussi \emph{Walpurgis Fechtbuch} ou \emph{Liber de arte dimicatoria}) est le plus vieux manuel qui nous soit parvenu – il a été rédigé dans la décennie de 1320.
L'auteur serait un prêtre nommé Liutger.

La référence pour l'étude du I.33 est la traduction et le commentaire de Cinato et Suprenant~\cite{cinato:I33:2009}.
Kenner a écrit un manuel sur le sujet~\cite{kenner:I33:2014}.
Les images en couleur peuvent être trouvées sur le site de Wiktenauer~\cite{wiktenauer:I33}.


% TODO: custodia, contraria, liages supérieur/inférieur gauche/droit (inférieur = par en dessous)


%%%%%%%%%%%%%%%%%%%%
\section{Liegniczer}
%%%%%%%%%%%%%%%%%%%%


% TODO: add german words in techniques

Le traité de Liegniczer sur l'épée-bocle est très court et contient seulement six enchaînements.
On peut en trouver diverses transcriptions/traductions~\cite{ardamhe:liegniczer, farrell:liegnieczer, lindholm:ringeck_others:2006} ainsi que des interprétations~\cite{farrell:pedagogy_liegnieczer:2014, youtube:sala_armi:liegniczer, youtube:memag:liegniczer, lindholm:ringeck_others:2006, Myers:LiegniczerBuckler, knight:epee_bocle}.
Notre interprétation suit fortement~\cite{youtube:sala_armi:liegniczer, farrell:pedagogy_liegnieczer:2014}, ainsi que~\cite{youtube:memag:liegniczer}.

Le style développé par Liegniczer est très proche du système de Liechtenauer par le fait qu'il repose sur le concept de liage et de winden.
Il se rapproche aussi du I.33 par le fait que la bocle sert avant tout à couvrir la main d'arme, comme cela est indiqué dès le début de la première technique.
Keith Farrell a écrit un article sur la pédagogie de Liegniczer~\cite{farrell:pedagogy_liegnieczer:2014} afin de montrer que la structuration des six techniques présente bien un système cohérent et puissant.

Dans notre interprétation nous nous éloignons du livre de Lindholm, Svard and Clements~\cite{lindholm:ringeck_others:2006} qui ne nous semble pas correspondre tout à fait à l'esprit du système de Liegniczer : par exemple dans la technique 2, l'attaquant et les défenseurs séparent leurs deux mains.
% TODO: citer la page
De même l'interprétation de H.\ Knight~\cite[part I]{knight:epee_bocle} ne nous a pas convaincu, entre autres à cause des positions faibles qu'il adopte.

Dans chaque technique \A attaque tandis que \D est globalement passif : il semble que ce dernier ne soit pas initié à l'escrime et réagisse simplement avec des déflexions.
De même qu'en épée longue, \A va réagir d'une manière différente selon la réaction de \D, en apportant une réponse adaptée au mouvement de \D (ressenti).
% \emph{fühlen}
Ainsi qu'il a été expliqué plus haut, la bocle doit toujours protéger la main d'épée en se plaçant du même côté de l'épée de \D.
De même si \D ne prend pas soin de couvrir son bras alors \A doit frapper cette cible plus facile à atteindre.
Le seul moment où la bocle n'est pas utilisé pour couvrir la main est lorsque \A se trouve près de \D : dans ce cas la bocle est utilisée pour frapper et/ou contrôler les bras/mains de \D (entre autres lorsque l'on change d'axe ou que l'on vient chercher une cible basse).

L'idéal pour pratiquer ces techniques est de décomposer les techniques en séquences où \A doit réagir correctement selon ce que fait \D – d'abord en augmentant progressivement la longueur de l'enchaînement, puis en variant d'une fois à l'autre.


\begin{technique}[Liegniczer 1]

\A et \D démarrent dans la custodia 2 (épaule droite).

\begin{enumerate}
	\item \A lance un oberhau sur l'épaule droite de \D, en avançant la jambe droite.
	
	\item \D pare avec un oberhau, en avançant la jambe droite, et prend le liage.
	
	\item \alt{Si \D est faible, \A estoque.}
		Si \D est relativement fort au liage, \A exécute un winden en levant les mains en bœuf (en supination ou en pronation).
	
	\item \alt{Si \D ne réagit pas, \A estoque.}
		Si \D pousse la lame vers l'extérieur alors \A quitte le liage et vient frapper \D de l'autre côté en passant sous la lame (\emph{schnappen}),
		tout en frappant les mains de \D avec la bocle pour éloigner la menace.
\end{enumerate}

Au temps 3) le plus simple est de lever directement la main en supination ; cette position est relativement faible si l'on n'utilise pas le pouce en appui comme expliqué dans l'introduction de ce chapitre.
Le \emph{schnappen} au temps 4) sera d'autant plus efficace que \D écarte fort la lame – à l'extrême si \D écarte fort dès le temps 2) alors il n'y a pas de winden.

Pour illustrer nos explications générales, \A n'a pas besoin d'exécuter l'ensemble de la technique si \D fait une erreur.
Par exemple si \D est faible au temps 2) alors \A peut estoquer directement.

Au temps 2) les rôles sont symétriques et \D peut prendre l'initiative, auquel cas les rôles s'inversent.
De ce point de vue une autre manière d'interpréter l'enchaînement est que \D attaque \A qui prend le liage avec un oberhau, et ensuite enchaîne.

\begin{figure}[htp]
	\centering
	\includegraphics{diagrammes/epee_bocle/liegniczer_1}
	\caption{Diagramme pour la technique 1 de Liegniczer.}
	\label{épée-bocle:fig:diagramme-liegniczer-1}
\end{figure}


\end{technique}



\begin{technique}[Liegniczer 2]
\A démarre dans la custodia 5 (queue), \D dans la custodia 2 (épaule droite).

\begin{enumerate}
	\item \A lance un unterhau en avançant la jambe droite.
	
	\item \D prend le liage avec un oberhau en avançant la jambe droite.
	
	\item \alt{Si \D est faible, \A estoque.}
		Si \D est fort \A exécute un winden en levant les mains.
	
	\item \alt{Si \D ne réagit pas, \A estoque.}
		Si \D repousse la lame, \A exécute un duplieren tout en se décalant sur la gauche.
		La bocle permet de coincer l'épée de \D.
	
	\item \D se protège du coup et \A vient frapper les jambes du vrai tranchant (dans l'intérieur de \D).
\end{enumerate}

Il est possible d'inverser les temps 1) et 2) si l'on voit l'unterhau de \A comme un liage (inférieur).

\D se trouve naturellement jambe droite devant et il s'agit de la cible principale, alors que le traité indique que \A vise la jambe gauche : une interprétation est que \D peut toujours reculer la jambe droite, mais pas la gauche et si l'on vise cette dernière (ce qui est possible comme le coup est à l'intérieur) alors on est sûr de toucher quelque chose.

\end{technique}

% talhoffer : Myers:LiegniczerBuckler, knight:epee_bocle


\begin{technique}[Liegniczer 3]
\A et \D démarrent dans la custodia 2 (épaule droite).

\begin{enumerate}
	\item \A lance un oberhau en avançant la jambe droite.
	
	\item Quand \D se protège, \A laisse tomber la pointe de son épée sous celle de \D et la relève ensuite en battant fortement l'épée de \D avec le faux tranchant.
		\A se sert de l'élan pour frapper (du vrai tranchant) la tête de \D du côté droit en avançant la jambe gauche (wechselhau).
	% mouvement de hanches pour le balayage : permet d'éviter l'attaque ?
	
	\item \D se protège et \A exécute un winden en tournant la main en supination.
	
	\item \alt{Si \D ne fait rien, \A estoque.}
		\D dévie l'estoc et \A vient frapper la jambe droite du vrai tranchant.
		La bocle écarte les mains de \D.
\end{enumerate}

Au temps 2) le balayage est exécuté avec le faux tranchant car cela est plus rapide qu'avec le vrai, qui nécessiterait plusieurs rotations du poignet.
Il est aussi important de passer la bocle du côté droit du bras d'arme lors de la frappe, autant pour protéger le bras que pour laisser ouverte la ligne basse d'attaque.

La position en 3) n'est pas très forte : à notre avis l'idée est de surprendre en changeant la ligne d'attaque (technique~\ref{struct:tech:changement-ligne}).

Dans ce contexte la première attaque est une feinte.
Une autre interprétation est possible si \A commence dans la garde du fou : dans ce cas la déflexion en 2) est utilisée contre un oberhau de \D (pour une garde à gauche le balayage se fera naturellement avec le vrai tranchant).

\end{technique}


\begin{technique}[Liegniczer 4]
\A et \D démarrent dans la custodia 2 (épaule droite).

\begin{enumerate}
	\item \A lance un mittelhau (avec le vrai tranchant) à droite en avançant la jambe droite.
	
	\item \D se protège et \A frappe de l'autre côté avec un second mittelhau (vrai tranchant) en avançant la jambe gauche.
	
	\item \D se protège et \A lance un coup crânien (vrai tranchant) en avançant la jambe droite.
	
	\item \D se protège et \A fait glisser l'épée le long de la bocle de \D pour estoquer le ventre.
		\A garde sa bocle haute pour empêcher \D de baisser ses mains et pour protéger sa tête.
\end{enumerate}

L'idée de l'enchaînement est d'exécuter une série rapide de plusieurs coups qui vise à faire progressivement monter les défenses de \D afin d'attaquer bas à la fin.
Cela marche d'autant mieux si l'on a pu faire entrer \D dans un schéma, par exemple en lançant plusieurs mittelhau à la suite (par exemple avec un enchaînement droit-bas puis gauche-haut).

Les temps 1) et 2) constituent un double zwerchau, où la différence par rapport à l'épée longue est que la première frappe est faite avec le vrai tranchant.
Il est en effet un peu plus difficile de tourner la main sur la garde pour une épée une main puisque la main gauche n'est pas là pour stabiliser, et en gardant le pouce en appui sur les quillons il n'est pas possible de frapper avec le faux tranchant au temps 2) et 3).

Au temps 4) \A ne doit pas ramener son épée en arrière pour estoquer car il perdrait le centre (et du temps).

\end{technique}


\begin{technique}[Liegniczer 4 (variante)]
\A et \D démarrent dans la custodia 2 (épaule droite).

\begin{enumerate}
	\item \A lance un mittelhau (avec le faux tranchant) à droite en avançant la jambe droite.
	
	\item \D se protège et \A frappe de l'autre côté avec un second mittelhau (vrai tranchant) en avançant la jambe gauche.
	
	\item \D se protège et \A lance un coup crânien (faux tranchant) en avançant la jambe droite.
	
	\item \D se protège et \A fait glisser l'épée le long de la bocle de \D pour estoquer le ventre.
		\A garde sa bocle haute pour empêcher \D de baisser ses mains et pour protéger sa tête.
\end{enumerate}

La différence avec la variante précédente est que deux des attaques – temps 1) et 3) – se font avec le faux tranchant, et ainsi la première attaque ressemble plus au zwerchau allemand que dans la première version.
Cet enchaînement est possible à condition que le pouce soit placé au niveau de la croix quillons/poignée, comme en épée longue, et que l'on conserve cette position tout le long.

Avec cette tenue la frappe en 3) avec le faux tranchant est très naturelle et plus rapide que si l'on devait ramener le vrai tranchant, et l'on dispose aussi d'un plus grand angle pour estoquer derrière la bocle (du fait de la position du poignet).

% Léo
\end{technique}

