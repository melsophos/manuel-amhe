\chapter{Épée et bocle}

\section{I.33}

% Royal Armouries
Le traité MS I.33 (appelé aussi \emph{Walpurgis Fechtbuch} ou \emph{Liber de arte dimicatoria}) est le plus vieux manuel qui nous soit parvenu – il a été rédigé dans la décennie de 1320.
L'auteur serait un prêtre nommé Liutger.

La référence pour l'étude du I.33 est la traduction et le commentaire de Cinato et Suprenant~\cite{cinato:I33:2009}.
Kenner a écrit un manuel sur le sujet~\cite{kenner:I33:2014}.
Les images en couleur peuvent être trouvées sur le site de Wiktenauer~\cite{wiktenauer:I33}.


\section{Liegniczer}

Le traité de Liegniczer sur l'épée-bocle est très court et contient seulement six enchaînements.
On peut en trouver diverses traductions~\cite{ardamhe:liegniczer, farrell:liegnieczer, lindholm:ringeck_others:2006} ainsi que des interprétations~\cite{youtube:sala_armi:liegniczer, youtube:memag:liegniczer} (le document~\cite[part I]{knight:epee_bocle} n'est pas excellent mais fourni un aide-mémoire correct).
Keith Farrel a écrit un article sur la pédagogie de Liegniczer~\cite{farrell:pedagogy_liegnieczer:2014} afin de montrer que la structuration des six techniques présente bien un système cohérent et puissante.


