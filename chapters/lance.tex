\chapter{Lance}


\section{Généralités}


% Romain
Si une épée fait face à une lance, c'est en général une bonne idée de glisser le long du manche pour avoir les doigts du lancier.

% Romain
La lance est une arme qui doit être très mobile et offensive.
Pour cette raison il n'est souvent pas intéressant de verrouiller la main (contre la hanche ou a tête) afin de pouvoir changer rapidement de direction d'attaque.
De plus en bougeant souvent la lance l'adversaire aura peu l'occasion de la dégager via un enroulé ; de même il sera beaucoup plus simple de ramener la pointe dans le centre si l'adversaire porte un coup puissant contre le manche.
Au contraire le verrouillage pourra être utile dans les formations serrées (par exemple de piquiers~\footnotemark).
\footnotetext{Ceux-ci verrouillaient le manche en le posant sur leur ceinture, qui était portée plus haut que nous.}

% TODO: gardes

\begin{coup}[Attaque basse]
En garde basse, la main avant est la même que celle du pied avant.
Avancer le pied avant et faire coulisser le manche dans la main avant pour estoquer.
\end{coup}


\begin{coup}[Attaque haute]
En garde haute, la main avant est la même que celle du pied avant.
Avancer le pied avant et faire coulisser le manche dans la main avant pour estoquer.
La lance a une direction descendante.
\end{coup}

Cette seconde attaque est utile pour piquer derrière un bouclier, à la tête.

\begin{technique}[Retour en garde haute]
Après une attaque basse, monter la main arrière et descendre sur les appuis.
La main avant se place sous la hampe pour protéger les doigts (si prise bâton, placer le poing).

Une amélioration consiste à placer l'avant bras.
\end{technique}

Dans la garde précédente où l'avant-bras est placé contre le manche, avec une hallebarde on est alors très bien placé pour frapper avec le tranchant par en haut.

\begin{technique}[Retour en garde basse]
Après une attaque haute, faire un tour avec la pointe en tournant la main gauche (arrière) dans le sens antihoraire (comme un rameur), et revenir en garde basse en ramenant la main en bas.
Le tour permet de reprendre le centre, sinon la lance peut se retrouver à l'extérieur.
Et si l'autre n'est pas prudent il est possible de le planter direct.
\end{technique}


\section{Lance contre épée longue}

% Romain
Les épées longues utilisées dans ce genre de combat (au moins au début) étaient d'une taille similaire à celles des épées longues italiennes.
Ce n'est que plus tard qu'elles évolueront vers les épées à deux mains (comme par exemple l'espadon).

\index{ricasso}
Ces lames comportent un ricasso, c'est-à-dire une partie émoussée (et éventuellement recouverte de cuir) située au-delà des quillons.
Cette partie se termine par des ergots qui permettent de délimiter les différentes zones, de servir de point de repère (quand on plaque la lance au sol, on peut savoir plus facilement à quelle hauteur elle se trouve) et d'accrocher les manches.
Le ricasso permettait de raccourcir la prise sur la lame tout en offrant une meilleure sécurité : en effet dans le type de combat dont il est question, il pouvait arriver que l'on se retrouve en contact avec son épée – par exemple en parant en prime avec le haut de la lame contre l'épaule.


\begin{technique}

\A est armé de la lance.

\begin{enumerate}
	\item \A porte un coup de lance bas.
	
	\item \D sort sur son pied gauche et pare en sixte.
	
	\item \D avance le pied droit et appuie sur le manche de la lance avec son épée, direction à 90° pour l'enfoncer dans le sol.
	Les mains sont loin devant pour gêner \A.
	
	\item \D avance le pied gauche et avance son tranchant contre D.
\end{enumerate}

Parer avec le faux tranchant permet d'avoir une rotation plus naturelle au temps suivant. 
À la fin \D doit être prêt à suivre \A s'il recule.

Source : Romain.
\end{technique}


\begin{technique}

\A est armé de la lance.

\begin{enumerate}
	\item \A porte un coup de lance haut.
	
	\item \D engage son épée en bœuf~\footnotemark, pied gauche en avant.
	\footnotetext{À voir plus comme un estoc dans le vide que comme une quinte.}
	
	\item \D avance son bras gauche sous son épée pour attraper le manche.
	
	\item \D donne un coup dans les tibias de \A.
\end{enumerate}

L'inertie permet d'avoir un coup efficace même avec une main (en demi-armure typique de l'époque les tibias ne sont pas protégés).
Il faut être haut pour ne pas permettre à l'autre d'avoir la tête.

Source : Romain.
\end{technique}


\begin{technique}

\A est armé de la lance.

\begin{enumerate}
	\item \A porte un coup de lance haut.
	
	\item \D engage son épée en bœuf, pied gauche en avant.
	
	\item Avancer le pied droit et frapper parallèlement au manche. 
\end{enumerate}

Ce coup permet d'avoir au moins les mains, sinon la tête, et de gêner \A.
Il peut se faire sur une mauvaise parade, par exemple où l'épée est contre l'épaule.

Source : Romain.
\end{technique}


\begin{technique}

\A est armé de l'épée longue.

\begin{enumerate}
	\item \A prend le contact avec la lance en quarte.
	
	\item \A passe par dessus la lance en octave.
	
	\item \A avance le pied droit.
	
	\item \A avancer le pied gauche et frappe.
\end{enumerate}

Cette technique est très surprenante car le coup arrive du côté opposé à celui prévu.

Il est possible de passer en quarte après avoir paré un coup d'estoc en tierce, par exemple.

Source : Romain.
\end{technique}


\section{Lance contre épée-bouclier}

Certaines techniques d'épée longue peuvent aussi s'utiliser.

% Romain
Contre un fantassin armé d'un bouclier, un lancier a trois cibles :
\begin{enumerate}
	\item la tête, par dessus le bouclier ;
	\item les jambes du côté du bouclier (zone définie par là où le bouclier ne peut pas voir) ;
	\item le ventre, entre l'épée et le bouclier.
\end{enumerate}


\begin{garde}
Contre une attaque de lance à la tête, \D tient le bouclier en avant (mais pas complètement, sinon D peut passer en dessous), et place son épée au dessus.

Source : Romain.
\end{garde}


\begin{technique}
Quand \D pare un coup de lance au niveau du ventre, il doit appuyer avec son bouclier sur le manche pour enfoncer la lance dans le sol.

Source : Romain.
\end{technique}



\begin{exercice}
\A a la lance.

\begin{enumerate}
	\item \A fait un estoc haut en visant le visage.
	
	\item \D se protège.
	
	\item \A revient en garde et estoque \D au ventre, à l'ouverture apparue après qu'il ait levé le bouclier.
	% avec la technique ...
	
	\item \D se protège.
\end{enumerate}

Source : Romain.
\end{exercice}



\begin{technique}
\A est armé de l'épée et du bouclier.

\begin{enumerate}
	\item \A vient prendre le contact avec la lance par en dessous, sans trop avancer.
	
	\item \A avance loin le bouclier pour dévier la lance sur le côté gauche, tout en faisant un pas.
	La partie latérale du bouclier est en contact, pas le haut.
	Avancer le pied.

	\item \A frappe \D au tibia.
\end{enumerate}

Le premier mouvement est comme un un estoc dans le vide.

Source : Romain.
\end{technique}


\section{Lance contre autre arme}


\begin{technique}
\D est armé d'un sabre d'abordage.

\begin{enumerate}
	\item \A fait une attaque haute.
	
	\item \D Pare en quinte, pointe vers l'avant (main du côté droit).
	
	\item \D fait un tour complet autour de sa jambe droite en ramenant le sabre contre son côté, pointe derrière son dos.
	
	\item \D estoque \A au ventre.
\end{enumerate}

Il s'agit d'une des rares techniques avec un tour sur soi-même.

Source : Romain.
\end{technique}



