\chapter{Lance}


\section{Généralités}


% Romain
Si une épée fait face à une lance, c'est en général une bonne idée de glisser le long du manche pour avoir les doigts du lancier.


\section{Lance contre épée longue}




\section{Lance contre épée-bouclier}

Certaines techniques d'épée longue peuvent aussi s'utiliser.

% Romain
Contre un fantassin armé d'un bouclier, un lancier a trois cibles :
\begin{enumerate}
	\item la tête, par dessus le bouclier ;
	\item les jambes du côté du bouclier (zone définie par là où le bouclier ne peut pas voir) ;
	\item le ventre, entre l'épée et le bouclier.
\end{enumerate}


\begin{garde}
Contre une attaque de lance à la tête, \D tient le bouclier en avant (mais pas complètement, sinon D peut passer en dessous), et place son épée au dessus.

Source : Romain.
\end{garde}


\begin{technique}
Quand \D pare un coup de lance au niveau du ventre, il doit appuyer avec son bouclier sur le manche pour enfoncer la lance dans le sol.

Source : Romain.
\end{technique}



\begin{exercice}
\A a la lance.

\begin{enumerate}
	\item \A fait un estoc haut en visant le visage.
	
	\item \D se protège.
	
	\item \A revient en garde et estoque \D au ventre, à l'ouverture apparue après qu'il ait levé le bouclier.
	% avec la technique ...
	
	\item \D se protège.
\end{enumerate}

Source : Romain.
\end{exercice}



\begin{technique}
\A est armé de l'épée et du bouclier.

\begin{enumerate}
	\item \A vient prendre le contact avec la lance par en dessous, sans trop avancer.
	
	\item \A avance loin le bouclier pour dévier la lance sur le côté gauche, tout en faisant un pas. La partie latérale du bouclier est en contact, pas le haut. Avance le pied.

	\item \A frappe \D au tibia.
\end{enumerate}

Le premier mouvement est comme un un estoc dans le vide.

Source : Romain.
\end{technique}



\section{Lance contre autre arme}


\begin{technique}
\D est équipé d'un sabre d'abordage.

\begin{enumerate}
	\item \A fait une attaque haute.
	
	\item \D Pare en quinte, pointe vers l'avant (main du côté droit).
	
	\item \D fait un tour complet autour de sa jambe droite en ramenant le sabre contre son côté, pointe derrière son dos.
	
	\item \D estoque \A au ventre.
\end{enumerate}

Il s'agit d'une des rares techniques avec un tour sur soi-même.
\end{technique}


