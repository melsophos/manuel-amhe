\chapter{Combat à mains nues}

\section{Généralités}

Il est beaucoup plus facile de faire une clé sur un bras tendu, donc il ne faut jamais finir le bras tendu après une frappe.


\section{Mains nues contre couteau}


\begin{exercice}

\begin{enumerate}
	\item \A donne un coup de couteau au ventre de \D.
	
	\item \D esquive sur le côté en venant couvrir (main droite ou gauche).
\end{enumerate}

\end{exercice}



\begin{technique}

\begin{enumerate}
	\item \A donne un coup de couteau au ventre de \D.
	
	\item \D esquive sur le côté gauche et vient contrôler l'arme de la main gauche.
	\D se retrouve en garde face à \A.
	
	\item \D a plusieurs solutions pour poursuivre, par exemple :
	\begin{enumerate}
		\item frapper du poing droit dans le flanc ;
		
		\item frapper \A sous le menton avec la paume droite (passer sous son bras) ;
		
		\item tourner la tête de \A vers sa gauche en appuyant sur le côté droit de son visage avec la paume de la main droite (passer sous son bras), et passer la jambe droite derrière \A pour le projeter au sol.
	\end{enumerate}

\end{enumerate}

L'intérêt de contrôler l'arme de la main gauche est d'avoir la main droite à distance de frappe, ainsi que pour généraliser au cas où la main droite tient une arme.

Au point (3a) il est important d'engager l'épaule pour protéger le visage des coups de \A.

Source : Romain.

\end{technique}


\begin{technique}

\begin{enumerate}
	\item \A donne un coup de couteau au ventre de \D.
	
	\item \D esquive sur le côté gauche et vient contrôler l'arme de la main droite.
	
	\item \D avance sa jambe gauche et place sa main gauche dans le creux du coude (droit) de \A pour plaquer le bras contre son corps.
	
	\item \D repousse en arrière le bras de \A grâce à sa main droite, tandis que sa main gauche vient attraper son propre coude droit afin de verrouiller la prise.
\end{enumerate}

Au point (4) \D doit tenir son coude et pas autre chose : cela permet de bloquer le seul angle d'attaque où \A aurait pu frapper.

Source : Romain.

\end{technique}


\begin{technique}

\begin{enumerate}
	\item \A donne un coup de couteau au ventre de \D.
	
	\item \D saute sur le côté gauche et vient contrôler l'arme de la main droite.
	
	\item \D donne un coup de pied (droit) au ventre de \A.
\end{enumerate}

Le coup de pied au point (3) peut être exécuté plus rapidement si \D n'a pas posé le pied droit au point (2).

Source : Romain.

\end{technique}


\begin{technique}

\begin{enumerate}
	\item \A donne un coup de couteau au ventre de \D.
	
	\item \D saute sur le côté gauche et vient contrôler l'arme de la main droite.
	
	\item \D donne un coup de pied/tibia (gauche) derrière le genou droit de \A.
\end{enumerate}

Après le dernier temps \D peut profiter de sa position pour faire une clé avec sa main gauche.
L'intérêt d'avoir frapper dans le creux du genou est de pouvoir appuyer dessus pour emmener facilement \A à terre.

Source : Romain.

\end{technique}


\begin{technique}

\begin{enumerate}
	\item \A donne un coup de couteau au ventre de \D.
	
	\item \D esquive sur le côté gauche et vient contrôler l'arme de la main droite.
	
	\item \D donne frappe du poing gauche sur le flanc (droit) de \A (en avançant le pied gauche).
	En même temps il ferme sa prise avec la main droite sur le bras droit de \A.
	
	\item \D se retourne et vient percuter \A avec son épaule gauche.
	
	\item \D peut :
	\begin{enumerate}
		\item soit mettre son bras gauche en travers de la poitrine de \A et tirer avec la main droite pour faire basculer \A par-dessus son bras gauche ;
		
		\item soit attraper le bras de \A avec les deux mains et tirer.
	\end{enumerate}

\end{enumerate}

Le coup de poing et la frappe de l'épaule aux temps (3) et (4) servent à déséquilibrer \A afin de le faire tomber.
En particulier le foie se trouve vers le bas du flanc droit et fait une excellente cible.

Au point (4) \D doit veiller à garder l'arme de \A loin de lui, à un endroit où \A ne peut pas se dégager ni la retourner.

Si \A est trop grand pour être mis à terre, \D peut ramener ses mains pour lui planter son couteau dans les jambes.

Source : Romain.

\end{technique}


Les exercices suivants permettent de préparer l'épée longue italienne (en particulier le jeu court).

\begin{exercice}

\begin{enumerate}
	\item \A et \D sont en garde, dos de la main droite en contact.
	
	\item Quand il le sent, \A passe sur la gauche et vient poser son avant-bras contre l'épaule de \D.
	
	\item \D vérifie si \A est stable en le poussant un peu.
\end{enumerate}

À faire en se déplaçant.

\A doit veiller à ne pas se prendre le coude de \D en passant.
De plus \A ne doit pas vraiment tirer ou attraper le bras de \D afin de ne pas donner d'indications.

L'intérêt de l'exercice est d'entrer dans la distance quand \A sent une poussée de la part de \D.

Source : Romain (ENS).

\end{exercice}


\begin{exercice}

\begin{enumerate}
	\item \A et \D sont en garde, dos de la main droite en contact.
	
	\item Au moment de son choix, \D exerce une poussée ou retire sa main pour mettre une cible.
	
	\item Si \D pousse, \A entre dans la distance comme dans l'exercice précédent. Si \D met une cible, \A vient toucher en fente.
\end{enumerate}

À faire en se déplaçant.

La raison est que \A ne peut pas prévoir la réaction de \D : ainsi même s'il est décidé à entrer en lutte, il doit garder en seconde intention de frapper si \D lui en donne la possibilité.

Source : Romain (ENS).

\end{exercice}


\begin{exercice}

\begin{enumerate}
	\item \A et \D sont en garde. \A attaque \D à l'épaule.
	
	\item \D laisse passer court en reculant un peu sa jambe avant.
	
	\item \D fait un pas à gauche et vient heurter le coude ou l'épaule droite de \A avec sa main gauche.
	
	\item \D avance le pied droit et vient frapper l'épaule gauche de \A avec sa main droite.
\end{enumerate}

À faire en se déplaçant.

Au temps (3) \D doit pousser dans une direction à 45°, sinon \A peut facilement changer de garde et se repositionner (a fortiori dans les cas extrêmes de 0° et 90° – faire le test à l'épée longue).
De plus \D doit pousser au niveau du coude ou de l'épaule, mais pas entre, car l'effet de levier serait beaucoup trop faible et \A pourrait résister.

Source : Romain (ENS).

\end{exercice}
