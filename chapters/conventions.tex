\chapter{Conventions}
\label{app:conventions}


Dans les exercices chaque point correspond à un temps. Ainsi même si aucune conjonction n'est précisée entre deux phrases, il faut comprendre que les deux actions doivent avoir lieu en même temps. Exemple : « Faire un pas à droite. Porter un estoc. » signifie que l'estoc a lieu en même temps que le pas à droite.

Les crochets servent à indiquer comment arrêter l'exercice à un temps donné.

\A désigne l'attaquant (c'est-à-dire le premier à agir) et \D le défenseur.

On appellera~\footnote{Ces dénominations viennent des attaques avec une épée courte. Elles ne sont pas aussi visuelles pour les autres armes – hast, épée longue… – mais elles permettent d'avoir une dénomination unique.} diagonale droite la diagonale NE–SO, et diagonale inverse la diagonale NE–SO.
\index{diagonale!droite}
\index{diagonale!inverse}

Les rotations sont précisées avec les termes direct/antihoraire et indirect/horaire.

On utilisera le terme de croix pour une épée afin de désigner la croix formée par la poignée et la garde.
Celle-ci servira de cible pour certains exercices (l'épée en nylon étant tenue par la lame).

Dans certains cas les temps sont extrêmement décomposés, mais une exécution fluide de la technique conduira à "fusionner" certains temps.

Nous utiliserons parfois un vocabulaire tiré de styles proches afin de préciser le mouvement.
Nous utiliserons parfois le terme français, parfois le terme original selon ce qui nous semble le plus précis.


